\documentclass[12pt]{report}
\usepackage{epsfig}
\usepackage{UPBcomsci}  
\usepackage{UPBnotations}
\usepackage{ tabularx}
\usepackage{comment}
\usepackage{amssymb, graphicx, amsmath}
\usepackage{subfigure} 
\usepackage{subcaption} 
\usepackage[hidelinks]{hyperref}
\usepackage{float}
\usepackage{algorithm}
\usepackage{algorithmic}
\usepackage{moreverb}
\usepackage{cite}
\usepackage{tabularray}
\usepackage{booktabs}
\usepackage{float,framed,bm}
\usepackage{longtable}
\usepackage{booktabs}
\usepackage{listings}

\usepackage{amsmath}
\usepackage{amssymb}
%\usepackage{rotating}

\begin{document}

\title{Assessing Respiratory Infectious Disease Transmission in Public School Settings: An Agent-Based Modeling Approach at Pines City National High School}

\authorfirst{Myla Jean C.}
\authorlast{Legaspi}
%\authorext{Jr.} \authorexttrue %make this a comment if you do not have an extension name e.g. Jr., II, III, etc.

\degree{Bachelor of Science in Computer Science} %enter the specific degree
\thesistype{Special Problem} %enter whether Master's Thesis or Doctoral Dissertation for this is Special Problem
\thesisdocument{Special Problem} %enter whether Thesis or Dissertation or Special Problem

%\specialization{Computer Science} \specializetrue %make this a comment if you are Ph.D. or BS student or no specialization

\defensedate{May 21, } %indicate the month, day and year the research was defended
\submitmonth{May} %indicate the month the manuscript was submitted to the office
\submityear{2024} %indicate the year the manuscript was submitted to the CS office

%% Change to name of Adviser
\adviser{Joel  M. Addawe, Ph.D.}%main adviser
%\reader{Joel  M. Addawe, Ph.D.}

%%%%%%%%%%%%%%%%%%%%%%%%%%%%%%%%%%%%%%%%%%%%%%%%%%%%%%%
%% highlight \bothabsenttrue if the student has only one adviser and one reader
%% otherwise comment \bothabsenttrue and highlight \coadviser or \coreader which 
%% ever is applicable. In this case, the package assumes either you have 1 Adviser 
%% and 2 Readers, or 2 Advisers and 1 Reader. 
  
\bothabsenttrue

%\coadviser{co-Adviser} 
%\coadvisetrue  
% Make the \coadvisetrue as a comment if there is no adviser

%\coreader{Co-Reader's Name}  %leave this blank if there is a coadviser

%%%%%%%%%%%%%%%%%%%%%%%%%%%%%%%%%%%%%%%%%%%%%%%%%%%%%%%


\chairman{Gilbert R. Peralta, Ph.D.} 
%\dean{Rosemary M. Guttierez, Ph.D.}

\acknowledge{\hspace{6 mm} }

\abs{In the Philippine educational landscape, congestion and overpopulation in classrooms present a significant challenge in curbing the transmission of respiratory infectious diseases. The close proximity of students and staff, stemming from limited space and high student-to-classroom ratios, escalates the risk of disease dissemination within school premises. This paper employs Agent-Based Modeling to develop a framework for understanding the dynamics of respiratory infectious disease transmission within school environments, particularly focusing on Pines City National High School. It identifies the effects of various key factors influencing disease spread, including protection rates, initial exposure rates, and the implementation of non-pharmaceutical interventions such as lockdowns and quarantine areas. The simulation employs QGIS and Autocad to create the structure of the shapefile or simulation area, and the GAMA Platform for the actual simulation. In scenarios where there’s a varying protection rate, it shows that starting with 0\% protection, the infection rate is 93.1\%, but it decreases to 25\% with 100\% protection. Each 25\% increase in protection results in approximately a 20\% decrease in infection rate. The study underscores the significance of comprehensive immunity measures, with higher protection rates associated with fewer infections. In addition, the impact of initial exposure rates on respiratory infectious disease transmission shows that infection rates rise steeply with increased initial exposure rates: 61.20\% at 25\% exposure (+3.63\%), 74.20\% at 50\% exposure (+16.63\%), and peak at 80.20\% at 75\% exposure (+22.63\%). This concludes that as the initial exposure rate increase, the percentage of the infected individuals also increases. Implementation of Lockdown and Quarantine Area also shows that without lockdown and quarantine, the rate peaks at 55.01\%. With lockdown but no quarantine, it drops to 28.54\% (-26.47\%). When only a quarantine area is implemented, the rate is 30.16\% (-24.85\%). With both lockdown and quarantine, it further decreases to 28.67\% (-26.34\%). These findings highlight the effectiveness of containment strategies in controlling infection rates during a pandemic.
\\
 }

%\biodatatrue 

\beforepreface 

\tableofcontents

\newpage

\acknowledgetrue
\acknowledgepage
\indent Completing this special problem has been a journey filled with the support and encouragement of many individuals, without whom it would not have been possible. I extend my heartfelt appreciation to the following individuals:

First and foremost, I express my deepest gratitude to my advisor, Mr. Joel M. Addawe, Ph.D. Your unwavering support, guidance, and insights have been instrumental throughout this process. Your patience, understanding, and constructive feedback have truly enhanced the quality of this paper.

To my adviser, Ma'am Ash, thank you for your guidance and support from my freshman years until my remaining months at the university. Your encouragement has been invaluable.

To my family—Mama, Papa, MJ, and Jamy—thank you for your unwavering support and countless sacrifices. Your constant encouragement and the warmth of home provided me with the strength and solace needed to navigate through the challenges of this journey.

I am immensely grateful to Chezka, my former partner in this Special Problem. Your collaboration, knowledge sharing, and exchange of ideas were integral to the completion of this paper. I also extend my gratitude to my ABG friends—Lance, Gab, Maj, Pat, and Rosseau—for their significant contributions and companionship throughout the process. Working on this paper with all of you never made me feel alone amidst the struggles.

To Chien, who has been my senior and a source of unwavering support since before this study began, thank you for your helpful guidance.

Lastly, I offer my deepest gratitude to God, whose steadfast presence has been my guiding light and greatest source of strength throughout this journey. In moments of doubt and weariness, His grace sustained me and gave me the courage to persevere.

Each of you has played a significant role in shaping this work, and for that, I am truly thankful.

\abstractpage
\listoftables
\listoffigures

\afterpreface

\chapter{Introduction}
\label{chap:intro}


\section{Background of the Study}

Since the first COVID-19 case was recorded in the Philippines on January 30, 2020, discussions about implementing a lockdown ensued. By the second week of March 2020, localized lockdowns were initiated, leading to the temporary closure of academic institutions by the Philippine government\cite{worldhealthorganization_2023_who}. This closure affected nearly 24 million Filipino students, compelling them to adapt to alternative modes of instruction such as modular and distance learning. However, challenges arose as a significant portion of the Filipino population lacked access to reliable internet connections, disparities in educational access and quality.

Despite school closures being a preventive measure to curb the spread of the virus, it became increasingly evident that such actions had adverse effects on students' learning. Reports indicated difficulties in maintaining engagement and academic progress, particularly among students from marginalized communities. The closure also highlighted the importance of schools not only as centers of education but also as providers of essential services such as nutrition and psycho social support.

Over the course of two years, from the initial one-week suspension to an extended period of lockdown, it became increasingly apparent that schools played a significant role in the transmission of COVID-19, particularly in densely populated areas with limited infrastructure for infection control \cite{jones_2017_school}.

As the fear of COVID-19 gradually diminished, there was a phased reopening of schools. Safety measures including vaccinations and Non-Pharmaceutical Interventions (NPIs) such as social distancing and lockdowns were implemented to mitigate transmission. However, some regions experienced a rise in infections following the reopening of schools. For instance, in Spain, the second wave of infections began before the commencement of the school year in September 2020, underscoring the complexities of managing public health risks in educational settings \cite{soriano_2021_main}.

Given these circumstances, it raises critical questions about the readiness of Philippine public schools in managing another outbreak of a respiratory infectious disease, including the adequacy of infrastructure, resources, and policies to safeguard the health and well-being of students, teachers, and staff amidst evolving public health challenges.

\section{Statement of the Problem}

In the Philippine educational context, the prevalence of congestion and overpopulation within classrooms presents a significant challenge in preventing the transmission of respiratory infectious diseases. The close proximity of students and staff due to limited space and high student-to-classroom ratios increases the risk of disease spread within school environments \cite{g_2023_facility}. Despite efforts to implement preventive measures such as vaccinations, mask-wearing, and promoting physical distancing, the persistent issue of congestion and overpopulation exacerbates the vulnerability of the school community to respiratory infectious diseases.

Therefore, there is a critical need to address the issue of congestion and overpopulation within Philippine classrooms to mitigate the risk of respiratory infectious disease transmission. By identifying effective strategies to alleviate congestion and optimize classroom space utilization, schools can create safer environments that promote the health and well-being of students and staff \cite{g_2023_facility}. Additionally, exploring innovative approaches to complement existing preventive measures is essential in ensuring the overall safety and resilience of the Philippine school community against respiratory infectious diseases

\indent \indent The data for population were collected from Pines City National High School. In this study, it aims to answer the following objectives:

\section{Objective of the Study}
\vspace{8pt} %
\indent \indent In this study, the model will categorize the population into five compartments: susceptible (\textbf{S}), exposed (\textbf{E}), infected (\textbf{I}), and recovered (\textbf{R}) agents. For each experiment, the model will track the number of individuals in each compartment at every iteration of the simulation (t). Different scenarios will be implemented in each simulation to test their respective effects. Through this approach, the study aims to address the following general and specific objectives.

\subsection{General Objective of the Study}

\indent \indent 
 This study aims to comprehensively understand the dynamics of respiratory infectious disease transmission within the school environment and identify key factors influencing disease spread, including individual protection rates, exposure rates, and the implementation of lockdown and quarantine areas, in order to inform the development of effective prevention and control strategies.
 
\subsection{Specific Objective of the Study}

\indent \indent
Specifically, this study aims to :
\begin{itemize}
	
	\item[1.] Assess the impact of Varying Protection Rates (0\%, 25\%, 50\%, 75\%, 100\%) on Disease Transmission within the PCNHS environment; 
	\item [2.] Examine how having different initial percentages of exposed individuals (0\%, 25\%, 50\%, 75\%) within the PCNHS environment contribute to the overall respiratory infectious disease transmission rates; and,
	\item[3.] Evaluate the effect of lockdown and quarantine area implementation on the transmission of respiratory infectious disease within the PCNHS environment 
\end{itemize}


\section{Significance of the Study}

Through agent-based modeling and compartmental modeling, we can replicate the behavior of respiratory infectious disease such as the COVID-19 virus within a school environment. This study focuses on investigating the transmission dynamics within the campus of PCNHS.

The study holds significant importance in understanding and addressing respiratory infectious disease transmission within school environments. 
Firstly, by assessing the impact of varying protection rates ranging from 0\% to 100\%, the study provides valuable insights into the effectiveness of immunity measures in controlling disease spread. This information is crucial for policymakers and school administrators in implementing appropriate preventive measures to safeguard the health of students and staff. 
Additionally, the examination of different initial percentages of exposed individuals within the school environment contributes to understanding overall disease transmission rates, aiding in the assessment of outbreak risks and the determination of necessary interventions. Furthermore, the evaluation of lockdown and quarantine measures within the school setting sheds light on the effectiveness of non-pharmaceutical interventions in controlling disease transmission. This information is instrumental in developing strategies to minimize disease spread and maintain a safe learning environment. 
Overall, the study's findings have significant implications for public health policy and school management practices, providing evidence-based insights into effective strategies for preventing and managing respiratory infectious diseases within school settings, ultimately contributing to efforts to protect the health and well-being of students, staff, and the broader community.

\section{Scope and Limitation}

The study focuses on assessing respiratory infectious disease transmission specifically within public school settings, with a particular emphasis on PCNHS, which serves as the virtual environment for the simulation. Utilizing an agent-based and compartmental modeling approach, the study allows for the simulation of disease transmission dynamics within the school environment. It explores the impact of parameters such as protection rate, exposure rate, and the implementation of lockdown and quarantine measures on disease transmission. Through scenario analyses and simulations, the study aims to provide insights into the effectiveness of various intervention strategies in mitigating disease spread within the school setting.

However, it's important to note several limitations. Firstly, the data and findings of this study may only be directly applicable to the specific context of PCNHS. Differences in school layout, student demographics, and local environmental factors may lead to variations in disease transmission dynamics between schools. Additionally, the study does not consider outside factors that may contribute to transmissions, such as community transmission rates, prevalence of the disease in the surrounding area, and effectiveness of public health measures implemented outside the school setting. Understanding these limitations is crucial for interpreting the findings accurately and for making informed decisions regarding disease control and prevention strategies in similar settings.




\chapter{Review of Related Literature}
\label{chap:Review of Related Literature}
 
 \section{Respiratory Infectious Diseases}

Respiratory infectious diseases (RIDs) are infections of the respiratory tract, which includes the lungs, airways, and other organs that are involved in breathing\cite{a2011_infectious}. RIDs are one of the most common human infections in all age groups and an important cause of mortality and morbidity worldwide. RIDs can be caused by a variety of pathogens, including viruses which are tiny infectious agents that live on organisms by replicating, bacteria are defined as single-celled microorganisms, that can decide grow and spread in the body, fungi are microorganisms that have cell walls made from a substance called chitin, and parasites that feeds on another organism to survive \cite{a2020_covid19}. Some common examples of RIDs include Pneumonia, an inflammation of the lung tissue, which can be caused by bacteria, viruses, or fungi. Tuberculosis, is a chronic infectious disease caused by the bacterium Mycobacterium tuberculosis\cite{a2011_infectious}. Influenza, is a respiratory illness caused by the influenza virus. Bronchitis: Inflammation of the bronchi, which are the tubes that carry air to and from the lungs. Sinusitis: Inflammation of the sinuses, which are air-filled cavities in the skull. RIDs can be spread through a variety of ways, including:

\begin{itemize}
	\item Airborne transmission: RIDs can be spread through the air when an infected person coughs or sneezes \cite{a2011_infectious}
	\item Droplet transmission: RIDs can be spread through contact with droplets from an infected person's cough or sneeze.
	\item Contact transmission: RIDs can be spread  through contact with an infected person's secretions, such as saliva or mucus.
	\item Fomite transmission: RIDs can be spread through contact with contaminated objects, such as doorknobs or towels. 
\end{itemize} 
The symptoms of RIDs can vary depending on the type of infection and the severity of the illness. However, some common symptoms of RIDs includes cough, fever, shortness of breath, chest pain, runny nose, sinus pain, sore throat, and, muscle aches\cite{a2011_infectious}. 

\subsection{COVID-19}
The latest of another respiratory infectious disease is COVID-19. Although coronaviruses have been causing human infections since the 1960s, it wasn't until the last two decades that their potential to trigger deadly epidemics became evident. COVID-19 marks the third major outbreak of a respiratory disease caused by a coronavirus within twenty years, and it has had a profound impact on the global socioeconomic equilibrium . It was recognized in December 2019 in Wuhan, Hubei Province, China\cite{a2020_what}. Through genome analysis, which studies an organism's genome, it was shown that it is a novel coronavirus related to SARS-CoV. With this discovery, it was named severe acute respiratory syndrome coronavirus 2 (SARS-CoV-2). On January 30, 2020, the World Health Organization declared the COVID-19 outbreak a Public Health Emergency of International Concern, and then, on March 11, 2020, it escalated the classification to a pandemic. \cite{worldhealthorganization_2020_covid19}\\

The source, transmission, and severity of 2019-nCoV are still mostly shrouded in uncertainty. Initially, a significant number of early infected patients were associated with the Huanan seafood wholesale market in Wuhan, China. Nevertheless, 13 out of the 41 early cases had no discernible connection to the market. Other sources and animals were also examined, like bats\cite{king_2011_virus}. SARS-CoV-2 or COVID-19 is a zoonotic virus with the ability to transmit from animals to humans and also among humans through airborne aerosols, a type of transmission that conveys infectious diseases through small particles suspended in the air. Specifically, the novel coronavirus exhibits a notably high rate of human-to-human \cite{agentbased} transmission, resulting in a broad range of clinical symptoms among infected patients.\\ \cite{explanation}

Having a high rate of human-to-human transmission has caused it to spread rapidly worldwide. As of this writing, there are 772,190,439 confirmed cases in the world, with a global death toll of 6,961,001 . The primary mode of human-to-human transmission of SARS-CoV-2 occurs when individuals are in close proximity to an infected person, typically through exposure to coughing, sneezing, respiratory droplets, or aerosols.


\section{Public Schools Health Protocol for COVID-19}   
After more than two years of school closures, a multitude of public school students are poised to return to classrooms this academic year following the Department of Education's (DepEd) directive to transition to five-day, in-person classes by November 2. However, school administrators may face considerable challenges in planning to avert overcrowding once students gradually resume on-campus attendance \cite{ditsworth_2019_infection}.

Nationwide, public schools could potentially face a shortage of approximately 400,000 classrooms if they adhere to a class size limit of 20 students to uphold physical distancing measures. This estimation is based on enrollment data from the 2020-2021 school year and information from the National School Building Inventory from the 2019-2020 school year \cite{chi_deped_nodate}.

In accordance with DepEd's guidelines for minimum health standards in schools during the initial year of the pandemic, as outlined in DepEd Order No. 14, s. 2020, students are expected to maintain a physical distance of one meter apart, with classroom occupancy ranging from 18 to 21 students, contingent upon the utilization of desks or armchairs. On the other hand, establishing a maximum class size of 40—similar to the typical pre-pandemic class size in public schools—would resolve classroom shortages for all regions except Calabarzon, NCR, and BARMM \cite{chi_2022_deped}.

Even if public schools opt to abandon the one-meter apart rule and accommodate 40 students per classroom, an estimated additional 10,000 classrooms would still need to be constructed. For example, NCR would still face a deficit of 5,793 classrooms, while Calabarzon would require an additional 3,624 classrooms. Additionally, BARMM would necessitate approximately 500 more classrooms.

In the Cordillera region, recent data from DepEd-CAR's Education Support Services Division indicates a requirement for 1,604 elementary classrooms, 1,381 secondary classrooms, and 121 integrated school classrooms. Mountain Province exhibits the highest deficit, with shortages of 355 elementary classrooms and 425 secondary classrooms, as per enrollment figures for 2022 \cite{g_2023_facility}.

According to Engineer Christopher Hadsan, the head of the Education Facilities Section, budget allocations for the region in past years and for the upcoming year 2023 have been inadequate to address the shortfall. Their most recent estimation suggests that constructing a classroom measuring 7×9 meters, capable of accommodating 40 to 45 students, incurs a cost of P3.5 million \cite{jones_2017_school}.

The simulation environment selected for this study is Pines City National High School, situated on Palma Street in Baguio City. As one of the most populous public high schools in the area, it accommodates a total of 2,136 students from grades 7 to 10, along with 89 teaching and non-teaching staff, totaling 2,225 individuals. Each classroom at the school accommodates between 40 and 55 students, which contravenes the DepEd protocol stipulating a maximum of 18 to 21 students per classroom to mitigate the transmission of COVID-19.

\section{Agent Based Modeling}

Agent-Based Modeling (ABM) belongs to a category of simulations where every member of a population is portrayed as a unique entity or "agent." These agents possess individual traits and behaviors and engage in interactions with one another, fostering opportunities for the transmission of contagions. Investigating epidemics is crucial, not just for comprehending outbreaks, but also for tackling associated social issues like governmental instability, crime, poverty, and inequality \cite{castiglione_2009_agent}.

\subsection{ABM for COVID-19 Transmission}

In a study developoed by Rojhun O. Macalinao, Jcob C. Malaguit, and Destiny S. Lutero developed an agent-based model to study COVID-19 transmission within Philippine classrooms. Utilizing NetLogo, their model examines the impact of student and teacher interactions on virus spread across four classroom layouts. The model features two types of agents, teachers, and students, each categorized as susceptible or infected. Agents exhibit mobility, and if a susceptible agent is within a two-seat node radius of an infected one, transmission can occur. Additionally, class rotations are incorporated, allowing students to move between classrooms. Macalinao et al.'s findings indicate that within-classroom mobility and class rotations contribute to increased COVID-19 transmission rates. Moreover, increasing the number of students per classroom layout correlates with higher infection rates within classrooms \cite{abm_covid19}.
\\
Another application of Agent Based Modeling is the study conducted by Christian Alvin H. Buhat, et. al. They developed an agent-based model and compartmental model (SEIR) to simulate the spread of respiratory infectious diseases between two neighboring cities. The study incorporates preventive measures such as social distancing and lockdowns within a city, along with the impact of protective measures. Factors including the likelihood of travel between cities and within them during lockdowns, as well as the initial percentage of exposed and infected individuals, influence the increase in newly-infected cases in both models. Results from simulations indicate several key findings: (i) An increase in exposed individuals correlates with a rise in new infections, underscoring the importance of enhanced testing and isolation efforts. (ii) Protective measures with effectiveness levels of 75-100\% significantly impede disease transmission. (iii) Travel within and between cities may be feasible under strict preventive measures, such as non-pharmaceutical interventions. (iv) Implementing lockdowns in neighboring cities during periods of high disease transmission risk, while emphasizing social distancing and protective measures, is optimal. Both the agent-based and compartmental models exhibit similar qualitative dynamics, with differences attributed to spatio-temporal heterogeneity and stochasticity. These models offer valuable insights for policymakers in formulating infectious disease-related policies aimed at safeguarding individuals while facilitating population movement. \cite{unknown}
\\
For other disease other than COVID-19, Agent Based Model was also used to asses the influenza interactions at the host level. They describe a novel agent-based model (ABM) of influenza transmission during interaction with another respiratory pathogen. The ABM produces authentic data for both pathogens, encompassing weekly incidences of PI cases, carriage rates, epidemic size, and epidemic timing. Notably, varied interaction hypotheses yielded diverse transmission patterns, leading to significant fluctuations in the associated PI burden. Additionally, the interaction strength played a pivotal role: instances where influenza heightened pneumococcus acquisition saw 4–27\% of the PI burden during the influenza season attributed to influenza, contingent upon the interaction strength. \cite{arduin-2017}

\section{Gama Platform}
For the purpose of building spatially explicit agent-based simulations, GAMA is an open-source modeling and simulation platform that is simple to use. It was designed to be used in any application domain. GAMA users have created models for usage in a variety of application domains, including urban mobility, epidemiology, disaster evacuation strategy design, and climate change adaptation.\\ \cite{unknown-author-no-date}

The high level of openness that accompanies the generality of the agent-based approach that GAMA promotes is demonstrated, for instance, by the creation of plugins tailored to particular requirements or by the ability to invoke GAMA from other programs or languages (like R or Python). The more than 2000 users of GAMA can use it for a wide range of applications due to its openness: communication tools, serious games, negotiation help, scientific simulation, scenario exploration and visualization.\\

Based on the RCP framework made available by Eclipse, GAMA is a single application. Using this one program, which is also called a platform, users can perform the majority of modeling and simulation tasks—such as editing models and simulating, visualizing, and exploring them with specialized tools—without the need for additional third-party software.
\chapter{Preliminaries}
\label{chap:Preliminaries}

\section{ Compartmental Model}


\hspace{1 cm}The compartmental model is a prevalent modeling technique in epidemiology, where the population is divided into compartments such as susceptible, exposed, infectious, recovered, or deceased. \cite{incremental} Individuals can transition between these compartments based on predefined rules. This model can be implemented using deterministic differential equations (equation-based model) or a stochastic framework (agent-based model). \cite{kermack-1927}

\hspace{1 cm}Epidemiological studies that utilize compartmental models categorize the population into distinct compartments. Among these models, the SIR model is widely used. It serves as a central component in various studies exploring the transmission dynamics of COVID-19. These models are instrumental in understanding the spread of COVID-19, aiding in predicting regional pandemic peaks and evaluating the impacts of different quarantine measures. \cite{abouismail_2020_compartmental}

\subsection{Susceptible-Infected (SI) Model}
\indent \indent The Susceptible-Infected (SI) Model is known to be the simplest form of disease models where individuals can be categorized either as susceptible or infected. In this model, individuals are naturally born into the simulation as susceptible or with no immunity. \cite{incremental} Once individuals are infected and have no access to treatment, they stay infected and remain contagious for the duration of their life, continuing to come into contact with the susceptible population.In the absence of vital processes such as birth and death, every susceptible individual will eventually become infected. \cite{AHMETOLAN202219}

\begin{figure}[H]
	\centering
	{\includegraphics[width = 0.70\textwidth, height=3cm]{images/SI.png}}
	
	\caption{An illustration depicting the progression of states within the Susceptible-Infected Compartmental Model. }%
	\label{1}%
\end{figure}

 \hspace{1 cm}Figure\ref{1} shows the two populations in the SI Model. The Susceptible (S) and Infected (I) compartments. The $\alpha$ represents the rate of transmission at which the susceptible agents become infected. The SI model can be expressed as the following ordinary differential equation (ODE) in equation \ref{2.1}.

\begin{equation}
		\begin{split}
		S' &= -\frac{\alpha SI}{N}\\
		I' &= \frac{\alpha SI}{N}
		\label{2.1}
	\end{split}
\end{equation}




The illustration shows that it starts with susceptible and proceeds to infection . An individual can become infected at a specific time t with an infection rate denoted as $\alpha$. The change in the number of susceptible individual as time progresses is denoted as $S'$.  

Additionally, the change in the number  of infected individual as time progresses is represented by $I'$ where  $N = S + I$  is the total population.  Notice how the change in the number of susceptible individuals decreases (-) while the number of infected individuals increases. 

\subsection{ Susceptible-Infected-Recovered (SIR) Model}

\begin{figure}[H]
	\centering
{\includegraphics[width = 0.70\textwidth, height=3cm]{images/SIR.png}}
\caption{An illustration depicting the progression of states within the Susceptible-Infected-Recovered Compartmental Model. }%
	\label{fig:2}
\end{figure}

Figure \ref{fig:2} describes the SIR model and its transitions between the compartments. It is one of the fundamental compartmental models, and several extensions of this basic model exist, including the SIR model. The SIR model comprises of three compartments: 'S' for the number of susceptible individuals, 'I' for the number of infectious individuals, and 'R' for the number of recovered, deceased, or immune individuals. The SI model can be expressed as the following ordinary differential equation (ODE). \cite{SIR}

\begin{equation}
	\begin{split}
		S' &= -\frac{\alpha SI}{N}\\
		I' &= \frac{\alpha SI}{N} - \delta I \\
		R' &= \delta I		
		\label{2.2}
	\end{split}
\end{equation}


Equation S' describes the dynamics of the reduction of susceptible individuals, where $\beta$ is the average number of people who come into contact with another person multiplied by the likelihood of infection in that contact.\cite{abouismail_2020_compartmental} Equation I' represents the variation for the I compartment, where the new infected ones according to the rate are added and those who were recovered or died are removed, proportional to the parameter $\gamma = \frac{1}{D}$, where D is the number of days that one individual stays infected. The last equation explains the variation on the compartment of the recovered/ mortality patients, which is also directly proportional to $\gamma$. 

\subsection{Susceptible-Exposed-Infected-Recovered (SEIR) Model}
\begin{figure}[H]
	\centering
{\includegraphics[width = 0.80\textwidth, height=3cm]{images/SEIR.png}}
\caption{An illustration depicting the progression of states within the Susceptible-Exposed-Infected-Recovered Compartmental Model. }%
	\label{fig:3}
\end{figure}

This model as seen in Figure \ref{fig:3} consists of 4 compartments \cite{bjrnstad-2020}, mainly the Susceptible (S), Exposed (E), Infected (I) and Recovered (R). It is a model proposed where after a specific period, the susceptible person can get infectious. This model is named the SEIR model and can be presented as an Ordinary Differential Equation (ODE) as presented in Equation \ref{2.3} . 

\begin{equation}
	\begin{split}
		S' &= -\frac{\beta SI}{N}\\
		E' &= \frac{\beta SI}{N} - \alpha E \\
		I' &=  \alpha E - \delta I \\
		R' &= \delta I 
		\label{2.3}  	
	\end{split}
\end{equation}


Analogous to the SIR representation, the sum of the compartments, which are now S(t) + E(t) + I(t) + R(t) = N, results in the total population (Fig. \ref{fig:3}). The parameters considered for the SEIR model are described below: 

\begin{itemize}
	\item Beta ($\beta$) - probability susceptible–infected contact results in a new exposure;
	\item Delta ($\delta$) - probability of one infected subject gets recovered
	\item  Alpha ($\alpha$) -  probability of one exposed person becoming infected.
\end{itemize}

\subsection{Agent-Based Modelling}
\vspace{8pt} %

An agent-based model constitutes a form of computer simulation wherein agents interact with each other within a defined environment. These agents' interactions and behaviors adhere to a set of programmed rules. Since agents can autonomously make decisions within the model based on these rules, the model is capable of capturing unforeseen collective phenomena arising from the combination of individual behaviors. 

Agent-based models (ABMs) belong to a category of computational models that utilize computer simulations to replicate the activities and interactions of autonomous agents, thereby assessing how these interactions influence the system as a whole. The agent-based approach places significant emphasis on learning through the interactions between agents and their environments. This approach aligns with a recent trend in computational models of learning, which seeks to develop novel methodologies for studying autonomous agents in virtual or real-world contexts.

For this simulation, GAMA was employed, a powerful platform for implementing Agent-Based Modeling (ABM). GAMA, short for Generalized Architecture for Modeling Anything, offers a versatile and programmable environment tailored to simulate diverse social and natural phenomena. Its robust framework enables the creation and analysis of agent-based models, facilitating researchers' exploration of complex systems and observation of emergent behavior.

  \subsection{ Simulation Space}
  \label{SSpace}
\begin{figure}[H]
	\centering
		{\includegraphics[width = 0.70\textwidth, height=9cm]{images/SSpace.png}}
	\caption{Illustration of the simulation space of the implemented model in GAMA}
	\label{space}
\end{figure}


Figure \ref{space} provides a comprehensive visualization of the simulation domain utilized in implementing the model within the GAMA platform. The illustration delineates the spatial boundaries of the simulation along both the x and y axes, denoted by the intervals $[L_x, U_x]$ and $[L_y, U_y]$, respectively. These boundaries define the extent of the simulated environment, encompassing the entire area where agent interactions and dynamics unfold. By establishing these spatial constraints, the model ensures a realistic representation of the simulated scenario, facilitating the accurate evaluation of agent behaviors and disease transmission dynamics within the defined space.

\subsection{Agents}

\textbf{\large{Position and Assignment of Agent into the Simulation}}
\label{Pos}
\begin{figure}[H]
	\centering{
	\includegraphics[width=10cm, height=8cm]{images/SI_Location.JPG}}
	\caption{Illustration of the x and y positions of agents in a 2-D plane}
	\label{pos}
\end{figure}

The initial position of each agent is determined by the following equations:
\begin{align*}
	a &= CL_x + \text{rnd}(0,1) \cdot (CU_x - CL_x) \\
	b &= CL_y + \text{rnd}(0,1) \cdot (CU_y - CL_y)
\end{align*}

Here, $a$ represents a point on the x-axis plane, and $b$ represents a point on the y-axis plane. Combining these points yields coordinates such as $(a_x, b_y)$, where $x \in (1, 2, 3,...n)$ and $y \in (1, 2, 3,...,n)$.

In the figure, if an initial susceptible agent, denoted $s_1$, is located at coordinates $(a_1, b_1)$, then another susceptible agent $s_2$ must be assigned different coordinates, ensuring $s_{1(loc)} \neq s_{2(loc)}$.

This distinctive positioning principle is maintained for all agents in the susceptible-exposed-infected-recovered (SEIR) model. Each agent is assigned a unique coordinate within the simulation space to prevent overlap. Furthermore, every agent is designated a singular health status, which could be susceptible $(s)$, exposed $(e)$, infected $(i)$, or recovered $(r)$. This approach guarantees spatial diversity and individual health status differentiation among all agents participating in the simulation.

\subsubsection{State of the Agents}
\label{state}
\begin{figure}[H]
	\centering
	{\includegraphics[width=10cm, height=10cm]{images/SEIR_Agents.jpg}
	\caption{Color-coded representation of agents in the simulation, with distinct colors indicating their current
		health status (Susceptible,Exposed, Infected, Recovered)}
	\label{agent1}}
\end{figure}


As represented in Figure \ref{agent1}. The initial state of the agents is defined at the start of the simulation at t = 0. Each agent can be assigned to four different compartments \textbf{S}(t), \textbf{E}(t), \textbf{I}(t), \textbf{R}(t) which represents susceptible, exposed, infected, and recovered state respectively. This distribution is visible in the simulation space during the first iteration $(t = 0)$. The agents in each compartment are initialized and can be represented by 

\begin{align}
	\textbf{S}_{b,k}^a(t) &= s_{b,1}^a(t), s_{b,2}^a(t), \ldots, s_{b,S}^a(t)\\
	\textbf{E}_{b,k}^a(t) &= e_{b,1}^a(t), e_{b,2}^a(t), \ldots, e_{b,I}^a(t)\\
	\textbf{I}_{b,k}^a(t) &= i_{b,1}^a(t), i_{b,2}^a(t), \ldots, i_{b,I}^a(t)\\
	\textbf{R}_{b,k}^a(t) &= r_{b,1}^a(t), r_{b,2}^a(t), \ldots, r_{b,I}^a(t)
\end{align}

where:
\begin{itemize}
	\item \quad $S$: total number of susceptible agents at each time $t$.
	\item \quad $E$: total number of exposed agents at each time $t$.
	\item \quad $I$: total number of infected agents at each time $t$.
	\item \quad $R$ : total number of recovered agents at each time $t$.
	\item \quad $a$: the x-axis location of the agent represented by $(1, 2, 3, \ldots, n)$.
	\item \quad $b$: the y-axis location of the agent represented by $(1, 2, 3, \ldots, n)$.
	\item \quad $k$: a specific identifier for each agent represented by $(1, 2, 3, \ldots, n)$.
\end{itemize}

The composition of  susceptible,exposed, infected and recovered agents forms the total population, as expressed by the equation:

\[
\textbf{P}(t) = \textbf{S}(t) +\textbf{E}(t) + \textbf{I}(t) + \textbf{R}(t)
\]

Here, \textbf{P}(t) denotes the total population at time 0, while \textbf{S}(t), \textbf{E}(t), \textbf{I}(t) and \textbf{R}(t) signify the counts of susceptible, exposed, infected, and recovered agents, respectively. This representation provides a visual understanding of the distribution of health states in the initial stage of the simulation.

\subsubsection{ Movement of Agents}    
\label{agent_move}
\begin{figure}[H]
	\centering
	{\includegraphics[width=8cm, height=7cm]{images/SI_Movement.JPG}}
	\caption{Illustration of the movement rule of the agent in the SI model.}
	\label{move}
\end{figure}

Figure \ref{move} reveals the movement dynamics of agents within the simulation space. Specifically, the movement of a susceptible agent $(s_1)$ at time $t+1$ is governed by the parameter $r$, representing the allowable distance of movement. The agent's new position is determined using the following equations:

\begin{align*}
	a_{t+1} &= a_t + \text{rnd}(-1,1) \cdot d \\
	b_{t+1} &= b_t + \text{rnd}(-1,1) \cdot d
\end{align*}


The given equations describe the update or calculation of the new location of an agent in a two-dimensional space $(a, b)$ at time $t+1$ based on its current location at time $t$. Here, $a_{(t+1)}$ represents the new x-axis location of the agent at time $t + 1$. It is determined by taking the current x-axis location $(a_t)$, and adding a random value between -1 and 1 (inclusive) multiplied by a constant $d$. The $rnd(-1,1)$ term generates a random number between -1 and 1. Similarly, $b_{(t+1)}$  this represents the new y-axis location of the agent time $t + 1 $. It is determined by taking the current y-axis location $b_t$, and adding a random value between -1 and 1 also multiplied by the constant d. 

These equations describe a stochastic process where the agent's new location is influenced by its current location $(a_t, b_t)$ and a random perturbation in both the c and y directions. The magnitude of the perturbation is controlled by the constant d.

In addition, the simulation visually represents the movement dynamics of each individual within the defined space. Individuals are capable of navigating freely within the simulation space, with their movements visible as the simulation progresses.

At the initiation of the simulation, when the social distancing variable is set to false, agents enjoy unrestricted mobility within the simulation space, constrained only by the specified limits $[L_x, U_x]$ and $[L_y, U_y]$.

\subsection{Exposure and Infection Rule}
\label{rules}
\begin{figure}[H]
	\centering
	\subfigure[Initial configuration of the SEIR model at time $(t)$]{
		\includegraphics[width = 2.4in]{images/SEIR a.png}
		\label{expo1} % Moved label inside the subfigure
	}
	\quad
	\subfigure[Final configuration of the SEIR model at time $(t+1)$]{
		\includegraphics[width = 2.4in]{images/SEIR b.png}
		\label{expo2}
	}
	\caption{Exposure and Infection Rule of the SEIR Model}
\end{figure}

The exposure rule in the SEIR (Susceptible-Exposed-Infectious-Recovered) model outlines how susceptible individuals transition to an exposed state upon contact with infected individuals. In Figure \ref{expo1}, the initial state of the SEIR model at time \(t\) is depicted. Notably, \(s_1\) and \(s_8\) are identified within the proximity range of infection for \(i_1\) and \(i_2\) respectively. According to the exposure rule, susceptible individuals like \(s_1\) and \(s_8\) are at risk of infection when near infectious individuals.


Figure \ref{expo2} illustrates the subsequent state at time \(t+1\), displaying the outcomes of the exposure rule. Due to their proximity to infected individuals in the previous configuration, \(s_1\) transitions from susceptible to exposed, indicating exposure to the infection. Simultaneously, \(s_8\) shifts from susceptible to exposed \(e_2\). This transition underscores the dynamic nature of the SEIR model, where exposure to infectious agents prompts susceptible individuals to enter an exposed state, setting the stage for further infection dynamics in the ongoing simulation. The exposure rule captures the crucial role of interactions and proximity in the transmission dynamics of infectious diseases within the simulated population.\\








\chapter{Methodology}
\label{chap:Methodology}

 
  \section{Model Implementation in PCNHS}
 The implemented system is a stochastic agent-based model designed to simulate the transmission of respiratoru infectious disease like the COVID-19 virus within Pines City National High School (PCNHS). The agent-based models are executed within the GAMA Platform, a specialized environment for agent-based modeling and simulations.
 
 In this study, we simulated a closed community residing within the confines of a shared finite environment, comprising individuals who interact with one another. This classification encapsulates the fundamental aspects of societal dynamics and interactions within the school setting. The model will utilize the SEIR (Susceptible-Exposed-Infectious-Recovered) framework, representing the different states of individuals within the school population. The susceptible \textbf{(S)} individuals, the exposed \textbf{ (E)} individuals, the infectious \textbf{ (I)} individuals, and the recovered \textbf{ (R)} individuals . 
 
 \section{SEIR Model}
 \begin{itemize}
 	\item Susceptible Agents - Individuals who are vulnerable to the disease but have not been infected yet. They can potentially contract the disease if exposed to it.
 	\item Exposed Agents - This group represents individuals who have been exposed to the pathogen (e.g., through contact with an infected person) but are not yet infectious themselves. The incubation period occurs during this stage.
 	\item Infected Agents -  These are individuals who are currently infected and can transmit the disease to others. They are actively contagious.
 	\item Recovered Agents -  Individuals who have recovered from the infection and are now immune.
 \end{itemize}
 
  \section{Parameters}
 The following parameters are used for the general simulation of the Agent-Based Modeling in GAMA
 \begin{table}[H] % Use [h] to indicate that you prefer the table to be placed here
 	\centering
 	\begin{tabular}{ll}
 	\toprule
 		\textbf{Variable/}\textbf{Parameter} & \textbf{Default Value}\\
 	\hline
 		init\_human\_pop & 2225\\
 		init\_inf & 0.25 * int\_human\_pop\\
 		init\_expo & exposed\_rate * int\_human\_pop \\
 		agent\_speed & 50.0 cm/s\\
 		infection\_distance & 1.8 meters\\
 		exp\_rate & 0.25 to 1.0\\
 		protection\_rate & 0 to 0.25\\
 		proba\_infection & 1.0 - protection\_rate\\
 		proba\_exposed & 0.50\\
 		step & 10 minutes\\
 		exposed\_period & 288 cycles to 1440 cycles\\
 		infectious\_period & 1440 cycles to 2016 cycles\\
 		recovered\_period & 2880 cycles to 8640 cycles\\ 
 		time\_cycle & 31680 cycles\\
 		\bottomrule
 	\end{tabular}
 	\caption{Variables used for the GAMA Simulation of the SEIR Model }
 \end{table}
 Furthermore, each parameters are defined by their role in the simulation, specifically
 \begin{itemize}     
 	\item Let \( N \) represent the total population, initially set to 2225.
 	
 	\item \( \textbf{S(t)} \) defines the number of susceptible agents at time \( t \). The number of agents can be calculated using the formula:
 	\[ \textbf{S(t)} = S(t-ts) - e + s \]
 	where \( ts \) is the 10-minute time step simulation, \( e \) is the number of susceptible agents that become exposed at time \( t \), and \( s \) is the number of agents that become susceptible again at time \( t \). At time 0, the initial number of \( S \) is given by \( S(0) = N - E(0) - I(0) \).
 	
 	\item \textit{E(t)} represents the number of exposed agents at time \( t \), given by the formula:
 	\[ \textit{E(t)} = E(t - ts) - i - s + e \]
 	where \( s \) is the number of exposed agents that return to being susceptible agents at time \( t \), \( i \) is the number of exposed agents that become infected agents at time \( t \), and \( e \) is the number of susceptible agents that become exposed agents at time \( t \). For time 0, the number of exposed agents is set to \( E(0) = 557 \), where 25\% of \( N \) is set to be exposed. For Scenario 2, the number of exposed agents varies depending on the Exposure Rate, and the initial number of exposed agents can be obtained by:
 	\[ E(0) = N \times exp\_rate \]
 	
 	\item \( I(t) \) represents the number of infected agents at time \( t \), given by:
 	\[ \textbf{\textit{I(t)} }= I(t - ts) - r + i \]
 	where \( r \) is the number of infected agents that become recovered agents at time \( t \) and \( i \) is the number of exposed agents that become infected agents at time \( t \).  For the simulation, the initial number of \( I \) at the start of the simulation is 25\% of the population, given by:
 	\[ I(0) = N \times (0.25) \]
 	
 	\item \textit{R(t)} represents the number of recovered agents at time \( t \), given by:
 	\[ \textbf{\textit{R(t)}} = R(t - ts) + r \]
 	where \( r \) is the number of infected agents that become recovered at time \( t \). For the simulation model, the initial number of \( R \) at time 0 is given by \( R(0) = 0 \).
 	
 	\item \textit{Susceptible\_rate} represents the susceptible rate of the virus inside the campus which is given by the formula
 	\[
 	Susceptible\_rate = \frac{\textbf{\textbf{S(t)}}}{N}
 	\]
 	\item \textit{Exposure\_rate} represents the rate of exposure of the virus inside the campus at time \textit{t} which is given by
 	\[
 	Exposure\_rate = \frac{\textbf{E(t)}}{N}
 	\]
 	\item \textit{Infection\_rate} represents the infection rate of the virus inside the campus at time \textit{t} which is given by
 	\[
 	Infection\_rate = \frac{\textbf{I(t)}}{N}
 	\]
 	\item \textit{Recovery\_rate} represents the recovery rate of the virus inside the campus at time \textit{t} which is given by
 	\begin{equation*}
 		Recovery\_rate = \frac{\textbf{R(t)}}{N}
 	\end{equation*}
 	
 	\item \textit{agent\_speed} this refers to the movement speed of every agent inside the facility, set to \textit{50 cm/s} for the simulation.
 	
 	\item \textit{time\_step} this represents the number of minutes per cycle, set to 10 minutes per cycle.
 	
 	\item \textit{infection\_distance} this represents the distance for infection to occur. If an infected agent contacts another agent for 20 minutes, the latter will be exposed to the virus.
 	
 	\item \textit{exp\_rate} this refers to the initial number of exposed rate at time 0, ranging from 0\% to 75\%, with four levels: 0\%, 25\%, 50\%, 75\%.
 	
 	\item \textit{protection\_rate} this indicates the level of protection for each agent, ranging from 0\% to 100\%, with five protection levels: 0\%, 25\%, 50\%, 75\%, 100\%.
 	
 
 	
 	\item \textit{proba\_infection} this defines the probability of an agent becoming infected, influenced by the \textit{protection\_rate}, represented by the formula:
 	\[
 	proba\_infection = 1.0 - protection\_rate
 	\]
 	
 	\item \textit{proba\_exposed} this is the probability of an agent being exposed to the virus, initially set to 0.50.
 	
 	\item \textit{exposed\_period} this defines the number of incubation days for exposed agents before becoming infected or susceptible, randomly set to 2 - 10 days from the day of exposure.
 	
 	\item \textit{infectious\_period} this is the number of days an infected agent remains infectious before recovery, set to 10 - 14 days from the day of infection.
 	
 	\item \textit{recovered\_period} this is the number of days an agent is considered recovered and immune to reinfection. After this period, they can be infected again. It is set to 20-60 days.
 	
 \end{itemize}
 
  \section{Simulation Space of the Simulation}
 
 
 The first step is to set up the virtual environment where the simulation would take place. This virtual space symbolized Pines City National High School (PCNHS), characterized by heightened chances of infection and interaction among its occupants. AutoCAD software facilitated the creation of classroom and hallway models. Following this, shapefiles for both classrooms and hallways were produced using the Quantum Geographic Information System (QGIS) of the school, posing a 2-dimensional coordinate plane formed by the intersection of the x-axis and y-axis. These shapefiles follow the experiment's boundaries based on the school's dimensions.
 
 \subsection{Constructing the Shapefile Structure}
 
 The structure of the shapefile was constructed using AutoCAD, as mentioned in the preliminaries. It allows users to create precise 2D and 3D drawings used in various industries such as architecture, engineering, construction, manufacturing, and more. AutoCAD provides a wide range of tools and features that enable users to design and document their ideas with accuracy and efficiency. 
 
 \begin{figure}[H]
 	\centering
 	\includegraphics[width=10cm, height=12cm]{images/CAD_shp.png}
 	\caption{Virtual Representation of the structure of the Shapefile in AutoCAD}
 	\label{cad_shp}
 \end{figure}   
 
 The simulation space was fashioned within a Euclidean complex network, demarcated by predefined lower and upper limits along the x-axis $[L_x, U_x]$ and y-axis $[L_y, U_y]$. Comprising interconnected rooms and hallways, the space was symbolized as $C = \{C_1, C_2, C_3, ..., C_n\}$, where 'n' represented the total number of rooms. The $U_y$ = 115m and $U_x$ = 105, this serves as the upper and lower bounds of the simulation respectively.\\
 
\subsection{Classrooms}
 
 \begin{figure}[H]
 	\centering
 	\includegraphics[width=10cm, height=12cm]{images/rooms_shp.png}
 	\caption{Visual representation of the shapefile image of the classrooms. The gray polygons represent the rooms}
 	\label{rooms_shp}
 \end{figure}   
 
 The shapefile depicted in Figure \ref{rooms_shp} represents the spatial layout of the classrooms within the school premises. These rooms serve as designated locations where each agent resides or operates within the simulated environment. As the simulation progresses, the agents can move beyond the confines of their respective classrooms, facilitated by the  road network integrated into the simulation model.
 
 \subsection{Quarantine Area}
 \begin{figure}[H]
 	\centering
 	\includegraphics[width=14cm, height=11cm]{images/quarantine.png}
 	\caption{Visual representation of the shapefile image of the quarantine area. The gray polygons represent the quarantine rooms}
 	\label{qua}
 	
 \end{figure}   
 
 The shapefile in Figure \ref{qua} represents the quarantine area in the simulation. It's where infected individuals are kept separate from others. This area is crucial in Scenario 3, where both Lockdown and Quarantine Area measures are used. By isolating infected individuals here, the simulation helps us understand how quarantine and lockdown affects the spread of disease.
 
 \subsection{Environment}
 \begin{figure}[H]
 	\centering
 	\includegraphics[width=14cm, height=12cm]{images/hallways_shp.png}
 	\caption{Visual representation of the shapefile image of the hallway. The gray polygons represent the environment}
 	\label{hallways}
 \end{figure}   
 
 The shapefile illustrated in Figure \ref{hallways} provides an overview of the environmental layout surrounding the classrooms. This area encompasses the hallways, corridors, and pathways that form the road network crucial for agent movement and navigation within the simulation environment. 
 \subsection{Road Network}
 
 The shapefile depicted in Figure \ref{road_shp} provides a visual representation of the road network associated with the classrooms. This network represents the pathways and routes that agents navigate within the simulation environment, extending beyond the confines of individual classrooms to encompass movement throughout the simulated space.
 
 Within this illustrated road network, agents traverse interconnected paths, corridors, and outdoor areas, enabling them to move dynamically within and between different zones of the simulated environment. Unlike the static nature of classroom spaces, the road network symbolizes the dynamic mobility and interactions that agents engage in throughout the simulation.
 \begin{figure}[H]
 	\centering
 	\includegraphics[width=10cm, height=12cm]{images/road_shp.png}
 	\caption{Visual representation of the shapefile image of the road network}
 	\label{road_shp}
 \end{figure}   
 
 
\begin{figure}[H]
	\centering
		\includegraphics[width=14cm, height=11cm]{images/GAMA_SHP.png}
			%\label{GAMA3D} % Moved label inside the subfigure
	
		\caption{View of the shapefile in GAMA Simulation}
		\label{GAMA3D}
\end{figure} 
 
 In Figure \ref{GAMA3D}, we are presented with a comprehensive 3D visualization of the shapefile within the GAMA simulation environment. This visualization encapsulates not only the physical environment but also integrates crucial elements such as classrooms, quarantine are and the road network shapefile. These elements collectively form the environment where the simulation unfolds, providing a rich and immersive place for agent interactions. Within this virtual space, agents navigate, collaborate, simulating dynamic exchanges and engagements akin to real-world scenarios.
 
 The simulated space becomes a dynamic playground where agents move through corridors,  and traverse the interconnected pathways of the road network. 
 
 After setting up the virtual simulation space for the agents, the next thing is the positioning, movement, and interaction rule of the agents. 
 
 
 \section{ The Agents}
 Within the simulation framework, individuals within the population are depicted as agents endowed with the capacity to navigate and interact with fellow agents within the designated simulation area. Each agent occupies a unique position within this spatial realm. Furthermore, these agents are classified into four distinct compartments: susceptible\textbf{(S}), Exposed \textbf{(E)}, Infected \textbf{(I)}, and Recovered \textbf{(R)}. This classification implies that at any given moment, an agent is situated in either the susceptible state, indicating vulnerability to infection, exposed, infected signifying an active infection within the simulation and recovered state. This provides a clear depiction of their respective health statuses within the simulated environment.
 
 \subsection{Position and Assignment of Agent into the Simulation}
 
 \begin{figure}[H]
 	\centering
 	\includegraphics[width=13cm, height=8cm]{images/imit_pos.png}
 	\caption{Initial position of each agent in the GAMA Simulation}
 	\label{init}
 \end{figure}
 
 Figure \ref{init} offers a detailed snapshot of the initial configuration of agents within the simulation environment. Each of the 2225 agents is meticulously positioned across the space, with specific classrooms serving as their designated starting points. This initial arrangement sets the stage for the subsequent dynamics of the simulation, dictating how agents interact and move within the environment over time.
 
 The positioning of agents plays a crucial role in shaping the overall trajectory of the simulation. Factors such as proximity to other agents, classroom locations, and spatial constraints all influence how agents navigate the environment and interact with one another. By examining the distribution of agents at the outset of the simulation, we can glean valuable insights into potential patterns of movement, clustering, and interaction that may emerge as the simulation progresses.
 
 \subsection{State of the Agents in the Simulation}
 \begin{figure}[H]
 	\centering
 	\includegraphics[width=13cm, height=10cm]{images/agent.png}
 	\caption{Color-coded representation of agents in the simulation, with distinct colors indicating their current
 		health status (Susceptible,Exposed, Infected, Recovered)}
 	\label{agent}
 \end{figure}
 
 
 Figure \ref{agent} provides a comprehensive visualization of the diverse states that agents can assume within the simulation framework, namely Susceptible\textbf{ (S}), Exposed \textbf{(E)}, Infected\textbf{ (I)}, and Recovered\textbf{ (R)}. Each agent is meticulously categorized into one of these distinct states, signifying their susceptibility to infection, current exposure status, infectious state, or recovery status.
 
 Upon closer inspection, we observe that agents are exclusively assigned to one of these states, reflecting the individual conditions they inhabit within the simulated environment. From the onset of the simulation \textit{(t=0)}, a predetermined number of agents are designated as infected or exposed, influencing the initial dynamics of disease transmission and population health outcomes.
 
 Each agent is counted at every time step, enabling the precise tracking of the number of agents in each state. This process allows for the continuous tracking of state transitions and fluctuations, facilitating the generation of accurate and insightful simulation outcomes. 
 
 \subsection{Movement Rule of the Agents}
 In this phase of simulation, the movement patterns of agents within the school environment are intricately tied to specific time intervals and the structure of the school day. Each agent is characterized by a consistent speed of \textit{50 cm/s}, representing a realistic pace for movement within indoor spaces.
 \begin{figure}[H]
 	\centering 
 	\includegraphics[width=14cm, height=10cm]{images/class.png}
 	\caption{Agent movement and spatial interactions within the classroom environment during scheduled class time.}
 	\label{classtime}
 \end{figure}
 
 The school day typically commences at 7:30 AM, showing the start of scheduled classes. During the initial period from 7:30 AM to 9:00 AM, agents' movements are primarily restricted to the confines of their respective classrooms. This restriction aligns with typical classroom activities and early morning instructional periods. This can be seen in Figure \ref{class} where the agents are inside the classrooms.
 
 
 At 9:00 AM, a transition occurs as students are granted a 30-minute break until 9:30 AM. This break period marks a shift in movement dynamics, allowing students to venture beyond their classrooms and navigate through the school's designated road network. This simulates the movement patterns observed during breaks or transitions between classes, offering a more nuanced representation of school dynamics.
 
 
 Following the break, at approximately 9:30 AM, agents revert to their assigned classrooms, signaling the end of the break period and a return to focused classroom activities. This cyclical pattern repeats throughout the school day, with similar movements permitted during the lunch break from 12:00 PM to 1:00 PM, offering agents another opportunity to explore the school environment beyond classroom walls. This scenario can be seen in Figure \ref{lunch} where some agents can be seen wandering outside the classrooms using the road networks.
 \begin{figure}[H]
 	\centering
 	\includegraphics[width=14cm, height=10cm]{images/lunch.png}
 	\caption{    Agent movement and spatial interactions within the classroom environment during scheduled lunch, break, and end of class.}
 	\label{lunch}
 \end{figure}
 Post-lunch break, movement is once again confined within classrooms until the end of the school day, typically around 4:00 PM, mirroring the structured nature of academic schedules and classroom engagements.
 
 
 Outside of regular school hours, from 8:00 PM until 7:00 AM the following day, agent movements are intentionally slowed down to zero. This slowdown reflects the assumption of minimal or no activity during nighttime hours when the school is officially closed, aligning with realistic expectations of school operation hours and periods of inactivity.
 
 By intricately modeling movement dynamics tied to specific time intervals and school day segments, the simulation captures nuanced behaviors reflective of real-world school environments. This detailed approach ensures a more accurate representation of agent interactions and spatial dynamics within the school setting, enhancing the overall realism and analytical depth of the simulation outcomes.
 
 \subsection{Transmission of the Respiratory Infectious Disease}
 \subsubsection{Exposure Rule: Identifying if a susceptible (s) agent will get exposed}
 
 
 For the simulation in GAMA, the rule follows:
 \begin{itemize}
 	\begin{figure}[H]
 		\centering
 		\includegraphics[width=14cm, height=10cm]{images/exposed.png}
 		\caption{ Snapshot of an exposed agent within the GAMA simulation at a specific time.}
 		\label{exposed}
 	\end{figure}
 	\item Rule 1: The model examines each susceptible agent in category S to determine if an infected agent is present within the defined neighborhood \textit{id}. If an infected agent is discovered within this neighborhood, it signifies a potential transmission event of the virus. The model sets the value of R at 1.8 meters, roughly equivalent to the distance classified as close contact with an infected individual in the context of COVID-19, which is 6 feet.
 \end{itemize}
 When agents within the defined \textit{infection\_distance} parameter come into contact with an infected individual for a minimum duration of 15 minutes or 2 cycles within the simulation timeframe, they face the possibility of exposure to the virus. Upon exposure, an \textit{exposed\_counter} mechanism is activated to monitor susceptible agents. This counter increments with each cycle of the simulation. If the \textit{exposed\_counter} reaches or exceeds a value of two, corresponding to approximately 20 minutes in simulation time, the susceptible agent transitions to the exposed state and will turn from blue to yellow which can be seen at Figure \ref{exposed} . 
 
 Subsequently, the agent is removed from the Susceptible \textbf{(S)} group and added in the group \textbf{E}. The moment of transition to the exposed state marks the \textit{exposed\_time} for the agent, serving as a reference point for subsequent infection-related rules and events within the simulation. \\
 
  \subsubsection{Infection Rule: Identifying if an exposed (e) agent will get infected}
 
 \begin{figure}[H]
 	\centering
 	\includegraphics[width=14cm, height=10cm]{images/infected.png}
 	\caption{ Snapshot of an infected agent within the GAMA simulation at a specific time.}
 	\label{infected}
 \end{figure}
 Once the incubation period of the exposed agent is done or if \textit{exposed\_time} of the exposed agent is greater than or equal to the \textit{exposed\_period} then an infection rule will be considered. 
 
 \begin{itemize}
 	\item Rule 2: For each exposed agent \textit{e} in \textbf{E}, once the incubation period has elapsed, the likelihood of infection (\textit{proba\_infection}) determines whether the agent becomes infected. This probability heavily depends on the protection level (\textit{protection\_rate}) set in the simulation. The simulation offers five protection levels: $0\%, 25\%, 50\%, 75\%, 100\%$. The probability of infection is computed as 1.0 minus the protection level. For instance, if the \textit{protection\_rate} is $25\%$, the infection probability is $75\%$, indicating a high chance of the exposed agent getting infected. Upon infection, the agent transitions from group \textbf{E} to \textbf{I} and will turn from yellow to red as seen in Figure \ref{infected}; otherwise, it returns to susceptibility and reenters group \textbf{S}.
 	
 	The transition to the exposed state signifies the onset of the \textit{infected\_time} or the duration of infection for the agent, which is assigned an infectious period of 10-14 days. Following the completion of the infectious period, the infected agent progresses to the recovered state. \\
 \end{itemize}
 
 
  \subsubsection{Recovery Rule: Identifying if an infected (i) agent will recover}
 \begin{figure}[H]
 	\centering
 	\includegraphics[width=14cm, height=10cm]{images/recovered.png}
 	\caption{ Snapshot of an recovered agent within the GAMA simulation at a specific time.}
 	\label{recovered}
 \end{figure}
 Once the infectious period of the for the infected agent is over, a new set of recovery rule will be considered to identify if the infected agent will recover or not. 
 
 \begin{itemize}
 	\item Rule 3: For each agent in \textbf{I}, if the \textit{infected\_time} exceeds or equals the \textit{infectious\_period}, the agent is considered recovered. Subsequently, the recovered agent transitions to group \textbf{R} and exits group \textbf{I}. In the simulation, their red color transforms to green, indicating recovery (refer to Figure \ref{recovered}). Additionally, a random recovery period between 20 to 60 days is assigned to the recovered agents, granting them immunity during this timeframe. However, once the recovery period elapses, they can become susceptible to the disease and can be infected again.
 \end{itemize}

 \section{ Simulation Scenarios}
 
 

 \subsection{ Scenario 1: Varying Protection Rate}
In this particular scenario, five sub-scenarios were executed, each with a unique set of variables and conditions. The simulation was conducted by varying the different protection rates, which ranged from 0\% to 100\%. The purpose of this simulation was to obtain accurate information about how protection rates affect the overall outcome of the scenario. By running multiple sub-scenarios, the simulation was able to take into account various factors and variables that may have affected the results. 

	\subsubsection{Scenario 1.1: 0\% Protection Rate}
	\begin{table}[H]
		\centering
		\begin{tabular}{ll}
			\toprule
			\textbf{Variable/}\textbf{Parameter} & \textbf{Default Value}\\
			\hline
			init\_human\_pop & 2225\\
			init\_inf & 0.25 * int\_human\_pop\\
			init\_expo & exposed\_rate * int\_human\_pop \\
			agent\_speed & 50.0 cm/s\\
			infection\_distance & 1.8 meters\\
			exp\_rate & 0.25\\
			protection\_rate & 0.0\\
			proba\_infection & 1.0 - protection\_rate\\
			proba\_exposed & 0.50\\
			step & 10 minutes\\
			exposed\_period & 288 cycles to 1440 cycles\\
			infectious\_period & 1440 cycles to 2016 cycles\\
			recovered\_period & 2880 cycles to 8640 cycles\\ 
			\bottomrule
		\end{tabular}
		\caption{Variables used for the GAMA Simulation of the Scenario 1.1 which has a 0\% Protection Rate}
		\label{1.1}
	\end{table}
	
 	\begin{figure}[H]
	\centering
		\includegraphics[width=16cm, height=10cm]{images/PR1_G.png}
		\caption{Snapshot of the GAMA Simulation with a 0\% Protection Rate, featuring int\_inf = 556, int\_exp = 556, exp\_rate = 0.25, inf\_rate = 0.25, proba\_expo = 0.50, and a simulation duration of 31680 cycles. Rooms highlighted in red indicate the presence of an infected individual.}
		\label{PR1G}
	\end{figure}
Table \ref{1.1} provides a detailed overview of the parameters employed in the simulation of Scenario 1.1, which aims to scrutinize the ramifications of a zero protection rate on virus transmission dynamics. In this scenario, the absence of any protective measures signifies a scenario where minimal efforts are undertaken to contain the virus. At the outset of the simulation, as per the specifications outlined in Table \ref{1.1}, the initial population consists of 2225 individuals, with 25\% of this population designated as both infected and exposed, amounting to 557 individuals each. The infection distance, representing the proximity required for transmission, is established at 1.8 meters, while the probability of an agent becoming exposed is set at 0.50. Each time step within the simulation corresponds to a duration of 10 minutes, allowing for a granular examination of infection dynamics over time. The simulation is projected to persist for a duration of 31680 cycles, equivalent to 220 days, facilitating an extensive exploration of virus transmission patterns and outcomes under the specified conditions. 

	\subsubsection{ Scenario 1.2: 25\% Protection Rate}
	\begin{table}[H]
		\centering
		{\begin{tabular}{ll}
			\toprule
			\textbf{Variable/}\textbf{Parameter} & \textbf{Default Value}\\
			\hline
			init\_human\_pop & 2225\\
			init\_inf & 0.25 * int\_human\_pop\\
			init\_expo & exposed\_rate * int\_human\_pop \\
			agent\_speed & 50.0 cm/s\\
			infection\_distance & 1.8 meters\\
			exp\_rate & 0.25\\
			protection\_rate & 0.25\\
			proba\_infection & 1.0 - protection\_rate\\
			proba\_exposed & 0.50\\
			step & 10 minutes\\
			exposed\_period & 288 cycles to 1440 cycles\\
			infectious\_period & 1440 cycles to 2016 cycles\\
			recovered\_period & 2880 cycles to 8640 cycles\\ 
			\bottomrule
		\end{tabular}
		\caption{Variables used for the GAMA Simulation of the Scenario 1.1 which has a 0\% Protection Rate}
		\label{1.2}}
	\end{table}
	
 	\begin{figure}[H]
	\centering
	\includegraphics[width=16cm, height=10cm]{images/PR2_G.png}
	\caption{Snapshot of the GAMA Simulation with a 25\% Protection Rate, featuring int\_inf = 557, int\_exp = 557, exp\_rate = 0.25, inf\_rate = 0.25, proba\_expo = 0.50, and a simulation duration of 31680 cycles equivalent to 220 days. Rooms highlighted in red indicate the presence of an infected individual rooms highlighted in green indicates that there is no presence of infected person.}
	\label{PR2G}
	\end{figure}
Table \ref{1.2} provides a detailed overview of the parameters employed in the simulation of Scenario 1.2, which aims to observe the effect of a 25\% protection rate on virus transmission dynamics. In this scenario, minimal preventive measures are implemented here . At the outset of simulation, as per the specifications outlined in Table \ref{1.2}, the initial population consists of 2225 individuals, with 25\% of this are designated as both infected and exposed, amounting to 557 individuals each. The infection distance, representing the proximity required for transmission, is established at 1.8 meters, while the probability of an agent becoming exposed is set at 0.50. Each time step within the simulation corresponds to a duration of 10 minutes, allowing for a granular examination of infection dynamics over time. The simulation is projected to persist for a duration of 31680 cycles, equivalent to 220 days, facilitating an extensive exploration of virus transmission patterns and outcomes under the specified conditions. 

\subsubsection{ Scenario 1.3: 50\% Protection Rate}
\begin{table}[H]
	\centering
	{\begin{tabular}{ll}
			\toprule
			\textbf{Variable/Parameter} & \textbf{Default Value}\\
			\hline
			init\_human\_pop & 2225\\
			init\_inf & 0.25 * int\_human\_pop\\
			init\_expo & exposed\_rate * int\_human\_pop \\
			agent\_speed & 50.0 cm/s\\
			infection\_distance & 1.8 meters\\
			exp\_rate & 0.25\\
			protection\_rate & 0.50\\
			proba\_infection & 1.0 - protection\_rate\\
			proba\_exposed & 0.50\\
			step & 10 minutes\\
			exposed\_period & 288 cycles to 1440 cycles\\
			infectious\_period & 1440 cycles to 2016 cycles\\
			recovered\_period & 2880 cycles to 8640 cycles\\ 
			\bottomrule
		\end{tabular}
		\caption{Variables used for the GAMA Simulation of Scenario 1.2 with a 50\% Protection Rate}
		\label{1.3}}
\end{table}

 	\begin{figure}[H]
	\centering
	\includegraphics[width=16cm, height=10cm]{images/PR3_G.png}
	\caption{Snapshot of the GAMA Simulation with a 50\% Protection Rate, featuring int\_inf = 557, int\_exp = 557, exp\_rate = 0.25, inf\_rate = 0.25, proba\_expo = 0.50,  and a simulation duration of 31680 cycles equivalent to 220 days. Rooms highlighted in red indicate the presence of an infected individual rooms highlighted in green indicates that there is no presence of infected person.}
	\label{PR3G}
\end{figure}

Table \ref{1.2} provides a comprehensive overview of the parameters utilized in the simulation of Scenario 1.3, which aims to examine the impact of a 50\% protection rate on virus transmission dynamics within the school setting. In this scenario, moderate preventive measures are implemented. At the initiation of the simulation, as specified in Table \ref{1.3}, the initial population comprises 2225 individuals, with 25\% designated as both infected and exposed, totaling 557 individuals each. The infection distance, representing the proximity required for transmission, is set at 1.8 meters, while the probability of an agent becoming exposed is established at 0.50. Each time step within the simulation corresponds to a duration of 10 minutes, facilitating a detailed analysis of infection dynamics over time. The simulation is projected to run for a duration of 31680 cycles, equivalent to 220 days, enabling a thorough exploration of virus transmission patterns and outcomes under the specified conditions.

\subsubsection{Scenario 1.4: 75\% Protection Rate}
\begin{table}[H]
	\centering
	{\begin{tabular}{ll}
			\toprule
			\textbf{Variable/Parameter} & \textbf{Default Value}\\
			\hline
			init\_human\_pop & 2225\\
			init\_inf & 0.25 * int\_human\_pop\\
			init\_expo & exposed\_rate * int\_human\_pop \\
			agent\_speed & 50.0 cm/s\\
			infection\_distance & 1.8 meters\\
			exp\_rate & 0.25\\
			protection\_rate & 0.75\\
			proba\_infection & 1.0 - protection\_rate\\
			proba\_exposed & 0.50\\
			step & 10 minutes\\
			exposed\_period & 288 cycles to 1440 cycles\\
			infectious\_period & 1440 cycles to 2016 cycles\\
			recovered\_period & 2880 cycles to 8640 cycles\\ 
			\bottomrule
		\end{tabular}
		\caption{Variables used for the GAMA Simulation of Scenario 1.2 with a 75\% Protection Rate}
		\label{1.4}}
\end{table}

 	\begin{figure}[H]
	\centering
	\includegraphics[width=16cm, height=10cm]{images/PR4_G.png}
	\caption{Snapshot of the GAMA Simulation with a 75\% Protection Rate, featuring int\_inf = 557, int\_exp = 557, exp\_rate = 0.25, inf\_rate = 0.25, proba\_expo = 0.50, and a simulation duration of 31680 cycles. Rooms highlighted in red indicate the presence of an infected individual rooms highlighted in green indicates that there is no presence of infected person.}
	\label{PR4G}
\end{figure}

Table \ref{1.4} presents an extensive breakdown of the parameters employed in the simulation of Scenario 1.5, which investigates the consequences of a 75\% protection rate on virus transmission dynamics within the school environment. In this scenario, high preventive measures are in place. At the outset of the simulation, as outlined in Table \ref{1.4}, the initial population consists of 2225 individuals, with 25\% identified as both infected and exposed, amounting to 557 individuals each. The infection distance, denoting the proximity necessary for transmission, remains at 1.8 meters, while the likelihood of an agent being exposed is fixed at 0.50. Each time interval within the simulation spans 10 minutes, facilitating an intricate examination of infection dynamics over the course of the simulation. The simulation duration is set at 31680 cycles, corresponding to 220 days, allowing for a comprehensive exploration of virus transmission patterns and outcomes under the specified conditions.

\subsubsection{Scenario 1.5: 100\% Protection Rate}
\begin{table}[H]
	\centering
	{\begin{tabular}{ll}
			\toprule
			\textbf{Variable/Parameter} & \textbf{Default Value}\\
			\hline
			init\_human\_pop & 2225\\
			init\_inf & 0.25 * int\_human\_pop\\
			init\_expo & exposed\_rate * int\_human\_pop \\
			agent\_speed & 50.0 cm/s\\
			infection\_distance & 1.8 meters\\
			exp\_rate & 0.25\\
			protection\_rate & 1.0\\
			proba\_infection & 1.0 - protection\_rate\\
			proba\_exposed & 0.50\\
			step & 10 minutes\\
			exposed\_period & 288 cycles to 1440 cycles\\
			infectious\_period & 1440 cycles to 2016 cycles\\
			recovered\_period & 2880 cycles to 8640 cycles\\ 
			\bottomrule
		\end{tabular}
		\caption{Variables used for the GAMA Simulation of Scenario 1.2 with a 100\% Protection Rate}
		\label{1.5}}
\end{table}

\begin{figure}[H]
	\centering
	\includegraphics[width=16cm, height=10cm]{images/PR5_G.png}
	\caption{Snapshot of the GAMA Simulation with a 100\% Protection Rate, featuring int\_inf = 557, int\_exp = 557, exp\_rate = 0.25, inf\_rate = 0.25, proba\_expo = 0.50, and a simulation duration of 31680 cycles equivalent to 220 days. Rooms highlighted in red indicate the presence of an infected individual rooms highlighted in green indicates that there is no presence of infected person.}
	\label{PR5G}
\end{figure}

Table \ref{1.5} presents an extensive breakdown of the parameters employed in the simulation of Scenario 1.5, which investigates the consequences of a 100\% protection rate on virus transmission dynamics within the school environment. In this scenario, rigorous preventive measures are in place. At the outset of the simulation, as outlined in Table \ref{1.5}, the initial population consists of 2225 individuals, with 25\% identified as both infected and exposed, amounting to 557 individuals each. The infection distance, denoting the proximity necessary for transmission, remains at 1.8 meters, while the likelihood of an agent being exposed is fixed at 0.50. Each time interval within the simulation spans 10 minutes, facilitating an intricate examination of infection dynamics over the course of the simulation. The simulation duration is set at 31680 cycles, corresponding to 220 days, allowing for a comprehensive exploration of virus transmission patterns and outcomes under the specified conditions.

\subsection{ Scenario 2: Varying Initial Exposure Rate}
In this specific scenario, four distinct sub-scenarios were executed, each characterized by a distinct array of variables and conditions. The simulation entailed varying exposure rates, spanning from 0\% to 75                                                                                                                                       \%. The primary objective of this simulation was to garner precise insights into the influence of exposure rates on the overall scenario outcome. Through the execution of multiple sub-scenarios, the simulation effectively encompassed diverse factors and variables that could potentially impact the outcomes, facilitating a comprehensive analysis of the dynamics at play.

\subsubsection{Scenario 2.1: 0\% Initial Exposure Rate}
\begin{table}[H]
	\centering
	\begin{tabular}{ll}
		\toprule
		\textbf{Variable/Parameter} & \textbf{Default Value}\\
		\hline
		init\_human\_pop & 2225\\
		init\_inf & 0.25 * int\_human\_pop\\
		init\_expo & exp\_rate * int\_human\_pop \\
		agent\_speed & 50.0 cm/s\\
		infection\_distance & 1.8 meters\\
		exp\_rate & 0.0\\
		protection\_rate & 0.50\\
		proba\_infection & 1.0 - protection\_rate\\
		proba\_exposed & 0.50\\
		step & 10 minutes\\
		exposed\_period & 288 cycles to 1440 cycles\\
		infectious\_period & 1440 cycles to 2016 cycles\\
		recovered\_period & 2880 cycles to 8640 cycles\\ 
		\bottomrule
	\end{tabular}
	\caption{Variables used for the GAMA Simulation of Scenario 2.1 with a 0\% Initial Exposure Rate}
	\label{2.1a}
\end{table}
\begin{figure}[H]
	\centering
	\includegraphics[width=16cm, height=10cm]{images/ER1_G.png}
	\caption{Snapshot of the GAMA Simulation in a 0\% Initial Exposure Rate, featuring int\_inf = 557, int\_exp = 0, exp\_rate = 0.0, inf\_rate = 0.25, protection\_rate = 0.50, proba\_expo = 0.50, and a simulation duration of 31680 cycles equivalent to 220 days. Rooms highlighted in red indicate the presence of an infected individual rooms highlighted in green indicates that there is no presence of infected person.}
\label{ER1G}
\end{figure}

Table \ref{2.1a} presents a detailed breakdown of the parameters utilized in the simulation of Scenario 2.1, which investigates the consequences of a 0\% initial exposure rate on virus transmission dynamics. This scenario aims to understand the impact of no individuals being initially exposed to the virus. With a protection rate set at 0.50, representing moderate protective measures, the simulation aims to assess their efficacy. At the beginning of the simulation, in line with the specifications outlined in Table \ref{2.1a}, the initial population consists of 2225 individuals, with 25\% of this population identified as infected, totaling 557 individuals. The infection distance, defining the required proximity for transmission, is set at 1.8 meters, while the likelihood of an agent succumbing to exposure is established at 0.50. Each temporal iteration within the simulation corresponds to a duration of 10 minutes, allowing for a detailed examination of infection dynamics over time. The simulation is scheduled to run for 31680 cycles, equivalent to a span of 220 days, enabling a comprehensive investigation of virus transmission patterns and outcomes under specific conditions.
	\subsubsection{Scenario 2.2 25\% Initial Exposure Rate}
	\begin{table}[H]
		\centering
		\begin{tabular}{ll}
			\toprule
			\textbf{Variable/}\textbf{Parameter} & \textbf{Default Value}\\
			\hline
			init\_human\_pop & 2225\\
			init\_inf & 0.25 * int\_human\_pop\\
			init\_expo & exp\_rate * int\_human\_pop \\
			agent\_speed & 50.0 cm/s\\
			infection\_distance & 1.8 meters\\
			exp\_rate & 0.25\\
			protection\_rate & 0.50\\
			proba\_infection & 1.0 - protection\_rate\\
			proba\_exposed & 0.50\\
			step & 10 minutes\\
			exposed\_period & 288 cycles to 1440 cycles\\
			infectious\_period & 1440 cycles to 2016 cycles\\
			recovered\_period & 2880 cycles to 8640 cycles\\ 
			\bottomrule
		\end{tabular}
		\caption{Variables used for the GAMA Simulation of the Scenario 2.2 which has a 25\% Initial Exposure Rate}
		\label{2.1b}
	\end{table}
	
	\begin{figure}[H]
		\centering
		\includegraphics[width=16cm, height=10cm]{images/ER2_G.png}
		\caption{Snapshot of the GAMA Simulation in a 25\% Initial Exposure Rate, featuring int\_inf = 557, int\_exp = 557, exp\_rate = 0.25, inf\_rate = 0.25, protection\_rate = 0.50, proba\_expo = 0.50, and a simulation duration of 31680 cycles equivalent to 220 days. Rooms highlighted in red indicate the presence of an infected individual rooms highlighted in green indicates that there is no presence of infected person.}
		\label{ER2G}
	\end{figure}

Table \ref{2.1b} furnishes a comprehensive breakdown of the parameters employed in the simulation of Scenario 2.2, aimed at dissecting the implications of a 25\% exposure rate on virus transmission dynamics. This scenario endeavors to probe the consequences of assuming that a quarter or 557 of the population is initially exposed. With a protection rate set at 0.50, denoting minimal to moderate protection measures, the simulation endeavors to assess the impact of these measures. At the initiation of the simulation, in accordance with the specifications outlined in Table \ref{2.1b}, the initial population comprises 2225 individuals, with 25\% of this populace identified as infected, equating to 557 individuals. The infection distance, showing the proximity required for transmission, is set at 1.8 meters, while the likelihood of an agent succumbing to exposure is established at 0.50. Each iteration within the simulation corresponds to a duration of 10 minutes, facilitating a nuanced exploration of infection dynamics over the time. The simulation is conducted to run for 31680 cycles, translating to a span of 220 days, thereby enabling a thorough investigation of virus transmission patterns and outcomes under specific conditions.

\subsubsection{Scenario 2.3: 50\% Initial Exposure Rate}
\begin{table}[H]
	\centering
	\begin{tabular}{ll}
		\toprule
		\textbf{Variable/Parameter} & \textbf{Default Value}\\
		\hline
		init\_human\_pop & 2225\\
		init\_inf & 0.25 * int\_human\_pop\\
		init\_expo & exp\_rate * int\_human\_pop \\
		agent\_speed & 50.0 cm/s\\
		infection\_distance & 1.8 meters\\
		exp\_rate & 0.50\\
		protection\_rate & 0.50\\
		proba\_infection & 1.0 - protection\_rate\\
		proba\_exposed & 0.50\\
		step & 10 minutes\\
		exposed\_period & 288 cycles to 1440 cycles\\
		infectious\_period & 1440 cycles to 2016 cycles\\
		recovered\_period & 2880 cycles to 8640 cycles\\ 
		\bottomrule
	\end{tabular}
	\caption{Variables used for the GAMA Simulation of Scenario 2.3 with a 50\% Initial Exposure Rate}
	\label{2.1c}
\end{table}

	\begin{figure}[H]
	\centering
	\includegraphics[width=16cm, height=10cm]{images/ER2_G.png}
	\caption{Snapshot of the GAMA Simulation in a 25\% Initial Exposure Rate, featuring int\_inf = 557, int\_exp = 1113, exp\_rate = 0.50, inf\_rate = 0.25, protection\_rate = 0.50, proba\_expo = 0.50, and a simulation duration of 31680 cycles equivalent to 220 days. Rooms highlighted in red indicate the presence of an infected individual rooms highlighted in green indicates that there is no presence of infected person.}
	\label{ER4G}
\end{figure}

Table \ref{2.1c} provides an overview of the parameters utilized in the simulation of Scenario 2.3, which investigates the effects of a 50\% initial exposure rate on virus transmission dynamics. This scenario aims to understand the consequences of half of the population or 1113 individuals as initially exposed, representing moderate containment efforts. With a protection rate set at 0.50, indicating moderate protective measures, the simulation aims to assess their effectiveness. At the start of the simulation, as specified in Table \ref{2.1c}, the initial population comprises 2225 individuals, with 25\% identified as infected, totaling 557 individuals. The infection distance, representing the required proximity for transmission, is set at 1.8 meters, while the probability of an agent becoming exposed is established at 0.50. Each iteration within the simulation corresponds to a duration of 10 minutes, facilitating a detailed analysis of infection dynamics over time. The simulation is projected to run for 31680 cycles, equivalent to 220 days, enabling a comprehensive investigation of virus transmission patterns and outcomes under specific conditions.


\subsubsection{Scenario 2.4: 75\% Initial Exposure Rate}
\begin{table}[H]
	\centering
	\begin{tabular}{ll}
		\toprule
		\textbf{Variable/Parameter} & \textbf{Default Value}\\
		\hline
		init\_human\_pop & 2225\\
		init\_inf & 0.25 * int\_human\_pop\\
		init\_expo & exp\_rate * int\_human\_pop \\
		agent\_speed & 50.0 cm/s\\
		infection\_distance & 1.8 meters\\
		exp\_rate & 0.75\\
		protection\_rate & 0.50\\
		proba\_infection & 1.0 - protection\_rate\\
		proba\_exposed & 0.50\\
		step & 10 minutes\\
		exposed\_period & 288 cycles to 1440 cycles\\
		infectious\_period & 1440 cycles to 2016 cycles\\
		recovered\_period & 2880 cycles to 8640 cycles\\ 
		\bottomrule
	\end{tabular}
	\caption{Variables used for the GAMA Simulation of Scenario 2.4 with a 75\% Initial Exposure Rate}
	\label{2.1d}
\end{table}

	\begin{figure}[H]
	\centering
	\includegraphics[width=16cm, height=10cm]{images/ER4_G.png}
	\caption{Snapshot of the GAMA Simulation in a 75\% Initial Exposure Rate, featuring int\_inf = 557, int\_exp = 1669, exp\_rate = 0.75, inf\_rate = 0.25, protection\_rate = 0.50, proba\_expo = 0.50, and a simulation duration of 31680 cycles equivalent to 220 days. Rooms highlighted in red indicate the presence of an infected individual rooms highlighted in green indicates that there is no presence of infected person.}
	\label{ER3G}
\end{figure}

Table \ref{2.1d} outlines the parameters employed in the simulation of Scenario 2.2, focusing on the effects of a 75\% initial exposure rate on virus transmission dynamics. This scenario aims to understand the consequences of three-quarters or 1669 of the population being initially exposed, reflecting moderate containment efforts. With a protection rate set at 0.50, indicating moderate protective measures, the simulation seeks to assess their effectiveness. At the beginning of the simulation, as specified in Table \ref{2.1d}, the initial population comprises 2225 individuals, with 25\% identified as both infected, totaling 557 individuals. The infection distance, representing the required proximity for transmission, is set at 1.8 meters, while the probability of an agent becoming exposed is established at 0.50. Each iteration within the simulation corresponds to a duration of 10 minutes, allowing for a detailed analysis of infection dynamics over time. The simulation is projected to run for 31680 cycles, equivalent to 220 days, enabling a comprehensive investigation of virus transmission patterns and outcomes under specific conditions.



\subsection{ Scenario 3: Implementation of Lockdown and Quarantine Area}
In this scenario, we conducted four sub-scenarios, each with distinct conditions, to investigate the effects of implementing Non-Pharmaceutical Interventions (NPIs) on disease transmission dynamics. The simulation varied by implementing Lockdown and Quarantine Areas, aiming to understand how different combinations of these interventions impact the spread of the disease. The scenarios included: No Lockdown and No Quarantine Area, Lockdown Implementation without Quarantine Area, No Lockdown but with Quarantine Area Implementation, and finally, Implementation of both Lockdown and Quarantine Areas. The objective of this simulation was to provide valuable insights into the effectiveness of NPIs in controlling the spread of infectious diseases and inform public health strategies for managing future outbreaks 

	\subsubsection{ Scenario 3.1 No Lockdown and No Quarantine Area}
	\label{S3.1}

	\begin{table}[H]
		\centering
		\begin{tabular}{ll}
			\toprule
			\textbf{Variable/}\textbf{Parameter} & \textbf{Default Value}\\
			\hline
			init\_human\_pop & 2225\\
			init\_inf & 0.10 * int\_human\_pop\\
			init\_expo & exp\_rate * int\_human\_pop \\
			agent\_speed & 50.0 cm/s\\
			infection\_distance & 1.8 meters\\
			exp\_rate & 0.10\\
			protection\_rate & 0.50\\
			proba\_infection & 1.0 - protection\_rate\\
			proba\_exposed & 0.50\\
			step & 10 minutes\\
			exposed\_period & 288 cycles to 1440 cycles\\
			infectious\_period & 1440 cycles to 2016 cycles\\
			recovered\_period & 2880 cycles to 8640 cycles\\ 
			lockdown & false\\
			\bottomrule
		\end{tabular}
		\caption{Variables used for the GAMA Simulation of Scenario 3.1 where there's no implementation of Lockdown and Quarantine Area}
		\label{3.1}
	\end{table}
		\begin{figure}[H]
		\centering
		\includegraphics[width=16cm, height=10cm]{images/NQNLD_G.png}
		\caption{Snapshot of the GAMA Simulation in a scenario where lockdown and quarantine are are not implemented, featuring int\_inf = 557, int\_exp = 557, exp\_rate = 0.25, inf\_rate = 0.25, protection\_rate = 0.50, proba\_expo = 0.50, and a simulation duration of 31680 cycles equivalent to 220 days. Rooms highlighted in red indicate the presence of an infected individual rooms highlighted in green indicates that there is no presence of infected person.}
		\label{ER5G}
	\end{figure}
	In this scenario, denoted as Scenario 3.1, the absence of both lockdown measures and a designated quarantine area implies that agents within the simulation have unrestricted mobility throughout the campus during designated intervals such as break times, lunch breaks, and the conclusion of classes. The absence of a quarantine area further worsens the situation, as infected agents remain in close proximity to susceptible individuals without any form of containment. Consequently, the potential for disease transmission is heightened, as infected agents have ample opportunity to come into contact with susceptible individuals during their interactions within the campus environment. This scenario mirrors real-world scenarios where lack of containment measures and unrestricted movement contribute to the rapid spread of infectious diseases within populations. The parameter for this certain scenario is shown in Table \ref{3.1}.
	
	\begin{figure}[H]
		\centering
		\includegraphics[width=12cm, height=11cm]{images/DNQNLD.png}
		\caption{Table showing the daily time schedule indicating whether an agent can move or not in a scenario where neither lockdown nor quarantine areas are implemented.}
		\label{3.1a} 
	\end{figure}

Figure \ref{3.1a} shows the time schedule for a day, it shows the time when an agent is allowed to go out or not.

\begin{itemize}
\item \textbf{Start of the Day:} The day begins between 7:00 AM and 7:30 AM, allowing agents to freely move around before class starts. During this time, agents can proceed to their respective classrooms or explore other areas within the simulation.

\item \textbf{Class Time}: Class time is divided into morning and afternoon sessions. The morning session runs from 7:30 AM to 9:30 AM, with a break until 10:30 AM, followed by classes until 12:00 PM. In the afternoon, classes are held from 1:00 PM to 4:00 PM. Agents are restricted to movement within their classrooms during class hours.

\item \textbf{Break Time}: The break period, from 9:30 AM to 10:30 AM, allows agents to move outside their classrooms. During this time, agents may choose to have snacks, buy refreshments, or remain within the classroom.

\item \textbf{Lunch Time} Lunchtime spans from 12:00 PM to 1:00 PM, providing agents with an opportunity to eat lunch either indoors or outdoors. They are permitted to move freely around the campus during this period.

\item \textbf{End of the day} Between 4:00 PM and 6:00 PM marks the end of the day, allowing agents to leave for home. This signals the conclusion of activities for the agents within the simulation.
\end{itemize}

\subsubsection{ Scenario 3.2 Implementation of Lockdown and No Quarantine Area}

	\begin{table}[H]
	\centering
	\begin{tabular}{ll}
		\toprule
		\textbf{Variable/}\textbf{Parameter} & \textbf{Default Value}\\
		\hline
		init\_human\_pop & 2225\\
		init\_inf & 0.10 * int\_human\_pop\\
		init\_expo & exp\_rate * int\_human\_pop \\
		agent\_speed & 50.0 cm/s\\
		infection\_distance & 1.8 meters\\
		exp\_rate & 0.10\\
		protection\_rate & 0.50\\
		proba\_infection & 1.0 - protection\_rate\\
		proba\_exposed & 0.50\\
		step & 10 minutes\\
		exposed\_period & 288 cycles to 1440 cycles\\
		infectious\_period & 1440 cycles to 2016 cycles\\
		recovered\_period & 2880 cycles to 8640 cycles\\ 
		lockdown & true\\
		quarantine\_area & false\\
		\bottomrule
	\end{tabular}
	\caption{Variables used for the GAMA Simulation of Scenario 3.2 where there's an implementation of Lockdown but no Quarantine Area}
	\label{3.2b}
\end{table}
		\begin{figure}[H]
	\centering
	\includegraphics[width=16cm, height=10cm]{images/LDNQ_G.png}
	\caption{Snapshot of the GAMA Simulation in a scenario where lockdown is implemented but no quarantine area, featuring int\_inf = 556, int\_exp = 556, exp\_rate = 0.25, inf\_rate = 0.25, protection\_rate = 0.50, proba\_expo = 0.50, and a simulation duration of 31680 cycles equivalent to 220 days. Rooms highlighted in red indicate the presence of an infected individual rooms highlighted in green indicates that there is no presence of infected person.}
	\label{3.2G}
\end{figure}
In Scenario \ref{3.2b}, the implementation of a lockdown and the absence of a designated quarantine area imply that agents within the simulation experience restricted mobility throughout the campus during specified intervals such as break times and lunch breaks. The lack of a quarantine area heightens the situation, as infected agents remain in close proximity to susceptible individuals without any form of containment. This scenario reflects real-world situations where containment measures aim to minimize the transmission of respiratory infectious diseases. The parameters used for this simulation is summarized in table \ref{3.2B}

\begin{figure}[H]
	\centering
	\includegraphics[width=12cm, height=11cm]{images/DLDNQ.png}
	\caption{Table showing the daily time schedule indicating whether an agent can move or not in a scenario where there is no lockdown but a quarantine areas are implemented.}
	\label{3.2a} 
\end{figure}

Figure \ref{3.2a} shows the time schedule for a day, it shows the time when an agent is allowed to go out or not.


\begin{itemize}
	\item \textbf{Start of the Day:} The day begins between 7:00 AM and 7:30 AM, allowing agents to freely move around before class starts. During this time, agents can proceed to their respective classrooms or explore other areas within the simulation.
	
	\item \textbf{Class Time}: Class time is divided into morning and afternoon sessions. The morning session runs from 7:30 AM to 9:30 AM, with a break until 10:30 AM, followed by classes until 12:00 PM. In the afternoon, classes are held from 1:00 PM to 4:00 PM. Agents are restricted to movement within their classrooms during class hours.
	
	\item \textbf{Break Time}: The break period, from 9:30 AM to 10:30 AM. Contrary to Scenario 3.1 during break time, the agents are not allowed to go out of their assigned classroom. This means that the students are only allowed to eat their snack inside the classroom.
	
	\item \textbf{Lunch Time} Lunchtime spans from 12:00 PM to 1:00 PM. Similar to break time, the students are also not allowed to go outside. They may only eat their lunch inside the classroom. 
	
	\item \textbf{End of the day} Between 4:00 PM and 6:00 PM marks the end of the day, allowing agents to leave for home. This signals the conclusion of activities for the agents within the simulation.
\end{itemize}

\subsubsection{ Scenario 3.3 Implementation of Quarantine Area but No Lockdown}

\begin{table}[H]
	\centering
	\begin{tabular}{ll}
		\toprule
		\textbf{Variable/}\textbf{Parameter} & \textbf{Default Value}\\
		\hline
		init\_human\_pop & 2225\\
		init\_inf & 0.10 * int\_human\_pop\\
		init\_expo & exp\_rate * int\_human\_pop \\
		agent\_speed & 50.0 cm/s\\
		infection\_distance & 1.8 meters\\
		exp\_rate & 0.10\\
		protection\_rate & 0.50\\
		proba\_infection & 1.0 - protection\_rate\\
		proba\_exposed & 0.50\\
		step & 10 minutes\\
		exposed\_period & 288 cycles to 1440 cycles\\
		infectious\_period & 1440 cycles to 2016 cycles\\
		recovered\_period & 2880 cycles to 8640 cycles\\ 
		lockdown & false\\
		quarantine\_area & true\\
		\bottomrule
	\end{tabular}
	\caption{Variables used for the GAMA Simulation of Scenario 3.3 where there's an implementation of Quarantine Area but no Lockdown}
	\label{3.2B}
\end{table}
		\begin{figure}[H]
	\centering
	\includegraphics[width=16cm, height=10cm]{images/QNLD_G.png}
	\caption{Snapshot of the GAMA Simulation in a scenario where Quarantine Area is implemented but not lockdown, featuring int\_inf = 557, int\_exp = 557, exp\_rate = 0.25, inf\_rate = 0.25, protection\_rate = 0.50, proba\_expo = 0.50, and a simulation duration of 31680 cycles equivalent to 220 days. Rooms highlighted in red indicate the presence of an infected individual rooms highlighted in green indicates that there is no presence of infected person.}
	\label{3.3G}
\end{figure}
In Scenario \ref{3.2B}, the methodology entails the absence of a lockdown measure but includes the integration of a designated quarantine area within the simulation environment. Agents experience limited mobility throughout the campus during designated intervals such as break times and lunch breaks. Despite the absence of a lockdown measure, the presence of a quarantine area facilitates the isolation of infected individuals from susceptible ones. This setup aims to replicate real-world scenarios where containment measures are implemented to minimize the transmission of respiratory infectious diseases. The designated quarantine area serves as a mechanism to isolate infected individuals, allowing for an examination of how containment measures, particularly quarantine, affect disease transmission dynamics within a simulated school setting.

\begin{figure}[H]
	\centering
	\includegraphics[width=12cm, height=11cm]{images/DQNLD.png}
	\caption{Table showing the daily time schedule indicating whether an agent can move or not in a scenario where there's no lockdown but quarantine areas are implemented.}
	\label{3.2c} 
\end{figure}

Figure \ref{3.2c} shows the time schedule for a day, it shows the time when an agent is allowed to go out or not.


\begin{itemize}
	\item \textbf{Start of the Day:} The day begins between 7:00 AM and 7:30 AM, allowing agents to freely move around before class starts. During this time, agents can proceed to their respective classrooms or explore other areas within the simulation. 
	
	\item \textbf{Class Time}: Class time is divided into morning and afternoon sessions. The morning session runs from 7:30 AM to 9:30 AM, with a break until 10:30 AM, followed by classes until 12:00 PM. In the afternoon, classes are held from 1:00 PM to 4:00 PM. Agents are restricted to movement within their classrooms during class hours. However, if there's an infected students, they would go to the designated quarantine area.
	
	\item \textbf{Break Time}: The break period, from 9:30 AM to 10:30 AM, allows agents to move outside their classrooms. During this time, agents may choose to have snacks, buy refreshments, or remain within the classroom. In the event of an infected student, they will be directed to the designated quarantine area.
	
	\item \textbf{Lunch Time} Lunchtime spans from 12:00 PM to 1:00 PM, providing agents with an opportunity to eat lunch either indoors or outdoors. They are permitted to move freely around the campus during this period. In the event of an infected student, they will be directed to the designated quarantine area.
	
	\item \textbf{End of the day} Between 4:00 PM and 6:00 PM marks the end of the day, allowing agents to leave for home. This signals the conclusion of activities for the agents within the simulation.
\end{itemize}
	
\subsubsection{ Scenario 3.4 Implementation of Lockdown and No Quarantine Area}

\begin{table}[H]
	\centering
	\begin{tabular}{ll}
		\toprule
		\textbf{Variable/}\textbf{Parameter} & \textbf{Default Value}\\
		\hline
		init\_human\_pop & 2225\\
		init\_inf & 0.10 * int\_human\_pop\\
		init\_expo & exp\_rate * int\_human\_pop \\
		agent\_speed & 50.0 cm/s\\
		infection\_distance & 1.8 meters\\
		exp\_rate & 0.10\\
		protection\_rate & 0.50\\
		proba\_infection & 1.0 - protection\_rate\\
		proba\_exposed & 0.50\\
		step & 10 minutes\\
		exposed\_period & 288 cycles to 1440 cycles\\
		infectious\_period & 1440 cycles to 2016 cycles\\
		recovered\_period & 2880 cycles to 8640 cycles\\ 
		lockdown & true\\
		quarantine\_area & true\\
		\bottomrule
	\end{tabular}
	\caption{Variables used for the GAMA Simulation of Scenario 3.4 where there's an implementation of Lockdown and Quarantine Area}
	\label{3.2D}
\end{table}
		\begin{figure}[H]
	\centering
	\includegraphics[width=16cm, height=10cm]{images/QLD_G.png}
	\caption{Snapshot of the GAMA Simulation in a scenario where Quarantine Area and Lockdown are both implemented, featuring int\_inf = 557, int\_exp = 557, exp\_rate = 0.25, inf\_rate = 0.25, protection\_rate = 0.50, proba\_expo = 0.50, and a simulation duration of 31680 cycles equivalent to 220 days. Rooms highlighted in red indicate the presence of an infected individual rooms highlighted in green indicates that there is no presence of infected person.}
	\label{3.4G}
\end{figure}
In Scenario \ref{3.2D}, the methodology involves the presence of a lockdown alongside the incorporation of a designated quarantine area within the simulation environment. Agents experience restricted mobility throughout the campus during specified intervals such as break times and lunch breaks due to the lockdown measure. However, despite the presence of a lockdown, the designated quarantine area enables the isolation of infected individuals from susceptible ones. This setup aims to mimic real-world scenarios where containment measures, including lockdowns and quarantine areas, are established to minimize the transmission of respiratory infectious diseases. The designated quarantine area serves as a means to isolate infected individuals, contributing to the examination of how containment measures influence disease transmission dynamics within a simulated school setting.

\begin{figure}[H]
	\centering
	\includegraphics[width=12cm, height=11cm]{images/DLDQ.png}
	\caption{Table showing the daily time schedule indicating whether an agent can move or not in a scenario where there's an implementation of Lockdown and Quarantine Area}
	\label{3.2C} 
\end{figure}

Figure \ref{3.2C} shows the time schedule for a day, it shows the time when an agent is allowed to go out or not.


\begin{itemize}
	\item \textbf{Start of the Day:} The day begins between 7:00 AM and 7:30 AM, allowing agents to freely move around before class starts. During this time, agents can proceed to their respective classrooms or explore other areas within the simulation. However, if there's an infected students, they would go to the designated quarantine area.
	
	\item \textbf{Class Time}: Class time is divided into morning and afternoon sessions. The morning session runs from 7:30 AM to 9:30 AM, with a break until 10:30 AM, followed by classes until 12:00 PM. In the afternoon, classes are held from 1:00 PM to 4:00 PM. Agents are restricted to movement within their classrooms during class hours.
	
	\item \textbf{Break Time}: The break period, from 9:30 AM to 10:30 AM. Contrary to Scenario 3.1 during break time, the agents are not allowed to go out of their assigned classroom. This means that the students are only allowed to eat their snack inside the classroom. In the event of an infected student, they will be directed to the designated quarantine area.
	
	\item \textbf{Lunch Time} Lunchtime spans from 12:00 PM to 1:00 PM. Similar to break time, the students are also not allowed to go outside. They may only eat their lunch inside the classroom. In the event of an infected student, they will be directed to the designated quarantine area.
	
	\item \textbf{End of the day} Between 4:00 PM and 6:00 PM marks the end of the day, allowing agents to leave for home. This signals the conclusion of activities for the agents within the simulation.
\end{itemize}

\section{Gathering of Data}
Real population data were gathered from Pines City National High School, revealing a total of 87 teaching personnel, 37 non-teaching personnel, and 2136 students. Thus, in this experiment, the researcher utilized a population of 2225 agents. Among them, 25\% were initially considered infected.

The simulation also spans 31680 cycles, equivalent to a duration of 220 days. These days align with the number of school days prescribed by the Department of Education, ensuring a comprehensive and realistic representation of the educational calendar within the simulation.

\chapter{Results and Discussion}
\label{chap:Results and Discussion}

\indent \indent Multiple runs were conducted to gather data for addressing the identification problem at hand. Each scenario was carefully designed to test the effects of varying parameters on the transmission dynamics of a respiratory infectious disease within the PCNHS campus.

For robustness, three runs were executed for each sub-scenario, and the average counts of susceptible, exposed, infected, and recovered individuals were calculated from these runs. These averages were then compared across different sub-scenarios to discern the behaviors exhibited by each group of individuals in response to the parameter changes.

The findings and subsequent analysis shed light on the nuanced behaviors of individuals under different implemented scenarios. This helps in understanding how changes in parameters influence the transmission patterns of infectious disease, offering valuable insights for designing effective intervention strategies and mitigating disease spread within the campus setting.


\section{ Scenario 1: Varying Protection Rates}
\label{S1}

\subsection{ Scenario 1.1 SEIR Dynamics in a 0\% Protection Rate}
\begin{figure}[H]
	\centering
	\includegraphics[width=16cm, height=4cm]{images/PR_0.png}
	\caption{Plot showing the number of susceptible, exposed, infected, and recovered individuals over time across 0\% Protection rate. }
	\label{fig:10}
\end{figure}

\begin{table} [H]
	\centering
	\begin{tabular}{|l|l|l|l|l|l|l|l|}
		
		\hline
		\multicolumn{8}{|c|}{\textbf{0\% Protection Rate}}\\
		\hline
		\multicolumn{2}{|c|}{\textbf{S}} &  \multicolumn{2}{c|}{\textbf{E}}&  \multicolumn{2}{c|}{\textbf{I}}&  \multicolumn{2}{c|}{\textbf{R}}\\
		\hline
		\%& Day & \% & Day & \%  & Day & \% & Day \\
		\hline
		56.85\% & 1 & 79.77\%  & 3 & 93.1  & 11 & 99.05  &33 \\
		\hline
	\end{tabular}
	\caption{Maximum percentage of susceptible, exposed, infected, and recovered individuals across a 220-day period, where \textit{basenum\_human} = 2225 individuals, \textit{time\_cycle} = 31680 cycles, \textit{init\_expo} = 25\%, \textit{init\_inf} = 25\%,  \textit{protection\_rate} = 0\%.}
	\label{PR1_Max}
\end{table}
From table \ref{PR1_Max} and Figure \ref{fig:10}, we can observe the following maximum percentage number of each state across the 220 day period. It can be seen that 
\begin{itemize}
	\item The peak percentage of susceptible individuals reached 56.85\%, occurring on Day 1, with approximately 1265 individuals affected.
	\item The peak percentage of exposed individuals reached 79.77\%, occurring on Day 3, with approximately 1775 individuals affected.
	\item The peak percentage of infected individuals reached 93.12\%, occurring on Day 11, with approximately 2072 individuals affected.
	\item The peak percentage of recovered individuals reached 99.05\%, occurring on Day 33, with approximately 2225 individuals affected.
\end{itemize}

Referring to Figure\ref{fig:10}, the population consisted of a total of 2225 individuals, with 25\% assumed to be infected and an additional 25\% considered exposed. A 0\% protection rate was applied to the simulation, which means that no minimal effort was made to contain the transmission of the virus. The simulation starts on day 1, and the figure shows a gradual increase in the number of infected individuals from days 2 to 10, suggesting that exposed individuals transitioned to the infected state after an incubation period ranging between 2 to 10 days. Notably, the peak infection rate was recorded on day 11, with 2072 individuals infected. Calculating the highest infection rate across the 220-day period yields a value of 93.12\%, derived from the formula:

\[
\%Inf_{max} = \frac{2072}{2225} \times 100\%  = 93.12\%
\]

Moreover, a noteworthy observation is the rise in the number of recovered individuals beginning on day 12, indicating the end of the infectious period (typically spanning 10 to 14 days). From no recovered agents from day 12 to reaching its highest number of recovered agents at day 33 with 2225 number of recovered agents. 

Interestingly, a secondary surge in infections occurs after the 20 to 60-day immunity period of those who recovered. This marks the onset of a second wave of infections starting on day 55, which gradually subsides by day 75 as the number of recovered individuals increases. This cyclic pattern repeats, manifesting as a third wave starting from day 108.

This scenario, characterized by a 0\% protection rate, highlights the occurrence of multiple infection waves, each with its peak infection rate of 93.12\%. The analysis underscores the importance of protective measures and immunity in controlling infectious disease outbreaks within populations. 

\subsection{ Scenario 1.2 SEIR Dynamics in a 25\% Protection Rate}

\begin{figure}[H]
	\centering
	\includegraphics[width=16cm, height=6cm]{images/PR_25.png}
	\caption{Plot showing the number of susceptible, exposed, infected, and recovered individuals over time across 50\% Protection rate. }
	\label{fig:11b}
\end{figure}

\begin{table} [H]
	\centering
	{\begin{tabular}{|l|l|l|l|l|l|l|l|}
		\hline
		\multicolumn{8}{|c|}{\textbf{25\% Protection Rate}}\\
		\hline
		\multicolumn{2}{|c|}{\textbf{S}} &  \multicolumn{2}{c|}{\textbf{E}}&  \multicolumn{2}{c|}{\textbf{I}}&  \multicolumn{2}{c|}{\textbf{R}}\\
		\hline
		\%& Day & \% & Day & \%  & Day & \% & Day \\
		\hline
		56.85\% & 1 & 79.73\%  & 3 &  80.26  & 11 & 93.88  &39 \\
		\hline
	\end{tabular}
	\caption{Maximum percentage of susceptible, exposed, infected, and recovered individuals across a 220-day period, where \textit{basenum\_human} = 2225 individuals, \textit{time\_cycle} = 31680 cycles, \textit{init\_expo} = 25\%, \textit{init\_inf} = 25\%,  \textit{protection\_rate} = 25\%.}}
	\label{PR2_Max}
\end{table}

In this simulation scenario, a 25\% protection rate was applied to investigate its effect on the number of infected individuals within the population. Similar to Scenario 1.1, the initial conditions on day 1 included 25\%  of the population assumed to be exposed and an additional 25\%  assumed to be infected.

From table \ref{PR2_Max} and Figure \ref{fig:11b}, we can observe the following maximum percentage number of each state across the 220 day period. It can be seen that 

\begin{itemize}
	\item The peak percentage of susceptible individuals reached 56.85\%, occurring on Day 1, with approximately 1265 susceptible agents.
	\item The peak percentage of exposed individuals reached 79.77\%, occurring on Day 3, with approximately 1775 exposed agents.
	\item The peak percentage of infected individuals reached 80.26\%, occurring on Day 11, with approximately 1784 individuals affected.
	\item The peak percentage of recovered individuals reached 93.88\%, occurring on Day 39, with approximately 2089 recovered agents.
\end{itemize}

The simulation revealed a notable trend where the number of infected individuals began to rise on day 2 following an incubation period of 2 to 10 days, during which exposed individuals either became infected or remained susceptible. The peak of infected cases occurred on day 12, totaling 1784 individuals, resulting in a peak percentage of of 80.2\%, calculated using the formula:

\[ \%Inf_{max}  = \frac{1784}{2225} \times 100\%  = 80.26\% \]

Subsequently, the number of infected individuals gradually declined, reaching a lower point by day 43. Conversely, the count of recovered individuals started to rise from day 12, peaking at day 39 as individuals recovered from the infection.

However, as the number of immune individuals declined over time, a resurgence in infections was observed starting from day 50. This resurgence marked the end of the immunity period ranging from 20 to 60 days, during which previously infected individuals became susceptible once again, initiating a second wave of infections. Similarly, a subsequent third wave occurred following the waning immunity period of individuals who recovered from the second wave.

A third and fourth wave of infection was also observed, although they have a lower crest compared to the first two waves, we can still see a resurgence of infection. 

This detailed analysis underscores the dynamic nature of infectious disease transmission within a population under a 25\% protection rate scenario. 

\subsection{ Scenario 1.3 SEIR Dynamics in a 50\% Protection Rate}

\begin{figure}[H]
	\centering
	\includegraphics[width=16cm, height=6cm]{images/PR_50.png}
	\caption{Plot showing the number of susceptible, exposed, infected, and recovered individuals over time across 50\% Protection rate. }
	\label{fig:12}
\end{figure}

\begin{table} [H]
	\centering
	\begin{tabular}{|l|l|l|l|l|l|l|l|}
		
		\hline
		\multicolumn{8}{|c|}{\textbf{50\% Protection Rate}}\\
		\hline
		\multicolumn{2}{|c|}{\textbf{S}} &  \multicolumn{2}{c|}{\textbf{E}}&  \multicolumn{2}{c|}{\textbf{I}}&  \multicolumn{2}{c|}{\textbf{R}}\\
		\hline
		\%& Day & \% & Day & \%  & Day & \% & Day \\
		\hline
		57.34\% & 1 &80.18\%  & 3 &  62.43  & 19 & 85.97  &44\\
		\hline
	\end{tabular}
	\caption{Maximum percentage of susceptible, exposed, infected, and recovered individuals across a 220-day period, where \textit{basenum\_human} = 2225 individuals, \textit{time\_cycle} = 31680 cycles, \textit{init\_expo} = 25\%, \textit{init\_inf} = 25\%,  \textit{protection\_rate} = 50\%.}
	\label{PR3_Max}
\end{table}


From table \ref{PR3_Max} and Figure \ref{fig:12}, we can observe the following maximum percentage number of each state across the 220 day period. It can be seen that 

\begin{itemize}
	\item The peak percentage of susceptible individuals reached 57.34\%, occurring on Day 1, with approximately 1276 susceptible agents.
	\item The peak percentage of exposed individuals reached 80.18\%, occurring on Day 3, with approximately 1784 exposed agents.
	\item The peak percentage of infected individuals reached 62.43\%, occurring on Day 19, with approximately 1389 individuals affected.
	\item The peak percentage of recovered individuals reached 85.97\%, occurring on Day 44, with approximately 1913 recovered agents.
\end{itemize}

This scenario delves into the dynamics of infection within a population under a 50\% protection rate, aiming to understand how this level of protection impacts disease transmission. Similar to the 25\% protection rate scenario and Scenario 1.1, the initial conditions on day 1 included 25\% of the population assumed to be exposed, alongside another 25\% assumed to be infected.

The simulation data from Figure \ref{fig:12} unveils a distinct trend where the number of infected individuals begins to ascend on day 2 post an incubation period lasting 2 to 10 days. This incubation window dictates whether exposed individuals progress to the infected state or remain susceptible. Interestingly, the peak of infected cases is noted on day 19, totaling 1389 individuals, yielding an infection rate of 62.40\%, derived from:

\[ \%Inf_{max} = \frac{1389}{2225} \times 100\%  = 62.40\% \]

Following this peak, at day 19 a gradual decline in the number of infected individuals is observed, reaching a trough by day 49 indicating that the 10 to 14 days of infectious period is over. Concurrently, the count of recovered individuals starts to climb from day 10, peaking at day 46 as individuals recover from the infection.

However, a resurgence in infections is witnessed from day 51 onwards, as the number of immune individuals dwindles over time. This resurgence signifies the conclusion of the immunity period spanning 20 to 60 days, during which previously infected individuals regain susceptibility, initiating a second wave of infections. A subsequent third wave emerges following the waning immunity period of individuals who recovered from the second wave.

This detailed analysis underscores the intricate interplay of immunity dynamics and infection cycles within a population under a 50\% protection rate scenario.

\subsection{Scenario 1.4 SEIR Dynamics in a 75\% Protection Rate}
\begin{figure}[H]
	\centering
	\includegraphics[width=16cm, height=7cm]{images/PR_75.png}
	\caption{Plot showing the number of susceptible, exposed, infected, and recovered individuals over time across 75\% Protection rate. }
	\label{fig:13S}
\end{figure}


\begin{table} [H]
	\centering
	\begin{tabular}{|l|l|l|l|l|l|l|l|}
		
		\hline
		\multicolumn{8}{|c|}{\textbf{75\% Protection Rate}}\\
		\hline
		\multicolumn{2}{|c|}{\textbf{S}} &  \multicolumn{2}{c|}{\textbf{E}}&  \multicolumn{2}{c|}{\textbf{I}}&  \multicolumn{2}{c|}{\textbf{R}}\\
		\hline
		\%& Day & \% & Day & \%  & Day & \% & Day \\
		\hline
		56.67\% & 1 &80.45\%  & 3 &  39.64  & 19 & 65.30  &49\\
		\hline
	\end{tabular}
	\caption{Maximum percentage of susceptible, exposed, infected, and recovered individuals across a 220-day period, where \textit{basenum\_human} = 2225 individuals, \textit{time\_cycle} = 31680 cycles, \textit{init\_expo} = 25\%, \textit{init\_inf} = 25\%,  \textit{protection\_rate} = 75\%.}
	\label{PR4_Max}
\end{table}


From table \ref{PR3_Max} and Figure \ref{fig:13}, we can observe the following maximum percentage number of each state across the 220 day period. It can be seen that 

\begin{itemize}
	\item The peak percentage of susceptible individuals reached 56.67\%, occurring on Day 1, with approximately 1261 susceptible agents.
	\item The peak percentage of exposed individuals reached 80.45\%, occurring on Day 3, with approximately 1791 exposed agents.
	\item The peak percentage of infected individuals reached 39.64\%, occurring on Day 11, with approximately 881 individuals affected.
	\item The peak percentage of recovered individuals reached 65.30\%, occurring on Day 49, with approximately 1452 recovered agents.
\end{itemize}

This simulation delves into the intricate dynamics of infection within a population characterized by a robust 75\% protection rate, aiming to unravel the ways in which this high level of protection influences the transmission dynamics of the disease. Analogous to the scenarios featuring a 75\% protection rate and Scenario 1.3, the initial conditions on day 1 encompassed 25\% of the population presumed to be exposed, alongside an additional 25\% presumed to be infected.

Analyzing Figure \ref{fig:13}, we observe that the peak of infection occurs on day 21. The peak percentage for this particular day peaks at 39.64\% meaning that a recorded count of 876 infected individuals, calculated using the formula:

\[ \%Inf_{max}  = \frac{876}{2225} * 100\% = 39.64\% \]

Moreover, starting from day 14, a gradual recovery trend becomes apparent among the population, with the highest number of recovered individuals documented on day 49.

Although there is a slight uptick in infections post-recovery phase, these fluctuations do not qualify as second or third waves of infection. This assertion is grounded in the observation that the infection numbers remain relatively stable and consistently lower than the peak count of infected individuals.

\subsection{ Scenario 1.5 SEIR Dynamics in a 100\% Protection Rate}

\begin{figure}[H]
	\centering
	\includegraphics[width=16cm, height=8cm]{images/PR_100.png}
	\caption{Plot showing the number of susceptible, exposed, infected, and recovered individuals over time across 100\% Protection rate. }
	\label{fig:14}
\end{figure}

\begin{table} [H]
	\centering
	\begin{tabular}{|l|l|l|l|l|l|l|l|}
		\hline
		\multicolumn{8}{|c|}{\textbf{100\% Protection Rate}}\\
		\hline
		\multicolumn{2}{|c|}{\textbf{S}} &  \multicolumn{2}{c|}{\textbf{E}}&  \multicolumn{2}{c|}{\textbf{I}}&  \multicolumn{2}{c|}{\textbf{R}}\\
		\hline
		\%& Day & \% & Day & \%  & Day & \% & Day \\
		\hline
		99.95\% & 1 &80.45\%  & 3 &  25  & 1 & 18.38  &15\\
		\hline
	\end{tabular}
	\caption{Maximum percentage of susceptible, exposed, infected, and recovered individuals across a 220-day period, where \textit{basenum\_human} = 2225 individuals, \textit{time\_cycle} = 31680 cycles, \textit{init\_expo} = 25\%, \textit{init\_inf} = 25\%,  \textit{protection\_rate} = 100\%.}
	\label{PR5_Max}
\end{table}


From table \ref{PR5_Max} and Figure \ref{fig:14}, we can observe the following maximum percentage number of each state across the 220 day period. It can be seen that 

\begin{itemize}
	\item The peak percentage of susceptible individuals reached 99.95\%, occurring on Day 75, with approximately 2225 susceptible agents.
	\item The peak percentage of exposed individuals reached 80.45\%, occurring on Day 3, with approximately 1790 exposed agents.
	\item The peak percentage of infected individuals reached 25\%, occurring on Day 1, with approximately 557 individuals affected.
	\item The peak percentage of recovered individuals reached 18.38\%, occurring on Day 15, with approximately 409 recovered agents. 
\end{itemize}
In this scenario, the simulation delves into the dynamics of a full or 100\% protection rate, aiming to rigorously examine the impact of complete protection for each individual on the transmission dynamics of a respiratory infectious disease. The simulation is structured to reflect real-world conditions, starting with 25\% of the population as exposed and another 25\% as infected. The simulation spans a duration of 220 days, which aligns with the length of an academic year.

Upon analyzing Figure \ref{fig:14}, it becomes evident that the number of infected individuals remains constant until day 10 and subsequently begins to decrease gradually, signaling the end of the infectious period. This observation is followed by the simultaneous increase in the number of recovered agents as the number of infected agents declines. The number of infected on Day 1 became the highest number of infected which means that the set initial number of agents infected were the only one who became infected all throughout the simulation. The percentage of the highest number of individual was computed by the given equation:

\[ \%Inf_{max}  = \frac{557}{2225} \times 100\%  = 25.0\%\]
Furthermore, examining the number of exposed agents reveals an interesting trend, with the count of exposed individuals peaking at 1790 individuals at Day 3, yet none of them transition to being infected. This outcome underscores the effectiveness of the full protection rate in preventing infection among exposed individuals.

No multiple waves of infections also occur in this scenario. As the simulation progresses, the number of susceptible agents remains constant starting from Day 71. This stabilization occurs as the count of recovered agents transitions to zero, indicating that all agents have fully recovered and reverted to being susceptible agents once again. This transition highlights the cyclical nature of disease dynamics within a population under the conditions of full protection.

\subsection{ Comparison on the number of Infected Individuals across different varying Protection rates (0\%, 25\%. 50\%. 75\%, 100\%)}


\begin{figure}[H]
	\centering
	
	\subfigure[0\%]{\includegraphics[width = 2.9in, height=1.5in]{images/PR0.png}
		\label{0}}
	\quad
	\subfigure[25\%]{\includegraphics[width = 2.9in,height=1.5in]{images/PR25.png}
		\label{25}}
	
	\subfigure[50\%]{\includegraphics[width = 2.9in, height=1.5in]{images/PR50.png}
		\label{50}}
	\quad
	\subfigure[75\%]{\includegraphics[width = 2.9in, height=1.5in]{images/PR75.png}
		\label{75}}
	
	\subfigure[100\%]{\includegraphics[width = 2.9in, height=1.5in]{images/PR100.png}
		\label{100}}
		
		\caption{The Number of Exposed and Infected Individuals across a 220-day period in varying protection rates}
		\label{Pr5}
\end{figure}
In this section, we will conduct a comparative analysis of the varying protection rates illustrated in Figure \ref{Pr5}, depicting the dynamics of exposed and infected individuals over a 220-day period. By comparing each protection rate against the others, our aim is to gain a comprehensive understanding of their respective impacts on respiratory infectious disease transmission.

Let's begin with a comparison between the plots representing 0\% protection, shown in Figure \ref{0}, and 25\% protection, shown in Figure \ref{25}. Here, we observe a slight difference between scenarios with no protection and those with some level of protection. Without any protection, the number of infected individuals can exceed 2000 mark, with a 100\% chance of exposed individuals transitioning to infection after the incubation period. Conversely, a 25\% protection rate noticeably reduces the number of infected individuals, even though the number of exposed individuals remains similar to the scenario with no protection.

Advancing to the 50\% protection rate, depicted in Figure \ref{50}, we notice a narrowing gap between the numbers of exposed and infected individuals. The count of infected individuals is slightly lower compared to the 25\% protection rate and significantly less than the scenario with 0\% protection rate.

Comparing the 50\% protection rate to the 75\% protection rate, illustrated in Figure \ref{75}, we observe a further decrease in the number of infected individuals. Additionally, the number of exposed individuals surpasses that of infected individuals, indicating that some exposed individuals revert to susceptibility after the incubation period, rather than progressing to infection. With a 75\% protection rate, the number of infected individuals is lower than the number of exposed individuals.

Finally, examining the 100\% protection rate, as depicted in Figure \ref{100}, after the incubation period, no exposed individuals transition to infection. The 100\% protection rate yields the lowest number of infected individuals among all protection rates considered.

\begin{figure}[H]
	\centering
	\includegraphics[width=16cm, height=7cm]{images/PR5.png}
	\caption{Plot showing the number of infected individuals over time across varying protection rates. }
	\label{6}
\end{figure}

The analysis from Figure \ref{6} offers a detailed comparison of the number of infected individuals over time across different protection rates, providing valuable insights into the effectiveness of varying levels of immunity or protection in mitigating disease spread.

Beginning with the scenario of no protection or a 0\% protection rate, we observe the highest number of infections occurring over time. This scenario reflects the vulnerability of the population to the disease without any immunity measures in place. Interestingly, the data indicates that a 25\% protection rate initially yields a similar number of infected individuals as the 0\% protection rate. However, as time progresses, we note that the occurrence of second and third waves of infection is relatively lower compared to the scenario with no protection. No new infection were observed after all the infected individuals recovered.

Moving to a 50\% protection rate, we witness a significant reduction in the number of infected individuals by half compared to the 0\% protection rate scenario. Although a second wave of infection may still occur, its magnitude is notably lower than that observed in the 0\% and 25\% protection rate scenarios.

Increasing the protection rate to 75\% demonstrates even more substantial benefits, leading to a considerable decrease in the number of infected individuals over time. The waves of infection become minimal, with infection rates remaining consistently lower, and only a small proportion (less than 28\% or 623 infected agents) of the population gets infected.

Imposing a 100\% or full protection rate on each individual results in the lowest number of infected individuals over time. In this scenario, only the predefined initially infected individuals show infection, and no new infections are recorded. Among the five levels of protection rates analyzed, a 100\% protection rate emerges as the most effective measure in controlling the number of infected individuals over time.

 

\begin{figure}[H]
	\centering
	\includegraphics[width=16cm, height=8cm]{images/PR_IR.png}
	\caption{Plot showing the number of the highest infection rates across different Protection Rates (0\%, 25\%. 50\%. 75\%, 100\%) }
	\label{fig:15}
\end{figure}

\begin{table}[H]
	\centering
	\begin{tabularx}{\textwidth}{|X|X|X|X|X|}
		\hline
		\multicolumn{5}{|c|}{\textbf{Infection Rate}} \\
		\hline
		0\%& 25\% & 50\% & 75\% & 100\% \\
		\hline
		93.1\% & 80.2\% & 62.4\% & 39.6\% & 18.3\%\\
		\hline
	\end{tabularx}
	\caption{Highest Infection Rates across different Protection Rates (0\%, 25\%. 50\%. 75\%, 100\%) where n= 225, int\_expo = 0.25, int\_inf= 0.25, cycle = 31680 cycles }
	\label{tab:PR_IR}
\end{table}

Table \ref{tab:PR_IR} and figure \ref{fig:15} provides a comparative analysis of infection rates at varying levels of protection rates within a population. Starting with a 0\% protection rate, which signifies no immunity or protection against the infectious disease, the observed infection rate stands at 93.1\%, this means that at one point in time where almost all of the agent were infected indicating a relatively high rate of disease transmission. As the protection rate increases to 25\%, 50\%, and 75\% corresponding to a quarter, half, and three-quarters of the population being immune or protected, the infection rates decrease to  0.802, 62.4\% and 39.6\%, respectively. These reductions highlight the significant impact of increased protection rates on lowering disease transmission. The differences in infection rate reductions between each protection rate are as follows: from 0\% to 25\% protection, there is a decrease of 13\%; from 25\% to 50\% protection, the decrease is 17.8\%; from 50\% to 75\% protection, the decrease is 0.228. Finally, at a 100\% protection rate, representing full immunity or protection for every individual, the infection rate drops significantly to 18.3\%, underscoring the effectiveness of complete protection in minimizing disease spread within the population. 



\section{ Scenario 2: Varying Exposure Rates}
\label{S2}
In this specific scenario, four sub-scenarios were implemented, each featuring a distinct set of variables and conditions related to exposed rates. The simulation was conducted by varying the levels of exposure rates, ranging from 0\% to 75\%. The objective of this simulation was to gather precise insights into how different exposure rates influence the overall outcome of the scenario. By executing multiple sub-scenarios, the simulation effectively considered various factors and variables that could have impacted the results.

\subsection{ Scenario 2.1 SEIR Dynamics in a 0\% Initial Exposure Rate}
\begin{figure}[H]
	\centering
	\includegraphics[width=16cm, height=7cm]{images/ER_0.png}
	\caption{Plot showing the number of susceptible, exposed, infected, and recovered individuals over time across 0\% Exposed rate. }
	\label{fig:13}
\end{figure}

\begin{table} [H]
	\centering
	\begin{tabular}{|l|l|l|l|l|l|l|l|}
		\hline
		\multicolumn{8}{|c|}{\textbf{0\% Exposure Rate}}\\
		\hline
		\multicolumn{2}{|c|}{\textbf{S}} &  \multicolumn{2}{c|}{\textbf{E}}&  \multicolumn{2}{c|}{\textbf{I}}&  \multicolumn{2}{c|}{\textbf{R}}\\
		\hline
		\%& Day & \% & Day & \%  & Day & \% & Day \\
		\hline
		75.01& 1 &74.74& 3 &  57.57& 11& 80.53&45\\
		\hline
	\end{tabular}
	\caption{Maximum percentage of susceptible, exposed, infected, and recovered individuals across a 220-day period, where \textit{basenum\_human} = 2225 individuals, \textit{time\_cycle} = 31680 cycles, \textit{init\_expo} = 0\%, \textit{init\_inf} = 25\%,  \textit{protection\_rate} = 25\%.}
	\label{ER1_Max}
\end{table}


From table \ref{ER1_Max} and Figure \ref{fig:18}, we can observe the following maximum percentage number of each state across the 220 day period. It can be seen that 

\begin{itemize}
	\item The peak percentage of susceptible individuals reached 75.01\%, occurring on Day 1, with approximately 1669 susceptible agents.
	\item The peak percentage of exposed individuals reached 74.74\%, occurring on Day 3, with approximately 1790 exposed agents
	\item The peak percentage of infected individuals reached 57.57\%, occurring on Day 19, with approximately 1281 individuals affected.
	\item The peak percentage of recovered individuals reached 80.53\%, occurring on Day 43, with approximately 1792 recovered agents. 
	
\end{itemize}
The simulation commences with a population of 2225 individuals, among whom 25\% are initially assumed to be infected, while 0\% are assumed to be exposed. An observable increase is noted by day 2, indicating the transition of some exposed agents to an infected state. The peak of infected individuals occurred on Day 19, totaling 1281 infected individuals, resulting in an highest infection rate of 57.57\%, calculated as follows:
\[ \%Inf_{max}= \frac{1276}{2225}\times 100\%  = 57.57\% \]

Subsequently, infected individuals begin to recover between Day 10 and Day 14, signifying the end of their infectious period. As the number of recovered agents increases, the number of infected individuals declines. Recovered individuals started to rise at day 10, signaling the end of the 10 to 14 days of infectious period for infected individuals. The highest peak number of  recovered individuals occurred on Day 43, just 34 days after the peak of infection. The number of recovered individuals started to decline after 20 days of recovery, following the 20 to 60 days of immunity. Interestingly, after the end of immunity, at around day 71, there is a resurgence in the number of infected individuals, indicating the onset of a second wave of infection.

Following the infectious period of the second wave, agents again start to recover, leading to a decline in the number of infected individuals. Although a slight increase in the number of infected individuals is observed later on, it remains relatively lower than during the first and second waves of infection. This observation suggests a tapering off or containment of subsequent waves compared to the initial peaks.

\subsection{ Scenario 2.2 SEIR Dynamics in a 25\% Initial Exposure Rate}
\begin{figure}[H]
	\centering
	\includegraphics[width=16cm, height=7cm]{images/ER_25.png}
	\caption{Plot showing the number of susceptible, exposed, infected, and recovered individuals over time across initial 25\% Exposure rate. }
	\label{fig:18}
\end{figure}
\begin{table} [H]
	\centering
	\begin{tabular}{|l|l|l|l|l|l|l|l|}
		\hline
		\multicolumn{8}{|c|}{\textbf{25\% Exposure Rate}}\\
		\hline
		\multicolumn{2}{|c|}{\textbf{S}} &  \multicolumn{2}{c|}{\textbf{E}}&  \multicolumn{2}{c|}{\textbf{I}}&  \multicolumn{2}{c|}{\textbf{R}}\\
		\hline
		\%& Day & \% & Day & \%  & Day & \% & Day \\
		\hline
		56.18& 1 &79.86& 3 &  61.20& 19& 84.22&43\\
		\hline
	\end{tabular}
	\caption{Maximum percentage of susceptible, exposed, infected, and recovered individuals across a 220-day period, where \textit{basenum\_human} = 2225 individuals, \textit{time\_cycle} = 31680 cycles, \textit{init\_expo} = 25\%, \textit{init\_inf} = 25\%,  \textit{protection\_rate} = 25\%.}
	\label{ER2_Max}
\end{table}
From table \ref{ER2_Max} and Figure \ref{fig:18}, we can observe the following maximum percentage number of each state across the 220 day period. It can be seen that 

\begin{itemize}
	
	\item The peak percentage of susceptible individuals reached 56.18\%, occurring on Day 1, with approximately 1251 susceptible agents.
	\item The peak percentage of exposed individuals reached 79.86\%, occurring on Day 3, with approximately 1777 exposed agents
	\item The peak percentage of infected individuals reached 61.20\%, occurring on Day 19, with approximately 1362 individuals affected.
	\item The peak percentage of recovered individuals reached 84.22\%, occurring on Day 43, with approximately 1874 recovered agents. 
\end{itemize}

The simulation commences with a population of 2225 individuals, among whom 25\% are initially assumed to be infected, while 25\% are assumed to be exposed. 

The scenario begins with a population of 2225 individuals, with 25\% of them initially assumed to be infected and 25\% assumed to be exposed to the infectious agent. This exposure rate signifies a significant portion of the population at risk of contracting the disease. By day 2 of the simulation, a noticeable increase is observed, indicating the transition of some exposed individuals to an infected state. This transition period highlights the critical phase of disease transmission within the exposed population subset.

From figure \ref{fig:18} The peak of infected individuals is recorded on Day 19, with a total of 105 infected individuals out of the population of 2225. This peak corresponds to an infection rate of 61.20\%, calculated as the ratio of infected individuals to the total population:
\[ \%Inf_{max} = \frac{1362}{2225} \times 100\%  = 61.20\% \]

It is crucial to note that this infection rate reflects the combined impact of both initial infections and subsequent exposures leading to infections. 

Following the peak, a significant phase of recovery occurs between Day 10 and Day 14, marking the end of the infectious period for many individuals. The increasing number of recovered agents during this period contributes to a decline in the active infections. However, around day 73, a resurgence in the number of infected individuals is observed, indicating the onset of a second wave of infections within the exposed population subset.

As with the initial peak, the infectious period of the second wave gradually subsides, leading to another phase of recovery among the agents. Although there is a slight increase in infections beyond this point, the magnitude remains relatively lower than during the first and second waves.

\subsection{Scenario 2.3 SEIR Dynamics in a 50\% Initial Exposure Rate}
\begin{figure}[H]
	\centering
	\includegraphics[width=16cm, height=7cm]{images/ER_50.png}
	\caption{Plot showing the number of susceptible, exposed, infected, and recovered individuals over time across 50\% Exposed rate. }
	\label{fig:19}
\end{figure}
\begin{table} [H]
	\centering
	\begin{tabular}{|l|l|l|l|l|l|l|l|}
		\hline
		\multicolumn{8}{|c|}{\textbf{50\% Exposure Rate}}\\
		\hline
		\multicolumn{2}{|c|}{\textbf{S}} &  \multicolumn{2}{c|}{\textbf{E}}&  \multicolumn{2}{c|}{\textbf{I}}&  \multicolumn{2}{c|}{\textbf{R}}\\
		\hline
		\%& Day & \% & Day & \%  & Day & \% & Day \\
		\hline
		37.49& 1 &84.18& 2&  74.2& 19& 88.81&44\\
		\hline
	\end{tabular}
	\caption{Maximum percentage of susceptible, exposed, infected, and recovered individuals across a 220-day period, where \textit{basenum\_human} = 2225 individuals, \textit{time\_cycle} = 31680 cycles, \textit{init\_expo} = 50\%, \textit{init\_inf} = 25\%,  \textit{protection\_rate} = 25\%.}
	\label{ER3_Max}
\end{table}
From table \ref{ER3_Max} and Figure \ref{fig:19}, we can observe the following maximum percentage number of each state across the 220 day period. It can be seen that 

\begin{itemize}
	
	\item The peak percentage of susceptible individuals reached 37.49\%, occurring on Day 1, with approximately 835 susceptible agents.
	\item The peak percentage of exposed individuals reached 86.97\%, occurring on Day 3, with approximately 1936 exposed agents
	\item The peak percentage of infected individuals reached 74.20\%, occurring on Day 19, with approximately 1651 individuals affected.
	\item The peak percentage of recovered individuals reached 88.40\%, occurring on Day 44, with approximately 1967 recovered agents. 
\end{itemize}

In the simulation's initial phase, a population of 2225 individuals is considered, with precisely half, or 50\%, assumed to be exposed to the infectious agent and an equivalent percentage initially infected. This exposure rate signifies a substantial risk segment within the population vulnerable to contracting the disease. By the second day of the simulation, a noticeable surge is observed, indicating the transition of a significant number of exposed individuals into the infected state. This rapid transition period underscores the criticality of disease transmission dynamics within the exposed population subset.

Day 19  as seen in figure \ref{fig:19} marks a significant milestone in the simulation, recording the peak of infected individuals at 1651 out of the total population of 2225. This peak corresponds to an infection rate of 74.20\%, calculated as the ratio of infected individuals to the entire population shown by the formula:
\[ \%Inf_{max} = \frac{1651}{2225} \times 100\%  = 74.20\%\]

This infection rate encapsulates the combined impact of both initial infections and subsequent exposures leading to new infections within the exposed group.

After reaching this peak, a significant recovery period occurs between Day 10 and Day 14, marking the end of the infectious phase for many individuals. The highest recovery was achieved on Day 44. The increasing number of individuals recovering during this period contributes substantially to reducing active infections within the exposed group. However, around Day 71, there is a resurgence in infections, indicating the beginning of a second wave among this exposed subset.

Similar to the initial peak, the infectious phase of this second wave gradually diminishes, leading to another recovery phase among the affected individuals. An increase in the number of infected individual was once again observed on day 151 in infections thereafter, indicating a third wave of infection. As the end of the year approaches, the number of recovered agents once again starts to rise. 

\subsection{ Scenario 2.4 SEIR Dynamics in a 75\% Initial Exposure Rate}
\begin{figure}[H]
	\centering
	\includegraphics[width=16cm, height=7cm]{images/ER_75.png}
	\caption{Plot showing the number of susceptible, exposed, infected, and recovered individuals over time across 75\% Exposure rate. }
	\label{fig:20} 
\end{figure}
\begin{table} [H]
	\centering
	\begin{tabular}{|l|l|l|l|l|l|l|l|}
		\hline
		\multicolumn{8}{|c|}{\textbf{75\% Exposure Rate}}\\
		\hline
		\multicolumn{2}{|c|}{\textbf{S}} &  \multicolumn{2}{|c|}{\textbf{E}}&  \multicolumn{2}{|c|}{\textbf{I}}&  \multicolumn{2}{|c|}{\textbf{R}}\\
		\hline
		\%& Day & \% & Day & \%  & Day & \% & Day \\
		\hline
		18.92& 1 &86.96& 3&  80.28& 19& 93.48&42\\
		\hline
	\end{tabular}
	\caption{Maximum percentage of susceptible, exposed, infected, and recovered individuals across a 220-day period, where \textit{basenum\_human} = 2225 individuals, \textit{time\_cycle} = 31680 cycles, \textit{init\_expo} = 75\%, \textit{init\_inf} = 25\%,  \textit{protection\_rate} = 25\%.}
	\label{ER4_Max}
\end{table}
From table \ref{ER4_Max} and Figure \ref{fig:20}, we can observe the following maximum percentage number of each state across the 220 day period. It can be seen that 

\begin{itemize}
	
	\item The peak percentage of susceptible individuals reached 18.92\%, occurring on Day 1, with approximately 421 susceptible agents.
	\item The peak percentage of exposed individuals reached 86.96\%, occurring on Day 3, with approximately 1935 exposed agents
	\item The peak percentage of infected individuals reached 80.28\%, occurring on Day 19, with approximately 1787 individuals affected.
	\item The peak percentage of recovered individuals reached 93.48\%, occurring on Day 42, with approximately 2080 recovered agents. 
\end{itemize}

For the scenario with a 75\% Exposure Rate, the simulation begins with a population of 2225 individuals, with 25\% initially infected and 75\% exposed to the infectious agent. This setup reflects a significant portion of the population susceptible to contracting the disease, emphasizing the potential for widespread transmission within the exposed group.

As the simulation progresses, a notable increase in exposed individuals is observed over time, as depicted in Figure \ref{fig:20}. This rise in exposure levels correlates with a subsequent increase in the number of infected individuals, noticeable from Day 2 onwards. The trend suggests a direct relationship between exposure and infection rates within the population.

The peak of infections is reached on Day 19, with a recorded count of 1787 infected individuals out of the total population of 2225. This peak coincides with the highest infection rate observed throughout the 220-day simulation period, calculated at 0.68 using the formula:

\[ \%Inf_{max} = \frac{1787}{2225}\times 100\%  = 80.28\%\]

Additionally, the simulation reveals the occurrence of multiple waves of infections, notably starting on Day 81 and Day 151. These subsequent waves, identified as the 2nd and 3rd waves of infection, though lower in magnitude compared to the initial wave, still lead to approximately 100 infected individuals. This observation underscores the persistence of infectious outbreaks even after initial peaks subside, highlighting the need for sustained vigilance and responsive healthcare strategies to manage recurrent infection waves effectively.


\subsection{ Comparison on the number of Infected Individuals across different varying Initial Exposure rates (0\%, 25\%. 50\%. 75\%)}
\begin{figure}[H]
	\centering
	\subfigure[0\%]{\includegraphics[width = 2.9in, height=1.5in]{images/ER0.png}
		\label{E0}}
	\quad
	\subfigure[25\%]{\includegraphics[width = 2.9in,height=1.5in]{images/ER25.png}
		\label{E25}}
	
	\subfigure[50\%]{\includegraphics[width = 2.9in, height=1.5in]{images/ER50.png}
		\label{E50}}
	\quad
	\subfigure[75\%]{\includegraphics[width = 2.9in, height=1.5in]{images/ER75.png}
		\label{E75}}
	
	\caption{Number of Infected Individuals Over a 220-Day Period at varying Initial Exposure Rates of 0\%, 25\%, 50\%, and 75\%}
	\label{E4}
\end{figure}

For the four sub-scenarios, the simulation parameters used, included a population size of 2225, a protection rate of 0.50, an initial infection rate of 0.25, and a simulated duration of 31680 cycles. 

Figure \ref{E4} presents separate graphs depicting the number of infected individuals at different initial exposure rates. Let's compare the scenario of 0\% Initial Exposure Rate with 25\% Initial Exposure Rate. Starting with no exposed individuals at the beginning of the simulation yields the fewest infected individuals among all initial exposure rates. Conversely, with a 25\% initial exposure rate, where a quarter of the population is exposed at the simulation's outset, there's a slight increase in the number of infected individuals compared to Figure \ref{E0}.

This trend persists when comparing the graph of the 25\% initial exposure rate with the 50\% initial exposure rate shown in Figure \ref{E50}. Here, we observe a rise in infected individuals, surpassing the 1500 mark in the graph.

Moving on to the scenario of 75\% Initial Exposure Rate, as seen in Figure \ref{E75}, it exhibits the highest infection rate among all initial exposure rates. This indicates that having 75\% of the population exposed also leads to a higher number of infections.

\begin{figure}[H]
	\centering
	\includegraphics[width=16cm, height=8cm]{images/ER4.png}
	\caption{Plot showing the number of infected individuals across different Initial Exposure Rates (0\%, 25\%. 50\%. 75\%) }
	\label{fig:22}
\end{figure}

The data from Figure \ref{fig:22} provides a comparative analysis of the number of infected individuals across different exposure rates presented in one graph, offering valuable insights into disease transmission dynamics within the simulated population.

Starting with the 75\% exposure rate, which represents a significant proportion of the population initially exposed, we observe the highest number of infected individuals, peaking at 1787 individuals. This finding underscores the direct correlation between exposure levels and infection rates, highlighting the heightened risk of disease spread when a larger portion of the population is susceptible to the infectious agent.

Moving to the 50\% exposure rate scenario, we note a substantial number of infected individuals, reaching up to 1651 individuals with 136 individuals lower than the 75\% exposure rate. Additionally, the observation of two distinct waves of infection in this scenario emphasizes the dynamic nature of disease transmission, with potential fluctuations in infection rates over time.

The 25\% exposure rate scenario follows, showing a peak of 1362 infected individuals. While 425 and 289 lower than the previous exposure rates, this number still signifies a significant risk of disease transmission within the exposed subset of the population. Notably, multiple waves of infection are observed here as well, indicating recurrent periods of heightened transmission within this exposure group.

Finally, at the 0\% exposure rate, where there is no initial exposure, we observe the least number of infected individuals, totaling 1281 individuals. This observation aligns with the expected trend that lower exposure levels correspond to reduced infection rates, highlighting the importance of preventive measures and containment strategies in limiting disease spread. This scenario gives the lowest number of infected individuals which means that having a lower exposure rate will result to lower number of infected individuals.

Overall, the data underscores the critical role of exposure rates in influencing infection dynamics, with higher exposure rates correlating with increased infection rates and the potential for multiple waves of infection.
\begin{figure}[H]
	\centering
	\includegraphics[width=16cm, height=8cm]{images/ER_IR.png}
	\caption{Plot showing the number of the highest infection rates across different Exposure Rates (0\%, 25\%. 50\%. 75\%) }
	\label{fig:21}
\end{figure}

\begin{table}[H]
	\centering
	\begin{tabularx}{\textwidth}{|X|X|X|X|X|}
		\hline
		\multicolumn{4}{|c|}{\textbf{Infection Rate}} \\
		\hline
		0\%& 25\% & 50\% & 75\% \\
		\hline
		57.57\%& 61.20\% & 74.20\%  & 80.20\%\\
		\hline
	\end{tabularx}
	\caption{Highest Percentage of Infection across different Exposure Rates (0\%, 25\%. 50\%. 75\%,) where N= 2225, protection\_rate = 0.50, int\_inf= 0.25, cycle = 31680 cycles }
	\label{tab:ER_IR}
\end{table}     

The data table present in Table \ref{tab:ER_IR} provides a comprehensive view of the highest infection rates observed across varying exposure rates within a simulated population, shedding light on the intricate dynamics of disease transmission. Examining the four exposure rates - 0\%, 25\%, 50\%, and 75\%, we gain valuable insights into how different levels of initial exposure impact the spread of infectious diseases.

At a 0\% exposure rate, where no initial exposure occurs, the  highest percentage of infection is measured at 57.57\%. This indicates that almost half of the population that time was infected. It also indicates a baseline level of disease transmission within the population, likely influenced by factors such as interactions among individuals and environmental conditions.

As we move to a 25\% exposure rate, reflecting a quarter of the population initially exposed, we observe a slight increase in the highest percentage of infection to 61.20\%. This uptick signifies a higher level of disease spread compared to the no-exposure scenario, highlighting the vulnerability of exposed individuals to infection and potential transmission to others.

The most notable escalation in infection rates is observed at a 50\% exposure rate, where half of the population is initially exposed. Here, the highest percentage of infection noted spikes to 74.20\%, showcasing a significant impact of increased exposure levels on disease transmission dynamics.

Surprisingly, after having three-quarters of the population being initially exposed at a 75\% exposure rate, the highest percentage of infection notably increases to 80.2\%. This substantial rise emphasizes the heightened risk of disease spread in environments with extensive exposure, where a larger proportion of individuals are susceptible to infection.



\section{Scenario 3: Implementation of Lockdown and Quarantine Area}
\label{S3}
In this scenario, we conducted four sub-scenarios, each with distinct conditions, to investigate the effects of implementing Non-Pharmaceutical Interventions (NPIs) on disease transmission dynamics. The simulation varied by implementing Lockdown and Quarantine Areas, aiming to understand how different combinations of these interventions impact the spread of the disease. The scenarios included: No Lockdown and No Quarantine Area, Lockdown Implementation without Quarantine Area, No Lockdown but with Quarantine Area Implementation, and finally, Implementation of both Lockdown and Quarantine Areas. The objective of this simulation was to provide valuable insights into the effectiveness of NPIs in controlling the spread of infectious diseases and inform public health strategies for managing future outbreaks 

\subsection{ Scenario 3.1 No Lockdown and No Quarantine Area}
\label{3.1b}

In this particular scenario, No Lockdown and No Quarantine Area were implemented. It involved a population of 2225 individuals, with 10\% of the population assumed to be infected and an equal proportion exposed. Additionally, there was a 50\% protection rate in place. In this setup, individuals could freely roam the hallways and environment during their breaks, which occurred at 10-10:30 AM, 12:00 NN Lunch, and the end of class at 5 PM.

\begin{figure}[H]
	\centering
	\subfigure[Snapshot of the simulation during break time]
	{\includegraphics[width=2.4in]{images/3.1a.png}
		%\caption{Snapshot of the simulation during break time}}
	\label{break}}
\quad
\subfigure[Snapshot of the simulation during lunch time]
{\includegraphics[width=2.4in]{images/3.1b.png}
	%\caption{Snapshot of the simulation during lunch time}
	\label{lunchtime}}

\caption{Snapshot of the simulation when No Lockdown and No Quarantine Area was implemented.}
\label{fig:figures}
\end{figure}

Figure \ref{break} depicts the simulation scenario during the designated break time, while Figure \ref{lunchtime} illustrates the situation during lunchtime. These visual representations highlight a crucial aspect of the simulation: the unrestricted movement of agents during these periods. As observed, agents are allowed to freely roam around, which significantly heightens the risk of transmitting the infectious disease. This unrestricted mobility increases the likelihood of close contact between individuals, facilitating the spread of the disease within the simulated environment.
\begin{figure}[H]
\centering
\includegraphics[width=16cm, height=7cm]{images/NQNLD.png}
\caption{Plot showing the number of susceptible, exposed, infected, and recovered individuals in No Lockdown and No Quarantine Area scenario. }
\label{LD1} 
\end{figure}
\begin{table} [H]
\centering
\begin{tabular}{|l|l|l|l|l|l|l|l|}
	\hline
	\multicolumn{8}{|c|}{\textbf{No Lockdown and No Quarantine Area}}\\
	\hline
	\multicolumn{2}{|c|}{\textbf{S}} &  \multicolumn{2}{|c|}{\textbf{E}}&  \multicolumn{2}{|c|}{\textbf{I}}&  \multicolumn{2}{|c|}{\textbf{R}}\\
	\hline
	\%& Day & \% & Day & \%  & Day & \% & Day \\
	\hline
	81.26& 1 &57.61& 11&  55.01& 24& 75.95&52\\
	\hline
\end{tabular}
\caption{Maximum percentage of susceptible, exposed, infected, and recovered individuals across a 220-day period, where \textit{basenum\_human} = 2225 individuals, \textit{time\_cycle} = 31680 cycles, \textit{init\_expo} = 10\%, \textit{init\_inf} = 10\%,  \textit{protection\_rate} = 50\%.}
\label{LD1_Max}
\end{table}
From table \ref{LD1_Max} and Figure \ref{LD1}, we can observe the following maximum percentage number of each state across the 220-day period. It can be seen that 

\begin{itemize}

\item The peak percentage of susceptible individuals reached 81.26\%, occurring on Day 1, with approximately 1809 susceptible agents.
\item The peak percentage of exposed individuals reached 57.61\%, occurring on Day 11, with approximately 1282 exposed agents
\item The peak percentage of infected individuals reached 55.01\%, occurring on Day 24, with approximately 1224 individuals affected.
\item The peak percentage of recovered individuals reached 75.95\%, occurring on Day 52, with approximately 1690 recovered agents. 
\end{itemize}

The simulation commences on Day 1, with only 10\% of the population initially infected. Over time, the number of exposed individuals gradually increases, reaching a significant rise by Day 2, marking the end of the incubation period. Concurrently, the number of infected individuals begins to rise, reaching its peak infection percentage across the 220-day period on Day 24, with 1224 infected individuals. Following the infectious period, which typically lasts 10 to 14 days, the number of recovered individuals starts to climb, beginning on Day 11. The highest percentage of the population that recovers is recorded on Day 52, with 1690 individuals having recovered after being infected by the disease. This calculation is determined by the formula:

\[ \%Inf_{max} = \frac{1224}{2225} \times 100\% = 55.01\%\]

Notably, after the immunity period of the recovered agents ends, another wave of infection is observed on Day 71. Although lower than the first wave of infection, it still surpasses the 500 individuals mark in the graph. Once again, following the infectious period, the number of recovered individuals starts to rise, indicating the end of the infectious period for the agents in the second wave. These multiple waves of infection maybe caused by the interaction of the students during the time where they can roam around the campus. This sped up the transmission of the virus increasing the chance of passing and being infected by the virus. 

\subsection{ Scenario 3.2 Implementation of Lockdown and No Quarantine Area}
\label{3.2}
Depicted in  Figure\ref{class} is the situation during lunch, it can be seen that the students are inside the class. Unlike Scenario 3.1, the students are not allowed to go out the classroom and roam around the hallways. They are only allowed to go out during the end of the class which is shown in Figure \ref{end} where the students can be seen moving through the road network. This represents the students going home. 

\begin{figure}[H]
\centering
\subfigure[Snapshot of the simulation during lunch time when lockdown is implemented]{\includegraphics[width = 2.8in, height=2.5in]{images/3.2a.png} 
\label{class}}
\quad
\subfigure[Snapshot of the simulation during end of class]{\includegraphics[width = 2.8in, height=2.5in]{images/3.2b.png} 
	\label{end}}
	\caption{Snapshot of the simulation during lunch time and end of class when lockdown is implemented}
	\end{figure}
	
	\begin{figure}[H]
\centering
\includegraphics[width=16cm, height=7cm]{images/LDNQ.png}
\caption{Plot showing the number of susceptible, exposed, infected, and recovered individuals in implementing a Lockdown and No Quarantine Area scenario. }
\label{LD2} 
\end{figure}
\begin{table} [H]
\centering
\begin{tabular}{|l|l|l|l|l|l|l|l|}
	\hline
	\multicolumn{8}{|c|}{\textbf{Lockdown and No Quarantine Area}}\\
	\hline
	\multicolumn{2}{|c|}{\textbf{S}} &  \multicolumn{2}{|c|}{\textbf{E}}&  \multicolumn{2}{|c|}{\textbf{I}}&  \multicolumn{2}{|c|}{\textbf{R}}\\
	\hline
	\%& Day & \% & Day & \%  & Day & \% & Day \\
	\hline
	80.98& 1 &40.04& 12&  28.53& 24& 48.63&56\\
	\hline
\end{tabular}
\caption{Maximum percentage of susceptible, exposed, infected, and recovered individuals across a 220-day period with Lockdown and No Quarantine Area, where \textit{basenum\_human} = 2225 individuals, \textit{time\_cycle} = 31680 cycles, \textit{init\_expo} = 10\%, \textit{init\_inf} = 10\%,  \textit{protection\_rate} = 50\%.}
\label{LD2_Max}
\end{table}
From table \ref{LD2_Max} and Figure \ref{LD2}, we can observe the following maximum percentage number of each state across the 220-day period. It can be seen that 

\begin{itemize}

\item The peak percentage of susceptible individuals reached 80.98\%, occurring on Day 1, with approximately 1802 susceptible agents.
\item The peak percentage of exposed individuals reached 40.4\%, occurring on Day 12, with approximately 891 exposed agents
\item The peak percentage of infected individuals reached 28.53\%, occurring on Day 24, with approximately 635 individuals affected.
\item The peak percentage of recovered individuals reached 48.63\%, occurring on Day 56, with approximately 1083 recovered agents. 
\end{itemize}

For this certain scenario, The simulation begins on Day 1, with an initial infection rate of only 10\% of the population. As time progresses, the number of exposed individuals steadily grows, experiencing a notable surge by Day 2, signifying the conclusion of the incubation period of 2-14 days. Meanwhile, the count of infected individuals starts to escalate, culminating in its highest percentage of infection over the 220-day period on Day 24, with 635 individuals affected showed by the equation:

\[ \%Inf_{max} = \frac{635}{2225} \times 100\%  = 40.04\%\]

The analysis reveals a noteworthy trend in the simulation: the emergence of a substantial increase in recovered agents after Day 10, signifying the conclusion of the 10 to 14-day infectious period. Notably, this trend reaches its pinnacle on Day 56, with 40.04\% of the population at that time having recovered from the infection.

Furthermore, the absence of additional waves in the graph can be also observed. The data indicates that the number of infected individuals remains relatively stable and fails to surpass the 1000-mark threshold throughout the observed period. This observation may be attributed to the implementation of a lockdown measure, which appears to have played a significant role in containing the spread of the virus.

The effectiveness of the lockdown measure lies in its ability to restrict the movement of individuals and limit social interactions, thereby reducing the likelihood of virus transmission. By confining individuals to limited spaces and minimizing opportunities for contact with others, the lockdown measure effectively curtails the spread of the virus within the simulated environment.

\subsection{ Scenario 3.3 Implementation of having a Quarantine Area and No Lockdown}
\label{3.3}
\begin{figure}[H]
\centering
\includegraphics[width=16cm, height=7cm]{images/QJNLD.png}
\caption{Plot showing the number of susceptible, exposed, infected, and recovered individuals in implementing a Quarantine Area and No Lockdown scenario. }
\label{LD3} 
\end{figure}
\begin{table} [H]
\centering
\begin{tabular}{|l|l|l|l|l|l|l|l|}
	\hline
	\multicolumn{8}{|c|}{\textbf{Lockdown and No Quarantine Area}}\\
	\hline
	\multicolumn{2}{|c|}{\textbf{S}} &  \multicolumn{2}{|c|}{\textbf{E}}&  \multicolumn{2}{|c|}{\textbf{I}}&  \multicolumn{2}{|c|}{\textbf{R}}\\
	\hline
	\%& Day & \% & Day & \%  & Day & \% & Day \\
	\hline
	81.25& 1 &36.54& 245&  30.16& 35& 59.73&61\\
	\hline
\end{tabular}
\caption{Maximum percentage of susceptible, exposed, infected, and recovered individuals across a 220-day period with Lockdown and No Quarantine Area, where \textit{basenum\_human} = 2225 individuals, \textit{time\_cycle} = 31680 cycles, \textit{init\_expo} = 10\%, \textit{init\_inf} = 10\%,  \textit{protection\_rate} = 50\%.}
\label{LD3_Max}
\end{table}
From table \ref{LD3_Max} and Figure \ref{LD3}, we can observe the following maximum percentage number of each state across the 220-day period. It can be seen that 

\begin{itemize}

\item The peak percentage of susceptible individuals reached 81.25\%, occurring on Day 1, with approximately 1808 susceptible agents.
\item The peak percentage of exposed individuals reached 36.54\%, occurring on Day 24, with approximately 814 exposed agents
\item The peak percentage of infected individuals reached 30.16\%, occurring on Day 36, with approximately 672 individuals affected.
\item The peak percentage of recovered individuals reached 59.73\%, occurring on Day 62, with approximately 1329 recovered agents. 
\end{itemize}

From Figure \ref{LD3}, the simulation begins on Day 1, with 10\% of the population, or 225 individuals, already infected. This initial number rapidly escalates, reaching its peak infection rate on Day 36, with 672 individuals affected. This peak infection rate is also the highest percentage recorded throughout the 220-day period, calculated as 30.16% using the formula:

\[ \%Inf_{max} = \frac{672}{2225} \times 100\% = 30.16\% \].

Following the conclusion of the infectious period on Day 10, the number of recovered agents starts to rise steadily, reaching its peak on Day 62 with 1329 recovered individuals. However, despite the peak in recoveries, the number of infected individuals remains relatively stable from Day 121 to Day 220.
\begin{figure}[H]
\centering
\subfigure[Quarantine Are for infected individuals.]{\includegraphics[width = 2.8in, height=2.5in]{images/qua.png} }
\label{quarantine}
\quad
\subfigure[Snapshot of susceptible agents' interaction with other infected individuals whn lockdown is absent]{\includegraphics[width = 2.8in, height=2.5in]{images/3.3b.png} }
\label{interaction}
\caption{Snapshot of the simulation during lunch time and end of class when lockdown is implemented}
\end{figure}

Additionally, the graph indicates that the number of infected individuals remains constant and barely exceeds the 500-mark threshold. This phenomenon may be attributed to the implementation of a quarantine area for infected individuals which can be seen , which effectively isolates them from the rest of the population. The sustained infection rate may also be influenced by the absence of a lockdown measure, allowing infected agents to continue interacting with susceptible individuals.

While the number of infections is lower compared to scenarios without a quarantine area, the presence of such a containment measure proves beneficial in limiting the transmission of the infectious disease within the simulated environment.

\subsection{ Scenario 3.4 Implementation of Lockdown and Quarantine Area}
\label{3.4}
\begin{figure}[H]
\centering
\includegraphics[width=16cm, height=7cm]{images/QLD.png}
\caption{Plot showing the number of susceptible, exposed, infected, and recovered individuals in implementing a Lockdown and Quarantine Area scenario. }
\label{LD4} 
\end{figure}
\begin{table} [H]
\centering
\begin{tabular}{|l|l|l|l|l|l|l|l|}
	\hline
	\multicolumn{8}{|c|}{\textbf{Lockdown and No Quarantine Area}}\\
	\hline
	\multicolumn{2}{|c|}{\textbf{S}} &  \multicolumn{2}{|c|}{\textbf{E}}&  \multicolumn{2}{|c|}{\textbf{I}}&  \multicolumn{2}{|c|}{\textbf{R}}\\
	\hline
	\%& Day & \% & Day & \%  & Day & \% & Day \\
	\hline
	80.62& 1 &33.17& 14&  28.67& 30& 51.60&58\\
	\hline
\end{tabular}
\caption{Maximum percentage of susceptible, exposed, infected, and recovered individuals across a 220-day period with Lockdown and No Quarantine Area, where \textit{basenum\_human} = 2225 individuals, \textit{time\_cycle} = 31680 cycles, \textit{init\_expo} = 10\%, \textit{init\_inf} = 10\%,  \textit{protection\_rate} = 50\%.}
\label{LD4_Max}
\end{table}
From table \ref{LD4_Max} and Figure \ref{LD4}, we can observe the following maximum percentage number of each state across the 220-day period. It can be seen that 

\begin{itemize}

\item The peak percentage of susceptible individuals reached 80.62\%, occurring on Day 1, with approximately 1794 susceptible agents.
\item The peak percentage of exposed individuals reached 33.17\%, occurring on Day 14, with approximately 738 exposed agents
\item The peak percentage of infected individuals reached 28.67\%, occurring on Day 30, with approximately 638 individuals affected.
\item The peak percentage of recovered individuals reached 51.60\%, occurring on Day 58, with approximately 1148 recovered agents. 
\end{itemize}

In this scenario, the simulation begins with 10\% of the population already infected, while another 10\% are exposed individuals. As the simulation progresses, the number of exposed individuals steadily rises, reflecting the 2 to 10-day incubation period. Subsequently, as the incubation period concludes, the number of infected individuals begins to increase, culminating in a peak infection rate on Day 30, where 28.67\% of the population, or 638 individuals, are considered infected. This infection rate is calculated using the equation:

\[ \%Inf_{max} = \frac{638}{2225} \times 100\%  = 28.67\%\]

Following this peak, the number of infected individuals gradually declines as their infectious period ends. Concurrently, the number of recovered agents starts to rise, eventually reaching a peak where 51.60\% of the population are recovered.

The observed peak in infections followed by a decline suggests that the infectious period plays a significant role in shaping the dynamics of disease transmission. Additionally, the subsequent rise in recovered individuals underscores the importance of recovery in mitigating the spread of the disease within the population. However, the sustained number of exposed and infected individuals until the end of the simulation indicates ongoing transmission and highlights the need for continued monitoring and intervention to control the outbreak effectively.
\subsection{ Comparison on the number of Infected Individuals across different varying Implementation of Lockdown and Quarantine Area}

\begin{figure}[H]
	\centering
	\subfigure[No Quarantine and No Lockdown]{\includegraphics[width = 2.9in, height=1.5in]{images/NQNLD3.png}
		\label{a}}
	\quad
	\subfigure[Lockdown and No Quarantine]{\includegraphics[width = 2.9in,height=1.5in]{images/LNQ2.png}
		\label{b}}
	\subfigure[Quarantine and No Lockdown]{\includegraphics[width = 2.9in, height=1.5in]{images/QNLD1.png}
		\label{c}}
	\quad
	\subfigure[Quarantine and Lockdown]{\includegraphics[width = 2.9in, height=1.5in]{images/QLD0.png}
		\label{d}}
	
	\caption{Number of Infected Individuals across different varying Implementation of Lockdown and Quarantine Area}
	\label{QLD}
\end{figure}


In each of the four sub-scenarios, the simulation utilized specific parameters: a population size of 2225, a protection rate set at 0.50, an initial infection rate of 0.10, and a simulated duration lasting 31680 cycles.

Figure \ref{QLD} illustrates the number of infected individuals under varying scenarios of implementing lockdown and establishing quarantine areas. We observe that the scenario with No Lockdown and No Quarantine Area, represented by Figure \ref{a}, yields the highest number of infections compared to the other three scenarios.

Upon examining the trends in the graphs represented by Figures \ref{b}, \ref{c}, and \ref{d}, it becomes evident that the number of infected individuals is notably lower and follows a similar pattern compared to the scenario with No Quarantine and No Lockdown.

Among the four graphs, the scenario resulting in the fewest infections is the one where both Lockdown and Quarantine Areas are implemented. This suggests that combining these measures leads to the most effective containment of the spread of infection.



\begin{figure}[H]
	\centering
	\includegraphics[width=16cm, height=8cm]{images/LDQ_4.png}
	\caption{Plot showing the number of susceptible, exposed, infected and recovered individual over time across the four scenarios involving Lockdown and Quarantine Area }
	\label{LDQ}
\end{figure} 

In Figure \ref{LDQ}, we observe significant disparities in the number of infected individuals across the 220-day simulation period among different scenarios. Notably, Scenario 3.1, characterized by the absence of both Quarantine Area and Lockdown measures, exhibits the highest number of infections over time compared to other scenarios. It peaks on Day 24 with 1224 infected individuals, surpassing the 1000-mark in the graph. 

Moving forward, the remaining three scenarios demonstrate similar trends, albeit with slight variations. Among these, Scenario 3.3 stands out as the second-highest in terms of infected individuals. Despite the minor discrepancies compared to Scenarios 3.2 and 3.4, its initial surge of infections occurs later, around Day 35, in contrast to Days 24 and 28 for Scenarios 3.1 and 3.4, respectively. This delay can be attributed to the presence of a quarantine area, which effectively contains the spread of the virus. However, the absence of a lockdown allows infected agents to continue interacting with others, prolonging the spread.

Interestingly, Scenarios 3.2 and 3.4 exhibit strikingly similar trends in the graph, consistently maintaining the lowest number of infections throughout the 220-day period. This suggests that the implementation of a lockdown significantly contributes to containing the virus by restricting agent movement and interactions, thereby reducing the probability of infection. 

Overall, these observations underscore the critical role of non-pharmaceutical interventions, such as quarantine measures and lockdowns, in mitigating the spread of infectious diseases within simulated populations. The results highlight the importance of implementing comprehensive strategies to effectively control outbreaks and safeguard public health.

This means that having no quarantine area to contain the infected individual and no Lockdown which allows the individual to roam around freely can increase the transmission of the virus. 

The provided data table offers an extensive overview of the number of infected individuals observed across different scenarios which contains the implementation of Lockdown and Quarantine Area. 

\begin{table}[H]
\centering
\begin{tabularx}{\textwidth}{|X|X|X|X|}
	\hline
	\multicolumn{4}{|c|}{\textbf{Infection Rate}} \\
	\hline
	Scenario 3.1& Scenario 3.2 & Scenario 3.3& Scenario 3.4 \\
	\hline
	55.01\% & 28.54\% & 30.16\% & 28.67\%\\
	\hline
\end{tabularx}

\caption{Highest Infection Rates across different scenarios involving Lockdown and having Quarantine Area where N= 2225, int\_expo = 0.10, int\_inf= 0.10, cycle = 31680 cycles }
\label{tab:LD_IR}

\end{table}

\begin{figure}[H]
\centering
\includegraphics[width=16cm, height=8cm]{images/LDQ_IR.png}
\caption{Plot showing the number of the highest infection rates across different scenarios involving the sneraios wiht Lockdown and Quarantine Area }
\label{LDQ_IR}
\end{figure}

The data from Table \ref{tab:LD_IR} and Figure \ref{LDQ_IR}provides a comparative analysis of the infection rates across different scenarios, shedding light on the dynamics of disease transmission within the simulated population.

Starting with Scenario 3.1, which represents the absence of both Quarantine Area and Lockdown measures, we observe the highest infection rate at 55.01\%. This scenario reflects a scenario where no preventive measures are in place, resulting in a substantial proportion of the population becoming infected. Compared to Scenario 3.2, the infection rate in this scenario is 26.47\% higher, indicating the significant impact of implementing lockdown measures in limiting transmission.

Transitioning to Scenario 3.2, where Lockdown measures are implemented but without Quarantine Area, we note a lower infection rate of 28.54\%. This signifies a significant reduction in the spread of the disease compared to Scenario 3.1, highlighting the effectiveness of lockdown measures in limiting transmission. Compared to Scenario 3.3, the infection rate in this scenario is 1.38\% lower, underscoring the importance of implementing comprehensive preventive measures to mitigate disease spread.

In Scenario 3.3, characterized by the presence of Quarantine Area but without Lockdown measures, we observe a slight increase in the infection rate to 30.16\%. Despite the containment efforts provided by the quarantine area, the absence of lockdown allows for continued interaction among individuals, contributing to a higher infection rate compared to Scenario 3.2. Compared to Scenario 3.4, the infection rate in this scenario is 1.49\% lower, suggesting that while quarantine measures can contain the spread to some extent, the absence of lockdown measures may limit their effectiveness.

Finally, in Scenario 3.4, where both Lockdown and Quarantine Area are implemented, we see a further decrease in the infection rate to 28.67\%. This scenario represents the most stringent preventive measures, resulting in the lowest infection rate among the four scenarios. The findings underscore the critical importance of implementing comprehensive strategies, including both lockdown and quarantine measures, to effectively control outbreaks and safeguard public health.



\chapter{Conclusion and Recommendation}
\label{chap:Conclusion and Recommendation}



\section{Conclusion}
\indent \indent This study introduces an agent-based model and compartmental model for simulating the transmission of respiratory infectious diseases, such as COVID-19, within a specific public high school campus, namely, Pines City National High School in Baguio City. The model integrates a shapefile of the campus, meticulously designed and measured using AutoCAD, with its shapefile subsequently generated using QGIS. This comprehensive representation includes the road network, environment, hallways, quarantine areas and classrooms, forming the virtual map. Within this simulated environment, students, teachers, and staff are depicted as agents, enabling them to interact and navigate the virtual setting. Each agent is assigned a state status based on their current health condition, including Susceptible, Exposed, Infected, and Recovered. Additionally, the model also accounts for virus transmission between agents, governed by rules. 

\indent \indent Three main scenarios were tested to evaluate the roles of various parameters, such as protection rate, exposure rate, and the implementation of lockdown and quarantine areas, in the transmission of respiratory infectious diseases within the campus. 

In Section \ref{S1} different percentages of protection rates were examined. Starting from a scenario with no protection, a concerning surge in infections was observed, highlighting the vulnerability of the population in the absence of protective measures. However, as protection rates increased, a substantial decline in the number of infected individuals over time was noted. This trend continued as protection rates escalated, with each increment leading to a significant decrease in infections and a potential reduction in the severity of subsequent waves of infection. Notably, implementing a 100\% protection rate resulted in the lowest number of infected individuals over time, with no new infections recorded after the recovery of initially infected individuals. This signifies that the higher the protection rate, the lower the number of infected individuals.  This underscores the critical importance of comprehensive immunity measures in effectively controlling disease transmission.

Moving on to the second scenario, in section \ref{S2} we evaluate the effect of varying initial exposure rate to the transmission of respiratory infectious disease. Beginning with the scenario of a 75\% exposure rate,  we observe the highest number of infected individuals. This underscores the heightened risk of disease spread when a larger portion of the population is susceptible to the infectious agent. Transitioning to scenarios with lower exposure rates, such as 50\% and 25\%, we note progressively fewer infected individuals, indicating a reduction in infection rates with decreasing exposure levels. However, even at lower exposure rates, the risk of disease transmission remains substantial, as evidenced by the observed peaks in infection numbers and the presence of multiple waves of infection. Finally, in the scenario with a 0\% exposure rate, where there is no initial exposure, we observe the lowest number of infected individuals. This underscores the heightened risk of disease spread when a larger portion of the population is susceptible to the infectious agent.

Lastly, in section \ref{S3}, we assess the impact of having a Lockdown and Quarantine Area in containing the transmission of respiratory infectious disease. In Figure \ref{LDQ}, significant disparities in the number of infected individuals are observed across the 220-day simulation period among different scenarios. Notably, Scenario \ref{3.1b}, characterized by the absence of both Quarantine Area and Lockdown measures, exhibits the highest number of infections over time. The remaining three scenarios demonstrate similar trends, albeit with slight variations. Scenario 3.3, with the presence of a Quarantine Area but no Lockdown, stands out as the second-highest in terms of infected individuals. Conversely, Scenario \ref{3.2}, featuring both Quarantine Area and Lockdown measures, consistently maintains one of the lowest numbers of infections throughout the 220-day period. Similarly, Scenario \ref{3.4}, which includes a Lockdown but no Quarantine Area, also exhibits consistently low numbers of infections over time. These observations underscore the critical role of non-pharmaceutical interventions, such as quarantine measures and lockdowns, in mitigating the spread of infectious diseases within simulated populations. 

Overall,  this study provides valuable insights into the effectiveness of various parameters in controlling the transmission of respiratory infectious diseases within a campus setting. The findings emphasize the importance of comprehensive immunity measures, with higher protection rates leading to fewer infections. Atleast 0\% to 25\% protection rate is needed to lower the number of infection. The examination of varying initial exposure rates underscores the direct correlation between exposure levels and infection rates. Higher exposure rates result in increased infection rates, highlighting the need for preventive strategies to minimize exposure and mitigate the risk of disease spread within susceptible populations. The assessment also of the impact of lockdown and quarantine measures demonstrates the effectiveness of non-pharmaceutical interventions in containing disease transmission. The presence of lockdown and quarantine measures significantly reduces the number of infections over time, underscoring the critical role of proactive interventions in controlling outbreaks and safeguarding public health.






\section{Recommendation}
\indent \indent For future or next study, the author recommends the following avenues for further exploration:
\begin{itemize}
	\item[1.] Instead of focusing on the entire school setting, consider narrowing down the scope of investigation to study the transmission of respiratory infectious diseases within individual classrooms. This targeted approach would provide insights into localized patterns of disease spread and help identify specific areas for intervention within the school environment.
	
	\item[2.] Explore additional parameters and non-pharmaceutical interventions, such as social distancing measures, mask-wearing compliance, and hand hygiene practices, to assess their role in mitigating the transmission of respiratory infectious diseases. By expanding the range of variables under investigation, a more comprehensive understanding of effective disease control strategies can be achieved.
	
	\item[3.] Extend the simulation studies to encompass other settings beyond the school environment. This could include community settings, workplaces, or public transportation systems, to explore the dynamics of disease transmission in diverse contexts. By broadening the scope of analysis, valuable insights can be gained into the effectiveness of intervention strategies across different environments.
\end{itemize}



 

\nocite*      % -- comment out if you want to include all references in the bibliog file
\bibliographystyle{siam}
\bibliography{bibliog}

\appendix
\chapter{Population Data for \\ Pines City National High School }

\begin{figure}[H]
	\centering
	\includegraphics[width=16cm, height=15cm]{images/popu.jpg}

	\label{fig:p}
\end{figure}

\begin{center}
	\begin{tabular}{|c|c|c|c|}
		\hline
		&  &  &  \\
		\hline
		Population & Male & Female & Total \\
		\hline
		Grade 7 & 250 & 254 & 504 \\
		\hline
		Grade 8 & 273 & 262 & 535 \\
		\hline
		Grade 9 & 274 & 303 & 577 \\
		\hline
		Grade 10 & 257 & 263 & 520 \\
		\hline
		Teaching Personnel & \multicolumn{2}{c|}{82} & 82 \\
		\hline
		Non Teaching Personnel & \multicolumn{2}{c|}{37} & 37 \\
		\hline
		Total & \multicolumn{2}{c|}{--} & 2225 \\
		\hline
		
	\end{tabular}
	
\end{center}

\chapter{Table for 0\% Protection Rate}


\begin{longtable}{|l|l|l|l|l|l|}
	\caption{0\% Exposed Rate} \\
	
	\toprule
	\textbf{DAY} & \textbf{N} & \textbf{S} & \textbf{E }& \textbf{I} &\textbf{ R} \\
	\midrule
	\endfirsthead
	\caption*{Table continued from previous page} \\
	\toprule
	\textbf{DAY} & \textbf{N} & \textbf{S} & \textbf{E }& \textbf{I} &\textbf{ R} \\
	\midrule
	\endhead
	\midrule
	\multicolumn{1}{r}{\footnotesize Continued on next page}
	\endfoot
	\bottomrule
	\endlastfoot
		5   & 2225 & 0   & 1375 & 850  & 0   \\
		10  & 2225 & 0   & 281  & 1944 & 0   \\
		15  & 2225 & 0   & 20   & 1801 & 404 \\
		20  & 2225 & 0   & 12   & 1650 & 563 \\
		25  & 2225 & 0   & 9    & 667  & 1549\\
		30  & 2225 & 0   & 6    & 36   & 2183\\
		35  & 2225 & 16  & 4    & 12   & 2193\\
		40  & 2225 & 52  & 11   & 7    & 2155\\
		45  & 2225 & 164 & 26   & 5    & 2030\\
		50  & 2225 & 309 & 108  & 21   & 1787\\
		55  & 2225 & 178 & 406  & 109  & 1532\\
		60  & 2225 & 17  & 559  & 403  & 1246\\
		65  & 2225 & 6   & 423  & 834  & 962 \\
		70  & 2225 & 4   & 355  & 1123 & 743 \\
		75  & 2225 & 7   & 343  & 1172 & 703 \\
		80  & 2225 & 0   & 294  & 1079 & 852 \\
		85  & 2225 & 2   & 214  & 892  & 1117\\
		90  & 2225 & 0   & 89   & 773  & 1363\\
		95  & 2225 & 2   & 44   & 550  & 1629\\
		100 & 2225 & 5   & 78   & 345  & 1797\\
		105 & 2225 & 27  & 121  & 205  & 1872\\
		110 & 2225 & 22  & 180  & 187  & 1836\\
		115 & 2225 & 21  & 240  & 280  & 1684\\
		120 & 2225 & 17  & 290  & 419  & 1499\\
		125 & 2225 & 9   & 315  & 587  & 1314\\
		130 & 2225 & 10  & 354  & 681  & 1180\\
		135 & 2225 & 9   & 321  & 800  & 1095\\
		140 & 2225 & 4   & 318  & 801  & 1102\\
		145 & 2225 & 2   & 227  & 860  & 1136\\
		150 & 2225 & 4   & 206  & 790  & 1225\\
		155 & 2225 & 2   & 181  & 683  & 1359\\
		160 & 2225 & 4   & 163  & 559  & 1499\\
		165 & 2225 & 6   & 158  & 478  & 1583\\
		170 & 2225 & 2   & 162  & 422  & 1639\\
		175 & 2225 & 9   & 163  & 423  & 1630\\
		180 & 2225 & 9   & 195  & 410  & 1611\\
		185 & 2225 & 10  & 255  & 445  & 1515\\
		190 & 2225 & 5   & 293  & 526  & 1401\\
		195 & 2225 & 6   & 275  & 637  & 1307\\
		200 & 2225 & 6   & 262  & 705  & 1252\\
		205 & 2225 & 3   & 249  & 733  & 1240\\
		210 & 2225 & 6   & 267  & 684  & 1268\\
		215 & 2225 & 3   & 227  & 665  & 1330\\
		220 & 2225 & 7   & 212  & 649  & 1357\\
		\bottomrule
\end{longtable}
\chapter{Table for 25\% Protection Rate}


\begin{longtable}{|l|l|l|l|l|l|}
	\caption{25\% Exposed Rate} \\
	
	\toprule
	\textbf{DAY} & \textbf{N} & \textbf{S} & \textbf{E }& \textbf{I} &\textbf{ R} \\
	\midrule
	\endfirsthead
	\caption*{Table continued from previous page} \\
	\toprule
	\textbf{DAY} & \textbf{N} & \textbf{S} & \textbf{E }& \textbf{I} &\textbf{ R} \\
	\midrule
	\endhead
	\midrule
	\multicolumn{1}{r}{\footnotesize Continued on next page}
	\endfoot
	\bottomrule
	\endlastfoot
5  & 2225 & 7   & 1465 & 753  & 0   \\
10 & 2225 & 3   & 557  & 1665 & 0   \\
15 & 2225 & 1   & 224  & 1590 & 410 \\
20 & 2225 & 0   & 108  & 1563 & 554 \\
25 & 2225 & 0   & 49   & 849  & 1327\\
30 & 2225 & 0   & 29   & 279  & 1917\\
35 & 2225 & 9   & 28   & 120  & 2068\\
40 & 2225 & 28  & 42   & 68   & 2087\\
45 & 2225 & 76  & 85   & 60   & 2004\\
50 & 2225 & 94  & 222  & 98   & 1811\\
55 & 2225 & 42  & 368  & 213  & 1602\\
60 & 2225 & 15  & 459  & 406  & 1345\\
65 & 2225 & 10  & 480  & 633  & 1102\\
70 & 2225 & 10  & 474  & 821  & 920 \\
75 & 2225 & 8   & 464  & 907  & 846 \\
80 & 2225 & 3   & 410  & 928  & 884 \\
85 & 2225 & 8   & 349  & 876  & 992 \\
90 & 2225 & 3   & 219  & 821  & 1182\\
95 & 2225 & 3   & 148  & 669  & 1405\\
100& 2225 & 5   & 133  & 503  & 1584\\
105& 2225 & 11  & 148  & 348  & 1718\\
110& 2225 & 13  & 207  & 283  & 1722\\
115& 2225 & 21  & 231  & 306  & 1667\\
120& 2225 & 21  & 292  & 371  & 1541\\
125& 2225 & 17  & 340  & 451  & 1417\\
130& 2225 & 9   & 368  & 557  & 1291\\
135& 2225 & 17  & 387  & 624  & 1197\\
140& 2225 & 10  & 363  & 705  & 1147\\
145& 2225 & 5   & 338  & 746  & 1136\\
150& 2225 & 7   & 297  & 744  & 1177\\
155& 2225 & 8   & 279  & 688  & 1250\\
160& 2225 & 8   & 254  & 634  & 1329\\
165& 2225 & 13  & 213  & 556  & 1443\\
170& 2225 & 6   & 229  & 494  & 1496\\
175& 2225 & 11  & 234  & 456  & 1524\\
180& 2225 & 13  & 260  & 421  & 1531\\
185& 2225 & 10  & 285  & 446  & 1484\\
190& 2225 & 6   & 318  & 495  & 1406\\
195& 2225 & 11  & 336  & 552  & 1326\\
200& 2225 & 12  & 346  & 604  & 1263\\
205& 2225 & 9   & 311  & 636  & 1269\\
210& 2225 & 6   & 331  & 649  & 1239\\
215& 2225 & 11  & 302  & 664  & 1248\\
220& 2225 & 11  & 291  & 605  & 1318\\
	\bottomrule
\end{longtable}
\chapter{Table for 50\% Protection Rate}
\begin{longtable}{|l|l|l|l|l|l|}
	\caption{50\% Exposed Rate} \\
	
	\toprule
	\textbf{DAY} & \textbf{N} & \textbf{S} & \textbf{E }& \textbf{I} &\textbf{ R} \\
	\midrule
	\endfirsthead
	\caption*{Table continued from previous page} \\
	\toprule
	\textbf{DAY} & \textbf{N} & \textbf{S} & \textbf{E }& \textbf{I} &\textbf{ R} \\
	\midrule
	\endhead
	\midrule
	\multicolumn{1}{r}{\footnotesize Continued on next page}
	\endfoot
	\bottomrule
	\endlastfoot

5 & 2225 & 23 & 1592 & 610 & 0 \\
10 & 2225 & 7 & 979 & 1239 & 0 \\
15 & 2225 & 5 & 595 & 1232 & 393 \\
20 & 2225 & 2 & 363 & 1361 & 499 \\
25 & 2225 & 3 & 223 & 994 & 1005 \\
30 & 2225 & 6 & 134 & 587 & 1498 \\
35 & 2225 & 9 & 96 & 353 & 1767 \\
40 & 2225 & 10 & 109 & 230 & 1876 \\
45 & 2225 & 32 & 128 & 171 & 1894 \\
50 & 2225 & 47 & 221 & 156 & 1801 \\
55 & 2225 & 50 & 356 & 205 & 1614 \\
60 & 2225 & 38 & 450 & 298 & 1439 \\
65 & 2225 & 28 & 523 & 410 & 1264 \\
70 & 2225 & 22 & 583 & 555 & 1065 \\
75 & 2225 & 18 & 593 & 645 & 969 \\
80 & 2225 & 14 & 583 & 744 & 884 \\
85 & 2225 & 10 & 561 & 759 & 895 \\
90 & 2225 & 14 & 472 & 757 & 982 \\
95 & 2225 & 10 & 396 & 717 & 1102 \\
100 & 2225 & 8 & 317 & 638 & 1262 \\
105 & 2225 & 8 & 299 & 543 & 1375 \\
110 & 2225 & 16 & 280 & 442 & 1487 \\
115 & 2225 & 16 & 293 & 402 & 1514 \\
120 & 2225 & 23 & 343 & 372 & 1487 \\
125 & 2225 & 17 & 366 & 390 & 1452 \\
130 & 2225 & 22 & 397 & 433 & 1373 \\
135 & 2225 & 18 & 458 & 482 & 1267 \\
140 & 2225 & 22 & 475 & 537 & 1191 \\
145 & 2225 & 11 & 495 & 570 & 1149 \\
150 & 2225 & 15 & 481 & 611 & 1118 \\
155 & 2225 & 12 & 456 & 594 & 1163 \\
160 & 2225 & 11 & 428 & 613 & 1173 \\
165 & 2225 & 17 & 405 & 582 & 1221 \\
170 & 2225 & 17 & 389 & 565 & 1254 \\
175 & 2225 & 12 & 353 & 548 & 1312 \\
180 & 2225 & 16 & 352 & 502 & 1355 \\
185 & 2225 & 18 & 367 & 490 & 1350 \\
190 & 2225 & 20 & 386 & 456 & 1363 \\
195 & 2225 & 19 & 399 & 463 & 1344 \\
200 & 2225 & 22 & 420 & 475 & 1308 \\
205 & 2225 & 13 & 441 & 517 & 1254 \\
210 & 2225 & 18 & 447 & 541 & 1219 \\
215 & 2225 & 11 & 439 & 577 & 1198 \\
220 & 2225 & 12 & 437 & 581 & 1195 \\

	\bottomrule
\end{longtable}
\chapter{Table for 75\% Protection Rate}
\begin{longtable}{|l|l|l|l|l|l|}
	\caption{75\% Exposed Rate} \\
	
	\toprule
	\textbf{DAY} & \textbf{N} & \textbf{S} & \textbf{E }& \textbf{I} &\textbf{ R} \\
	\midrule
	\endfirsthead
	\caption*{Table continued from previous page} \\
	\toprule
	\textbf{DAY} & \textbf{N} & \textbf{S} & \textbf{E }& \textbf{I} &\textbf{ R} \\
	\midrule
	\endhead
	\midrule
	\multicolumn{1}{r}{\footnotesize Continued on next page}
	\endfoot
	\bottomrule
	\endlastfoot
5 & 2225 & 48 & 1663 & 514 & 0 \\
10 & 2225 & 23 & 1372 & 830 & 0 \\
15 & 2225 & 16 & 1098 & 703 & 408 \\
20 & 2225 & 10 & 890 & 875 & 450 \\
25 & 2225 & 15 & 709 & 785 & 716 \\
30 & 2225 & 13 & 571 & 627 & 1014 \\
35 & 2225 & 21 & 465 & 508 & 1231 \\
40 & 2225 & 20 & 412 & 423 & 1370 \\
45 & 2225 & 34 & 400 & 352 & 1439 \\
50 & 2225 & 28 & 441 & 305 & 1451 \\
55 & 2225 & 45 & 497 & 285 & 1398 \\
60 & 2225 & 51 & 560 & 291 & 1323 \\
65 & 2225 & 60 & 645 & 307 & 1213 \\
70 & 2225 & 46 & 720 & 361 & 1098 \\
75 & 2225 & 27 & 796 & 405 & 997 \\
80 & 2225 & 36 & 766 & 478 & 945 \\
85 & 2225 & 28 & 810 & 511 & 876 \\
90 & 2225 & 24 & 793 & 546 & 862 \\
95 & 2225 & 24 & 761 & 544 & 896 \\
100 & 2225 & 31 & 701 & 540 & 953 \\
105 & 2225 & 22 & 666 & 513 & 1024 \\
110 & 2225 & 24 & 647 & 487 & 1067 \\
115 & 2225 & 18 & 642 & 452 & 1113 \\
120 & 2225 & 29 & 605 & 421 & 1170 \\
125 & 2225 & 29 & 606 & 401 & 1189 \\
130 & 2225 & 31 & 596 & 403 & 1195 \\
135 & 2225 & 30 & 629 & 379 & 1187 \\
140 & 2225 & 36 & 661 & 371 & 1157 \\
145 & 2225 & 33 & 677 & 401 & 1114 \\
150 & 2225 & 31 & 712 & 418 & 1064 \\
155 & 2225 & 33 & 719 & 461 & 1012 \\
160 & 2225 & 25 & 732 & 497 & 971 \\
165 & 2225 & 26 & 728 & 493 & 978 \\
170 & 2225 & 24 & 718 & 488 & 995 \\
175 & 2225 & 27 & 680 & 483 & 1035 \\
180 & 2225 & 19 & 671 & 475 & 1060 \\
185 & 2225 & 30 & 643 & 459 & 1093 \\
190 & 2225 & 26 & 671 & 431 & 1097 \\
195 & 2225 & 31 & 666 & 440 & 1088 \\
200 & 2225 & 28 & 658 & 444 & 1095 \\
205 & 2225 & 33 & 683 & 440 & 1069 \\
210 & 2225 & 28 & 680 & 462 & 1055 \\
215 & 2225 & 32 & 678 & 455 & 1060 \\
220 & 2225 & 23 & 693 & 465 & 1044 \\

	\bottomrule
\end{longtable}
\chapter{Table for 100\% Protection Rate}


\begin{longtable}{|l|l|l|l|l|l|}
	\caption{0\% Exposed Rate} \\
	
	\toprule
	\textbf{DAY} & \textbf{N} & \textbf{S} & \textbf{E }& \textbf{I} &\textbf{ R} \\
	\midrule
	\endfirsthead
	\caption*{Table continued from previous page} \\
	\toprule
	\textbf{DAY} & \textbf{N} & \textbf{S} & \textbf{E }& \textbf{I} &\textbf{ R} \\
	\midrule
	\endhead
	\midrule
	\multicolumn{1}{r}{\footnotesize Continued on next page}
	\endfoot
	\bottomrule
	\endlastfoot
5 & 2225 & 66 & 1750 & 409 & 0 \\
10 & 2225 & 95 & 1721 & 409 & 0 \\
15 & 2225 & 544 & 1272 & 0 & 409 \\
20 & 2225 & 1570 & 246 & 0 & 409 \\
25 & 2225 & 1787 & 29 & 0 & 409 \\
30 & 2225 & 1804 & 12 & 0 & 409 \\
35 & 2225 & 1838 & 7 & 0 & 380 \\
40 & 2225 & 1894 & 4 & 0 & 327 \\
45 & 2225 & 1949 & 3 & 0 & 273 \\
50 & 2225 & 1992 & 3 & 0 & 230 \\
55 & 2225 & 2039 & 3 & 0 & 183 \\
60 & 2225 & 2106 & 1 & 0 & 118 \\
65 & 2225 & 2148 & 1 & 0 & 76 \\
70 & 2225 & 2200 & 1 & 0 & 24 \\
75 & 2225 & 2224 & 1 & 0 & 0 \\
80 & 2225 & 2224 & 1 & 0 & 0 \\
85 & 2225 & 2224 & 1 & 0 & 0 \\
90 & 2225 & 2224 & 1 & 0 & 0 \\
95 & 2225 & 2224 & 1 & 0 & 0 \\
100 & 2225 & 2224 & 1 & 0 & 0 \\
105 & 2225 & 2224 & 1 & 0 & 0 \\
110 & 2225 & 2224 & 1 & 0 & 0 \\
115 & 2225 & 2224 & 1 & 0 & 0 \\
120 & 2225 & 2224 & 1 & 0 & 0 \\
125 & 2225 & 2224 & 1 & 0 & 0 \\
130 & 2225 & 2224 & 1 & 0 & 0 \\
135 & 2225 & 2224 & 1 & 0 & 0 \\
140 & 2225 & 2224 & 1 & 0 & 0 \\
145 & 2225 & 2224 & 1 & 0 & 0 \\
150 & 2225 & 2224 & 1 & 0 & 0 \\
155 & 2225 & 2224 & 1 & 0 & 0 \\
160 & 2225 & 2224 & 1 & 0 & 0 \\
165 & 2225 & 2224 & 1 & 0 & 0 \\
170 & 2225 & 2224 & 1 & 0 & 0 \\
175 & 2225 & 2224 & 1 & 0 & 0 \\
180 & 2225 & 2224 & 1 & 0 & 0 \\
185 & 2225 & 2224 & 1 & 0 & 0 \\
190 & 2225 & 2224 & 1 & 0 & 0 \\
195 & 2225 & 2224 & 1 & 0 & 0 \\
200 & 2225 & 2224 & 1 & 0 & 0 \\
205 & 2225 & 2224 & 1 & 0 & 0 \\
210 & 2225 & 2224 & 1 & 0 & 0 \\
215 & 2225 & 2224 & 1 & 0 & 0 \\
220 & 2225 & 2224 & 1 & 0 & 0 \\

	\bottomrule
\end{longtable}
\chapter{Table for the comparison of\\ all the Protection Rates}
\begin{center}
	\begin{tabularx}{\textwidth}{|X|X|X|X|X|X|}
		\hline
		\multicolumn{6}{|c|}{PROTECTION RATE} \\
		\hline
		& \multicolumn{4}{c|}{Infection rate} &  \\
		\hline
		& 0\% & 25\% & 50\% & 75\% & 100\% \\
		\hline
		S & 0.568 & 0.566 & 0.573 & 0.566 & 0.999 \\
		\hline
		E & 0.798 & 0.797 & 0.8.2 & 0.812 & 0.804 \\
		\hline
		I & 0.931 & 0.802 & 0.624 & 0.396 & 0.250 \\
		\hline
		R & 0.924 & 0.933 & 0.856 & 0.653 & 0.183 \\
		\hline
	\end{tabularx}
\end{center}

\chapter{Table for 0\% Initial Exposure Rate}


\begin{longtable}{|l|l|l|l|l|l|}
	\caption{0\% Initial Exposure Rate} \\
	
	\toprule
	\textbf{DAY} & \textbf{N} & \textbf{S} & \textbf{E }& \textbf{I} &\textbf{ R} \\
	\midrule
	\endfirsthead
	\caption*{Table continued from previous page} \\
	\toprule
	\textbf{DAY} & \textbf{N} & \textbf{S} & \textbf{E }& \textbf{I} &\textbf{ R} \\
	\midrule
	\endhead
	\midrule
	\multicolumn{1}{r}{\footnotesize Continued on next page}
	\endfoot
	\bottomrule
	\endlastfoot

5 & 2225 & 10 & 1503 & 712 & 0 \\
10 & 2225 & 14 & 1000 & 1211 & 0 \\
15 & 2225 & 5 & 691 & 973 & 556 \\
20 & 2225 & 4 & 466 & 1139 & 616 \\
25 & 2225 & 2 & 318 & 871 & 1034 \\
30 & 2225 & 6 & 206 & 579 & 1434 \\
35 & 2225 & 10 & 169 & 386 & 1660 \\
40 & 2225 & 17 & 189 & 271 & 1748 \\
45 & 2225 & 34 & 203 & 206 & 1782 \\
50 & 2225 & 41 & 299 & 186 & 1699 \\
55 & 2225 & 46 & 409 & 233 & 1537 \\
60 & 2225 & 40 & 483 & 313 & 1389 \\
65 & 2225 & 22 & 569 & 431 & 1203 \\
70 & 2225 & 25 & 642 & 509 & 1049 \\
75 & 2225 & 18 & 694 & 592 & 921 \\
80 & 2225 & 16 & 681 & 631 & 897 \\
85 & 2225 & 12 & 629 & 690 & 894 \\
90 & 2225 & 15 & 551 & 708 & 951 \\
95 & 2225 & 12 & 497 & 674 & 1042 \\
100 & 2225 & 16 & 416 & 603 & 1190 \\
105 & 2225 & 13 & 391 & 521 & 1300 \\
110 & 2225 & 20 & 378 & 455 & 1372 \\
115 & 2225 & 30 & 373 & 428 & 1394 \\
120 & 2225 & 20 & 391 & 399 & 1415 \\
125 & 2225 & 17 & 420 & 420 & 1368 \\
130 & 2225 & 25 & 437 & 439 & 1324 \\
135 & 2225 & 20 & 500 & 452 & 1253 \\
140 & 2225 & 22 & 509 & 489 & 1205 \\
145 & 2225 & 22 & 539 & 508 & 1156 \\
150 & 2225 & 17 & 551 & 531 & 1126 \\
155 & 2225 & 21 & 542 & 570 & 1092 \\
160 & 2225 & 15 & 525 & 572 & 1113 \\
165 & 2225 & 16 & 496 & 579 & 1134 \\
170 & 2225 & 16 & 480 & 550 & 1179 \\
175 & 2225 & 17 & 453 & 527 & 1228 \\
180 & 2225 & 17 & 445 & 504 & 1259 \\
185 & 2225 & 15 & 457 & 476 & 1277 \\
190 & 2225 & 14 & 460 & 467 & 1284 \\
195 & 2225 & 22 & 442 & 481 & 1280 \\
200 & 2225 & 11 & 440 & 501 & 1273 \\
205 & 2225 & 23 & 460 & 510 & 1232 \\
210 & 2225 & 16 & 468 & 507 & 1234 \\
215 & 2225 & 20 & 518 & 488 & 1199 \\
220 & 2225 & 26 & 492 & 524 & 1183 \\
	
	\bottomrule
\end{longtable}





\chapter{Table for 25\% Initial Exposure Rate}


\begin{longtable}{|l|l|l|l|l|l|}
	\caption{25\% Initial Exposure Rate} \\
	
	\toprule
	\textbf{DAY} & \textbf{N} & \textbf{S} & \textbf{E }& \textbf{I} &\textbf{ R} \\
	\midrule
	\endfirsthead
	\caption*{Table continued from previous page} \\
	\toprule
	\textbf{DAY} & \textbf{N} & \textbf{S} & \textbf{E }& \textbf{I} &\textbf{ R} \\
	\midrule
	\endhead
	\midrule
	\multicolumn{1}{r}{\footnotesize Continued on next page}
	\endfoot
	\bottomrule
	\endlastfoot
	
5 & 2225 & 16 & 1578 & 631 & 0 \\
10 & 2225 & 7 & 1005 & 1213 & 0 \\
15 & 2225 & 6 & 626 & 1174 & 419 \\
20 & 2225 & 5 & 404 & 1310 & 506 \\
25 & 2225 & 3 & 264 & 955 & 1003 \\
30 & 2225 & 2 & 174 & 557 & 1492 \\
35 & 2225 & 9 & 137 & 376 & 1703 \\
40 & 2225 & 16 & 139 & 223 & 1847 \\
45 & 2225 & 36 & 161 & 160 & 1868 \\
50 & 2225 & 49 & 249 & 144 & 1783 \\
55 & 2225 & 53 & 354 & 203 & 1615 \\
60 & 2225 & 53 & 477 & 276 & 1419 \\
65 & 2225 & 33 & 577 & 407 & 1208 \\
70 & 2225 & 24 & 664 & 511 & 1026 \\
75 & 2225 & 23 & 654 & 645 & 903 \\
80 & 2225 & 13 & 624 & 737 & 851 \\
85 & 2225 & 14 & 603 & 777 & 831 \\
90 & 2225 & 13 & 500 & 771 & 941 \\
95 & 2225 & 10 & 418 & 696 & 1101 \\
100 & 2225 & 9 & 345 & 615 & 1256 \\
105 & 2225 & 14 & 311 & 546 & 1354 \\
110 & 2225 & 14 & 285 & 450 & 1476 \\
115 & 2225 & 18 & 305 & 387 & 1515 \\
120 & 2225 & 19 & 344 & 371 & 1491 \\
125 & 2225 & 33 & 372 & 367 & 1453 \\
130 & 2225 & 22 & 437 & 390 & 1376 \\
135 & 2225 & 30 & 484 & 424 & 1287 \\
140 & 2225 & 24 & 501 & 493 & 1207 \\
145 & 2225 & 20 & 550 & 539 & 1116 \\
150 & 2225 & 16 & 550 & 587 & 1072 \\
155 & 2225 & 12 & 502 & 623 & 1088 \\
160 & 2225 & 13 & 474 & 634 & 1104 \\
165 & 2225 & 2 & 444 & 621 & 1158 \\
170 & 2225 & 13 & 405 & 584 & 1223 \\
175 & 2225 & 13 & 391 & 545 & 1276 \\
180 & 2225 & 16 & 400 & 487 & 1322 \\
185 & 2225 & 15 & 399 & 465 & 1346 \\
190 & 2225 & 17 & 392 & 454 & 1362 \\
195 & 2225 & 18 & 391 & 462 & 1354 \\
200 & 2225 & 23 & 433 & 447 & 1322 \\
205 & 2225 & 16 & 471 & 463 & 1275 \\
210 & 2225 & 23 & 446 & 505 & 1251 \\
215 & 2225 & 15 & 465 & 554 & 1191 \\
220 & 2225 & 17 & 475 & 571 & 1162 \\

	\bottomrule
\end{longtable}





\chapter{Table for 50\% Initial Exposure Rate}


\begin{longtable}{|l|l|l|l|l|l|}
	\caption{50\% Initial Exposure Rate} \\
	
	\toprule
	\textbf{DAY} & \textbf{N} & \textbf{S} & \textbf{E }& \textbf{I} &\textbf{ R} \\
	\midrule
	\endfirsthead
	\caption*{Table continued from previous page} \\
	\toprule
	\textbf{DAY} & \textbf{N} & \textbf{S} & \textbf{E }& \textbf{I} &\textbf{ R} \\
	\midrule
	\endhead
	\midrule
	\multicolumn{1}{r}{\footnotesize Continued on next page}
	\endfoot
	\bottomrule
	\endlastfoot
	
5 & 2225 & 21 & 1682 & 522 & 0 \\
10 & 2225 & 13 & 977 & 1235 & 0 \\
15 & 2225 & 4 & 586 & 1356 & 279 \\
20 & 2225 & 4 & 325 & 1517 & 379 \\
25 & 2225 & 1 & 203 & 1087 & 934 \\
30 & 2225 & 4 & 116 & 572 & 1533 \\
35 & 2225 & 1 & 89 & 337 & 1798 \\
40 & 2225 & 8 & 89 & 199 & 1929 \\
45 & 2225 & 24 & 107 & 125 & 1969 \\
50 & 2225 & 62 & 181 & 116 & 1866 \\
55 & 2225 & 60 & 288 & 162 & 1715 \\
60 & 2225 & 40 & 435 & 243 & 1507 \\
65 & 2225 & 41 & 526 & 359 & 1299 \\
70 & 2225 & 26 & 631 & 506 & 1062 \\
75 & 2225 & 22 & 625 & 657 & 921 \\
80 & 2225 & 14 & 646 & 760 & 805 \\
85 & 2225 & 9 & 601 & 836 & 779 \\
90 & 2225 & 4 & 493 & 829 & 899 \\
95 & 2225 & 5 & 395 & 774 & 1051 \\
100 & 2225 & 6 & 327 & 646 & 1246 \\
105 & 2225 & 8 & 274 & 528 & 1415 \\
110 & 2225 & 14 & 242 & 435 & 1534 \\
115 & 2225 & 20 & 254 & 372 & 1579 \\
120 & 2225 & 17 & 294 & 353 & 1561 \\
125 & 2225 & 38 & 357 & 329 & 1501 \\
130 & 2225 & 18 & 427 & 368 & 1412 \\
135 & 2225 & 21 & 483 & 429 & 1292 \\
140 & 2225 & 23 & 508 & 515 & 1179 \\
145 & 2225 & 22 & 535 & 564 & 1104 \\
150 & 2225 & 15 & 527 & 620 & 1063 \\
155 & 2225 & 6 & 477 & 693 & 1049 \\
160 & 2225 & 12 & 426 & 700 & 1087 \\
165 & 2225 & 8 & 384 & 683 & 1150 \\
170 & 2225 & 20 & 344 & 612 & 1249 \\
175 & 2225 & 12 & 339 & 533 & 1341 \\
180 & 2225 & 12 & 347 & 481 & 1385 \\
185 & 2225 & 15 & 336 & 467 & 1407 \\
190 & 2225 & 17 & 358 & 446 & 1404 \\
195 & 2225 & 14 & 395 & 436 & 1380 \\
200 & 2225 & 21 & 405 & 443 & 1356 \\
205 & 2225 & 21 & 443 & 451 & 1310 \\
210 & 2225 & 20 & 466 & 486 & 1253 \\
215 & 2225 & 17 & 470 & 535 & 1203 \\
220 & 2225 & 8 & 469 & 573 & 1175 \\

	
	\bottomrule
\end{longtable}





\chapter{Table for 75\% Initial Exposure Rate}


\begin{longtable}{|l|l|l|l|l|l|}
	\caption{75\% Initial Exposure Rate} \\
	
	\toprule
	\textbf{DAY} & \textbf{N} & \textbf{S} & \textbf{E }& \textbf{I} &\textbf{ R} \\
	\midrule
	\endfirsthead
	\caption*{Table continued from previous page} \\
	\toprule
	\textbf{DAY} & \textbf{N} & \textbf{S} & \textbf{E }& \textbf{I} &\textbf{ R} \\
	\midrule
	\endhead
	\midrule
	\multicolumn{1}{r}{\footnotesize Continued on next page}
	\endfoot
	\bottomrule
	\endlastfoot
	
5 & 2225 & 27 & 1740 & 458 & 0 \\
10 & 2225 & 4 & 880 & 1341 & 0 \\
15 & 2225 & 2 & 457 & 1630 & 136 \\
20 & 2225 & 1 & 211 & 1760 & 253 \\
25 & 2225 & 6 & 109 & 1110 & 1000 \\
30 & 2225 & 1 & 66 & 525 & 1633 \\
35 & 2225 & 2 & 52 & 242 & 1929 \\
40 & 2225 & 6 & 38 & 113 & 2068 \\
45 & 2225 & 41 & 63 & 75 & 2046 \\
50 & 2225 & 93 & 153 & 71 & 1908 \\
55 & 2225 & 106 & 309 & 112 & 1698 \\
60 & 2225 & 43 & 469 & 243 & 1470 \\
65 & 2225 & 26 & 546 & 435 & 1218 \\
70 & 2225 & 9 & 536 & 655 & 1025 \\
75 & 2225 & 15 & 569 & 784 & 857 \\
80 & 2225 & 11 & 543 & 876 & 795 \\
85 & 2225 & 10 & 493 & 874 & 848 \\
90 & 2225 & 6 & 391 & 842 & 986 \\
95 & 2225 & 5 & 296 & 755 & 1169 \\
100 & 2225 & 3 & 202 & 637 & 1383 \\
105 & 2225 & 10 & 163 & 505 & 1547 \\
110 & 2225 & 16 & 194 & 395 & 1620 \\
115 & 2225 & 20 & 228 & 319 & 1658 \\
120 & 2225 & 18 & 254 & 312 & 1641 \\
125 & 2225 & 18 & 327 & 354 & 1526 \\
130 & 2225 & 23 & 380 & 397 & 1425 \\
135 & 2225 & 22 & 456 & 473 & 1274 \\
140 & 2225 & 14 & 463 & 574 & 1174 \\
145 & 2225 & 11 & 437 & 675 & 1102 \\
150 & 2225 & 7 & 411 & 725 & 1082 \\
155 & 2225 & 7 & 383 & 716 & 1119 \\
160 & 2225 & 4 & 366 & 683 & 1172 \\
165 & 2225 & 7 & 306 & 626 & 1286 \\
170 & 2225 & 10 & 292 & 585 & 1338 \\
175 & 2225 & 8 & 284 & 515 & 1418 \\
180 & 2225 & 14 & 285 & 454 & 1472 \\
185 & 2225 & 14 & 304 & 423 & 1484 \\
190 & 2225 & 14 & 337 & 440 & 1434 \\
195 & 2225 & 21 & 338 & 475 & 1391 \\
200 & 2225 & 8 & 363 & 523 & 1331 \\
205 & 2225 & 18 & 378 & 561 & 1268 \\
210 & 2225 & 8 & 394 & 588 & 1235 \\
215 & 2225 & 12 & 388 & 603 & 1222 \\
220 & 2225 & 8 & 399 & 607 & 1211 \\

	
	\bottomrule
\end{longtable}





\chapter{Table for the comparison of\\ the Implementation of Lockdown  \\and Quarantine Area}

\begin{center}
\begin{tabularx}{\textwidth}{|X|X|X|X|X|}
	\hline
	\multicolumn{5}{|c|}{Initial Exposure Rate} \\
	\hline
	& \multicolumn{4}{c|}{Exposure rate} \\
	\hline
	& 0\% & 25\% & 50\% & 75\% \\
	\hline
	S & 0.751 & 0.560 & 0.375 & 0.189 \\
	\hline
	E & 0.747 & 0.798 & 0.841 & 0.869 \\
	\hline
	I & 0.575 & 0.612 & 0.742 & 0.802 \\
	\hline
	R & 0.802 & 0.842 & 0.888 & 0.935 \\
	\hline
\end{tabularx}
\end{center}
\chapter{Table for No Lockdown \\ and No Quarantine Area}


\begin{longtable}{|l|l|l|l|l|l|}
	\caption{No Lockdown and No Quarantine Area} \\
	
	\toprule
	\textbf{DAY} & \textbf{N} & \textbf{S} & \textbf{E }& \textbf{I} &\textbf{ R} \\
	\midrule
	\endfirsthead
	\caption*{Table continued from previous page} \\
	\toprule
	\textbf{DAY} & \textbf{N} & \textbf{S} & \textbf{E }& \textbf{I} &\textbf{ R} \\
	\midrule
	\endhead
	\midrule
	\multicolumn{1}{r}{\footnotesize Continued on next page}
	\endfoot
	\bottomrule
	\endlastfoot
	
5 & 2225 & 1129 & 866 & 230 & 0 \\
10 & 2225 & 473 & 1251 & 501 & 0 \\
15 & 2225 & 189 & 1128 & 713 & 195 \\
20 & 2225 & 96 & 843 & 1080 & 206 \\
25 & 2225 & 61 & 597 & 1208 & 359 \\
30 & 2225 & 43 & 433 & 1001 & 748 \\
35 & 2225 & 43 & 329 & 720 & 1133 \\
40 & 2225 & 54 & 261 & 486 & 1424 \\
45 & 2225 & 71 & 214 & 332 & 1608 \\
50 & 2225 & 89 & 208 & 244 & 1684 \\
55 & 2225 & 135 & 214 & 203 & 1673 \\
60 & 2225 & 203 & 254 & 189 & 1579 \\
65 & 2225 & 280 & 308 & 200 & 1437 \\
70 & 2225 & 305 & 418 & 241 & 1261 \\
75 & 2225 & 277 & 515 & 305 & 1128 \\
80 & 2225 & 271 & 587 & 402 & 965 \\
85 & 2225 & 242 & 658 & 500 & 825 \\
90 & 2225 & 173 & 702 & 616 & 734 \\
95 & 2225 & 134 & 663 & 697 & 731 \\
100 & 2225 & 106 & 597 & 730 & 792 \\
105 & 2225 & 81 & 502 & 712 & 930 \\
110 & 2225 & 84 & 420 & 656 & 1065 \\
115 & 2225 & 77 & 389 & 544 & 1215 \\
120 & 2225 & 97 & 350 & 469 & 1309 \\
125 & 2225 & 106 & 335 & 379 & 1405 \\
130 & 2225 & 120 & 343 & 336 & 1426 \\
135 & 2225 & 160 & 354 & 306 & 1405 \\
140 & 2225 & 196 & 390 & 312 & 1327 \\
145 & 2225 & 242 & 426 & 329 & 1228 \\
150 & 2225 & 233 & 495 & 374 & 1123 \\
155 & 2225 & 205 & 550 & 420 & 1050 \\
160 & 2225 & 196 & 574 & 464 & 991 \\
165 & 2225 & 153 & 597 & 538 & 937 \\
170 & 2225 & 134 & 581 & 593 & 917 \\
175 & 2225 & 122 & 551 & 610 & 942 \\
180 & 2225 & 124 & 512 & 589 & 1000 \\
185 & 2225 & 129 & 478 & 546 & 1072 \\
190 & 2225 & 116 & 477 & 498 & 1134 \\
195 & 2225 & 125 & 424 & 495 & 1181 \\
200 & 2225 & 121 & 423 & 463 & 1218 \\
205 & 2225 & 139 & 425 & 441 & 1220 \\
210 & 2225 & 151 & 408 & 419 & 1247 \\
215 & 2225 & 146 & 442 & 399 & 1238 \\
220 & 2225 & 199 & 463 & 380 & 1183 \\

	\bottomrule
\end{longtable}





\chapter{Table for Lockdown and \\ No Quarantine Area}


\begin{longtable}{|l|l|l|l|l|l|}
	\caption{No Lockdown and No Quarantine Area} \\
	
	\toprule
	\textbf{DAY} & \textbf{N} & \textbf{S} & \textbf{E }& \textbf{I} &\textbf{ R} \\
	\midrule
	\endfirsthead
	\caption*{Table continued from previous page} \\
	\toprule
	\textbf{DAY} & \textbf{N} & \textbf{S} & \textbf{E }& \textbf{I} &\textbf{ R} \\
	\midrule
	\endhead
	\midrule
	\multicolumn{1}{r}{\footnotesize Continued on next page}
	\endfoot
	\bottomrule
	\endlastfoot
	
5 & 2225 & 1308 & 690 & 227 & 0 \\
10 & 2225 & 1016 & 846 & 363 & 0 \\
15 & 2225 & 804 & 853 & 367 & 201 \\
20 & 2225 & 628 & 849 & 538 & 210 \\
25 & 2225 & 500 & 805 & 635 & 285 \\
30 & 2225 & 420 & 752 & 566 & 487 \\
35 & 2225 & 350 & 697 & 501 & 677 \\
40 & 2225 & 319 & 650 & 439 & 817 \\
45 & 2225 & 323 & 592 & 371 & 939 \\
50 & 2225 & 335 & 553 & 315 & 1022 \\
55 & 2225 & 344 & 540 & 267 & 1074 \\
60 & 2225 & 382 & 530 & 241 & 1072 \\
65 & 2225 & 423 & 525 & 245 & 1032 \\
70 & 2225 & 491 & 530 & 242 & 962 \\
75 & 2225 & 535 & 554 & 232 & 904 \\
80 & 2225 & 574 & 566 & 238 & 847 \\
85 & 2225 & 607 & 593 & 244 & 781 \\
90 & 2225 & 675 & 588 & 271 & 691 \\
95 & 2225 & 666 & 618 & 298 & 643 \\
100 & 2225 & 643 & 634 & 324 & 624 \\
105 & 2225 & 622 & 655 & 339 & 609 \\
110 & 2225 & 606 & 653 & 359 & 607 \\
115 & 2225 & 565 & 663 & 353 & 644 \\
120 & 2225 & 498 & 686 & 367 & 674 \\
125 & 2225 & 463 & 681 & 380 & 701 \\
130 & 2225 & 455 & 652 & 366 & 752 \\
135 & 2225 & 450 & 654 & 340 & 781 \\
140 & 2225 & 472 & 629 & 314 & 810 \\
145 & 2225 & 482 & 619 & 294 & 830 \\
150 & 2225 & 489 & 618 & 288 & 830 \\
155 & 2225 & 478 & 618 & 289 & 840 \\
160 & 2225 & 478 & 641 & 291 & 815 \\
165 & 2225 & 461 & 658 & 317 & 789 \\
170 & 2225 & 472 & 654 & 322 & 777 \\
175 & 2225 & 469 & 660 & 332 & 764 \\
180 & 2225 & 503 & 630 & 336 & 756 \\
185 & 2225 & 482 & 645 & 345 & 753 \\
190 & 2225 & 474 & 659 & 335 & 757 \\
195 & 2225 & 464 & 665 & 330 & 766 \\
200 & 2225 & 491 & 636 & 334 & 764 \\
205 & 2225 & 467 & 634 & 349 & 775 \\
210 & 2225 & 467 & 641 & 316 & 801 \\
215 & 2225 & 476 & 627 & 302 & 820 \\
220 & 2225 & 487 & 631 & 304 & 803 \\


	
	\bottomrule
\end{longtable}





\input{appendix12}
\chapter{Table for Quarantine Area and Lockdown}


\begin{longtable}{|l|l|l|l|l|l|}
	\caption{Table for Quarantine Area and Lockdown} \\
	
	\toprule
	\textbf{DAY} & \textbf{N} & \textbf{S} & \textbf{E }& \textbf{I} &\textbf{ R} \\
	\midrule
	\endfirsthead
	\caption*{Table continued from previous page} \\
	\toprule
	\textbf{DAY} & \textbf{N} & \textbf{S} & \textbf{E }& \textbf{I} &\textbf{ R} \\
	\midrule
	\endhead
	\midrule
	\multicolumn{1}{r}{\footnotesize Continued on next page}
	\endfoot
	\bottomrule
	\endlastfoot
	
5 & 2225 & 1357 & 622 & 246 & 0 \\
10 & 2225 & 1146 & 651 & 428 & 0 \\
15 & 2225 & 1014 & 606 & 410 & 195 \\
20 & 2225 & 766 & 690 & 558 & 211 \\
25 & 2225 & 464 & 813 & 596 & 352 \\
30 & 2225 & 353 & 690 & 634 & 548 \\
35 & 2225 & 332 & 528 & 664 & 701 \\
40 & 2225 & 248 & 485 & 641 & 851 \\
45 & 2225 & 176 & 452 & 585 & 1012 \\
50 & 2225 & 184 & 385 & 477 & 1179 \\
55 & 2225 & 195 & 341 & 404 & 1285 \\
60 & 2225 & 211 & 329 & 363 & 1322 \\
65 & 2225 & 274 & 334 & 312 & 1305 \\
70 & 2225 & 314 & 351 & 297 & 1263 \\
75 & 2225 & 322 & 388 & 310 & 1205 \\
80 & 2225 & 391 & 384 & 327 & 1123 \\
85 & 2225 & 422 & 427 & 387 & 989 \\
90 & 2225 & 412 & 485 & 384 & 944 \\
95 & 2225 & 401 & 516 & 398 & 910 \\
100 & 2225 & 377 & 566 & 404 & 878 \\
105 & 2225 & 364 & 533 & 455 & 873 \\
110 & 2225 & 302 & 552 & 504 & 867 \\
115 & 2225 & 292 & 536 & 507 & 890 \\
120 & 2225 & 259 & 532 & 486 & 948 \\
125 & 2225 & 261 & 511 & 469 & 984 \\
130 & 2225 & 275 & 460 & 474 & 1016 \\
135 & 2225 & 228 & 463 & 464 & 1070 \\
140 & 2225 & 238 & 434 & 446 & 1107 \\
145 & 2225 & 249 & 439 & 442 & 1095 \\
150 & 2225 & 256 & 422 & 427 & 1120 \\
155 & 2225 & 284 & 426 & 409 & 1106 \\
160 & 2225 & 282 & 428 & 388 & 1127 \\
165 & 2225 & 287 & 439 & 387 & 1112 \\
170 & 2225 & 320 & 435 & 386 & 1084 \\
175 & 2225 & 297 & 450 & 426 & 1052 \\
180 & 2225 & 304 & 465 & 440 & 1016 \\
185 & 2225 & 318 & 475 & 442 & 990 \\
190 & 2225 & 316 & 496 & 421 & 992 \\
195 & 2225 & 277 & 514 & 420 & 1014 \\
200 & 2225 & 294 & 497 & 425 & 1009 \\
205 & 2225 & 323 & 473 & 419 & 1010 \\
210 & 2225 & 288 & 481 & 450 & 1006 \\
215 & 2225 & 289 & 466 & 448 & 1022 \\
220 & 2225 & 266 & 485 & 455 & 1019 \\

	
	
	
	\bottomrule
\end{longtable}





\chapter{Table for the comparison of\\ all the Initial Exposure Rates }

\begin{center}
\begin{tabularx}{\textwidth}{|X|X|X|X|X|}
	\hline
	\multicolumn{5}{|c|}{Lokdown and Quarantine} \\
	\hline
	& \multicolumn{4}{c|}{Infection rate} \\
	\hline
	& NLDQ & LNQ & LDNQ & LDQ \\
	\hline
	S & 0.8126 & 0.8099 & 0.8126 & 0.8062 \\
	\hline
	E & 0.5762 & 0.4004 & 0.3654 & 0.3317 \\
	\hline
	I & 0.5501 & 0.2853 & 0.3015 & 0.2867 \\
	\hline
	R & 48.6200 & 0.4862 & 0.5973 & 0.5159 \\
	\hline
\end{tabularx}
\end{center}
\input{code}
\input{code2}
\chapter{Source Code for Implementation of \\ Lockdown and
Quarantine Area}
\scriptsize
\begin{verbatimtab}[4]
    /**
    * Name: Simple SEIR Model
    * Author: Legaspi, Myla Jean
    * Description: SEIR Agent-based Model for randomly placed agents in the whole campus
    * Tags: tutorial, gis
    */
    
    model CR4_School
    
    global {
        int basenum_human <-2225;  // This is the actual number of population in the School
        int init_human_pop <- basenum_human * congestion_rate; //initialization of population
         in the simulation with respect to the congestion rate
        int total_pop <- init_human_pop update: init_human_pop; //total population at the end
         of simulation
        int init_inf <- 0.10*init_human_pop; //initial number of infected with respect to the
         rate of infected individuals
        int init_expo <- 0.10*init_human_pop;
        
        list<people_in_rooms> list_people_in_rooms update: (room accumulate each.people_in_rooms);
        float stay_coeff update: 10.0 ^ (1 + min([ abs(current_date.hour - 17)]));
    //	float lockdown <- true update: (current_day.hour > 7 and current_day.hour <= 21);
    //	int infected <- init_inf update: people count (each.is_infected);
    //	int recovered <- 0 update: people count (each.is_recovered);
    //	int exposed <-init_expo update: people count (each.is_exposed);
    //	int susceptible <- init_human_pop - (init_inf + init_expo) update: people count
     (each.is_susceptible);
        int infected <- init_inf update: (people + list_people_in_rooms) count (each.is_infected);
        int recovered <- 0 update: (people + list_people_in_rooms) count (each.is_recovered);
        int exposed <-init_expo update: (people + list_people_in_rooms) count (each.is_exposed);
        int susceptible <- init_human_pop - (init_inf + init_expo) update:
         (init_human_pop-(exposed+infected+recovered));
    
        int cycle_number <- 0 update: cycle_number + 1;
        
       //Parameters 
        float agent_speed <- 50.0 #cm/#s;    
        float infection_distance <- 1 #m;
        float exposed_distance <- 2 #m;
        float min_distance <- 1.0 #m;
        float protection_rate <- 0.50;
        float congestion_rate <- 1.0;  
        float proba_infection <- 1.0 - protection_rate;
        float proba_exposed <- 0.25;
       float step <- 10 #minutes;
       int current_day <-1 update:current_date.hour = 0 and current_date.minute = 0 ? current_day + 1: current_day;
       bool is_night <- true update: current_date.hour < 7 or current_date.hour >= 18;
       bool lockdown <- true update:(current_date.hour > 7 and current_date.hour <=17);
        file hallways_shapefile <- file("../includes/hallways.shp");
        file road_shapefile <- file("../includes/roaad.shp");	
        file rooms_shapefile <- file("../includes/nqbuilding.shp");
        file space_shapefile <- file("../includes/space.shp");	
        file qroom_shapefile <- file("../includes/qua.shp");
        graph road_network;	
        //file road_shapefile <- file("../includes/road5.shp");		
        geometry shape <- envelope(hallways_shapefile);	
        graph the_graph;
        
        int min_work_start <- 9;
        int max_work_start <- 10;
        int min_work_end <- 14; 
        int max_work_end <- 15; 
        int min_start_class <-7;
        int max_start_class <-7;
        int min_end_class <-16;
        int max_end_class <-20;
        int min_start_break <- 9;
        int max_start_break <- 9;
        int min_end_break<-10;
        int max_end_break<-11;
        int min_start_lunch <-12;
        int max_start_lunch <-12;
        int min_end_lunch <- 13;
        float max_end_lunch <- 13.5;
        
        //float min_speed <- 0.001 #km / #h;
        //float max_speed <- 0.002 #km / #h;
        
        init {
            /* Create the simulation space from the shapefiles */
            create hallway from: hallways_shapefile;
            create room from: rooms_shapefile;
            //create openfield from: field_shapefile;
            create road from: road_shapefile ;
            road_network <- as_edge_graph(road);
            create space from: space_shapefile;
            //the_graph <- as_edge_graph(road);
            create qroom from: qroom_shapefile;
            
            create people number: init_human_pop {
                assigned_room <- one_of(room);
                   location <- any_location_in(assigned_room);
                   
                   
                   //speed <- rnd(min_speed, max_speed);
                start_work <- rnd (min_work_start, max_work_start);
                end_work <- rnd(min_work_end, max_work_end);
                class_start <- rnd(min_start_class, max_start_class);
                class_end <- max_end_class;
                break_start <- min_start_break;
                break_end <- min_end_break;
                lunch_start <- min_start_lunch;
                lunch_end <- min_end_lunch;
                
                speed <- agent_speed;
                
            }//end create people 
            
            ask init_inf among people {
               is_infected <- true;
               is_susceptible <- false;
               is_recovered <- false;
               is_exposed <-false;   
               is_immuned <- false;  
               
               infected_time <- 0;
               infectious_period <- rnd(1440, 2016); //10-14 days of being infectious
            }//end ask init inf   
            
            ask init_expo among people{
                is_exposed <- true;
                is_infected <-false;
                is_susceptible <- false;
                is_recovered <- false;
                is_immuned<- false;
                exposed_time <- 0;
                exposed_period <-rnd(288,1440); 
            }
            
       }//end init 
       
    /* Behavior  */
    
        species people skills: [moving] {
            bool is_susceptible <- true;
            bool is_infected <- false;
            bool is_recovered <- false;
            bool is_exposed <-false;
            bool is_immuned <- false;
            float range <- 3.0 #m;
            list<people> neighbors;
            bool social_distancing <- false;
            float distance <- 1.0;
            float agent_speed <- 50 #cm/#s;
            int infectious_days <- 0;
            int exposed_days <-0;
            int infectious_period <- rnd(1440, 2016);
            int exposed_period <- rnd(288,1440);
            int recovered_days <-0;
            int recovered_period <- rnd(288,1440);
            int expose_counter <-0;
            int exposed_time <-0;
            int infected_time <- 0;
            int recovered_time <- 0;
            room assigned_room;
            hallway all_hallways;
            
            bool is_night <- true;
            bool can_wander_outside <- true;
            point the_target <- nil ;
            int start_work ;
            int end_work  ;
            int class_start;
            int class_end;
            int break_start;
            int break_end;
            int lunch_start;
            int lunch_end;
            
            int stay_counter;
        
            
            point target;
            
    //	reflex time_to_class when: current_date.hour = class_start{
    //		do wander;
    //		
    //		target <- any_location_in (assigned_room);
    //	}
    //	reflex time_to_break when: current_date.hour = break_start{
    //		do wander;
    //		do goto target: target on:road_network;	
    //	}
    //	reflex time_to_class_afterbreak when: current_date.hour = break_end{
    //		do wander;
    //		target <- any_location_in (assigned_room);	
    //	}
    //	reflex time_to_lunch when: current_date.hour = lunch_start{
    //		do wander;
    //		do goto target: target on:road_network;	
    //	}
    //	reflex time_to_class_afterlunch when: current_date.hour = lunch_end{
    //		do wander;
    //		
    //		target <- any_location_in (assigned_room);
    //	}	
    //	reflex time_to_go_home when: current_date.hour = class_end{
    //		do wander;
    //		do goto target: target on: road_network;
    //	} 
    //	reflex night when: is_night = true{
    //		do wander;
    //		target <- any_location_in (assigned_room);
    //	}
    
    //	reflex lockdown when:lockdown {
    //		target <- nil;
    //		stay_counter <- 0;
    //		if flip(stay_counter/stay_coeff){
    //			target <- any_location_in (one_of(room));
    //		}
    //	}
        reflex lockdown when: lockdown{
            target <- nil;
            stay_counter <-0;
            if flip(stay_counter/ stay_coeff){
                target <- any_location_in(one_of(room));
            }
        }
        
        reflex move when: target != nil {
            do goto target: target on: road_network;
            if (location = target) {
                if (is_night = true){
                    agent_speed <- 0.0;
                    target <- nil;
                    stay_counter <- 0;
                } else{
                    target <- nil;
                    stay_counter <- 0;
                }
                
            }
            
    
        }
        reflex stay when: target = nil{
            if(not lockdown){
                stay_counter <- stay_counter + 1;
                if flip (stay_counter/stay_coeff){
                    if(is_infected){
                        agent_speed <- 0.0;
                        if ((cycle_number - infected_time >= 144) and (cycle_number - infected_time <= 2016)){
                            target <- any_location_in(one_of(qroom));
                        } else {
                            target <- any_location_in(one_of(room));
                        }
                    }
                } else {
                    if (is_infected){
                        if ((cycle_number - infected_time >= 144) and (cycle_number - infected_time)){
                            target <- any_location_in(one_of(qroom));
                        }
                    }else if (is_recovered){
                        target <- any_location_in(one_of(room));
                    }
                }
            }
        }
        reflex exposed when: is_infected{
                ask people at_distance exposed_distance {
                    if (is_immuned = true){
                        
                        is_susceptible <- false;
                        is_exposed <- false;
                        is_infected <- false;
                    }else {
                        expose_counter <- expose_counter + 1;
                        if(is_susceptible = true){
                            if (expose_counter >= 2 and flip(proba_exposed)){
                            is_exposed<-true;
                            is_susceptible<-false;
                            is_recovered <- false;
                            is_infected <- false;
                            is_immuned <- false;
                            
                            float immunity <- rnd(0.0, 1.0);
                               exposed_time <- cycle_number;
                               exposed_period <-rnd(288,1440); //2 days - 10 days incubation
                                perios, before maging infected or not.
                               //write "exposed:" +people + "exposed time" + cycle_number/144;
                            } //end inner if
                        }
                    } //end if else
                    
                    
                        
                 }//end ask people at distance
            }//end exposed; agents to be exposed
               
    //		reflex days_of_exposure when: is_exposed {
    //			if (is_exposed){
    //		   		exposed_days <- cycle_number - exposed_time;
    //		   		//write "exposed days:" + exposed_days/144;
    //		   	}// end if	
    //		}//end reflex days of exposure; exposure counter 
    //			
            reflex infect when: is_exposed {
                ask people at_distance infection_distance {
                    if(is_exposed=true){
                        if ((cycle_number - exposed_time) >= exposed_period){
                        float immunity <- rnd(0.0,1.0);
                        //write "Immunity" + immunity;
                        if (immunity < proba_infection) {
                            is_infected <- true;
                            is_susceptible <- false;
                            is_exposed <-false;
                            is_recovered <-false;
                            is_immuned <- false;
                            
                            infected_time <- cycle_number;
                            infectious_period <- rnd(2016,2736) ; //14 -19  days
                            //write "infected:" + people;
                        }//end inner if
                        else {
                            is_exposed <-false;
                            is_infected <-false;
                            is_susceptible <- true;
                            is_recovered <-false;
                            is_immuned <- false; 
                        }
                    }//end outer if	
                    }
                    
                    
                }//end people at distance;
                 computes if iyong distance ng neighbor is pasok sa infection distance 
            } //end reflex
            
    //		   
    //		reflex days_of_infection when: is_infected {
    //			if (is_infected){
    //				infectious_days <- cycle_number - infected_time;
    //		   		//write "infectious days:" + infectious_days;
    //		   	} //end if
    //		}// end reflex days; day of infection counter
    //	
    //	
            reflex recovered when: is_infected {
                if ((cycle_number - infected_time) >= infectious_period) {  
                	// Use >= instead of =
                    is_susceptible <- false;
                    is_recovered <- true;
                    is_infected <- false;
                    is_exposed <-false;
                    is_immuned <-true;
                    recovered_days <- 0;
                        
                     recovered_time <- cycle_number;
                    // Assuming recovered_period should be based on some parameter,
                     adjust accordingly
                    recovered_period <- rnd(2880, 8640); // Set a meaningful value or use a parameter 
                    //write "Recovered: " + self; // Use 'self' to refer to the current agent
                }//end if
            }//end recovered
                
    //		reflex days_of_recovery when: is_recovered {
    //			recovered_days <- cycle_number -  recovered_time;
    //		}//end days of recovery
                
            reflex susceptible when: is_recovered {
                if ((cycle_number - recovered_time) >= recovered_period) {  // Use >= instead of =
                    is_susceptible <- true;
                    is_exposed <- false;
                    is_infected <- false;
                    is_recovered <- false;
                    is_immuned <- false;
                    
                }//end if
            }//end reflex
        
        aspect circle {
            draw circle(0.2) color: is_exposed ? #yellow : (is_infected ? #red :
             (is_recovered ? #green : (is_susceptible ? #green : #blue)));
            //draw triangle(0.2) color: is_exposed ? #yellow : (is_infected ? #red :
             (is_recovered ? #green : (is_susceptible ? #green : #blue)));
        }	
        
            aspect sphere3D {
            draw sphere(0.2) at: {location.x, location.y, location.z + 2} color:
             is_susceptible? #blue :(is_exposed ? #yellow : (is_infected ? #red :
              (is_recovered ? #green: #green)));
        
        }
            
        }//end species students
        
        species space{
            aspect default{
                draw shape color: #white border: #black;
            }
        }
        species intersection skills: [intersection_skill] ;
        species hallway{
            geometry display_shape <- shape + 0.5;
            aspect default{
                draw display_shape color: #tan depth: 0.5;
            }
        } // end hallway
        
    //	species room{
    //
    //		float height <- rnd(5#m, 7#m);
    //		aspect default{
    //			draw shape color: #gray border: #black depth: height;
    //		}
    //	} // end room
        
        species road{
            geometry display_shape <- shape + 0.5;
            aspect default{
                draw display_shape color: #black depth: 0.5;
            }
        }
    species qroom{
        float height <- 5#m;
        
        aspect default {
            draw shape color: #pink border: #black depth: height;
        }
    }
        species room{
            int num_susceptible <- 0 update: self.people_in_rooms count each.is_susceptible;
            int num_exposed <- 0 update: self.people_in_rooms count each.is_exposed;
            int num_infected <- 0 update: self.people_in_rooms count each.is_infected;
            int num_recovered <- 0 update: self.people_in_rooms count each.is_recovered;
            int num_total <- 0 update: length(self.people_in_rooms);
            float height <- rnd(5 #m, 10 #m);
            
            species people_in_rooms parent: people schedules: [] {
            }
            
            reflex people_leave {
                ask people_in_rooms {
                    stay_counter <- stay_counter + 1;
                }
            
                release people_in_rooms where (flip(each.stay_counter / stay_coeff)) as:
                 people in: world {
                    if(is_infected and (is_night = true)){
                        agent_speed <- 0.0;
                        if((cycle_number - infected_time >= 144) and (cycle_number - infected_time <= 2016)){
                            target <- any_location_in(one_of(qroom));
                        } else{
                            target <- any_location_in(one_of(room));
                        }
                    } else if(is_infected and (is_night =false)){
                        //agent_speed <- 0.0;
                        if((cycle_number - infected_time >= 144) and (cycle_number - infected_time <= 2016)){
                            target <- any_location_in(one_of(qroom));
                        } else{
                            target <- any_location_in(one_of(room));
                        }
                        
                    } else{
                        target <- any_location_in(one_of(room));
                    }
                }
            }
            reflex let_people_enter {
            capture (people inside self where (each.target = nil)) as: people_in_rooms;
        }
    //		reflex let_people_enter {
    //		capture (people inside self where (each.target = nil)) as: people_in_rooms;
    //	}
        aspect default {
            draw shape color: num_total = 0 ? #gray : (float(num_infected) / num_total > 0.5
             ? #red : #green) border: #black depth: height;
        }
        }//end species room
        
        
        float susceptible_rate <- 1.0 update: susceptible / init_human_pop;
        float exposed_rate update: exposed / init_human_pop;
        float infection_rate update:infected/init_human_pop;
        float recovery_rate update: recovered/init_human_pop;
        
        int highest_susceptible_rate_day <- 0;
        float highest_susceptible_rate <- 0.0;
        reflex susceptible_rate when: susceptible_rate > highest_susceptible_rate{
            highest_susceptible_rate <- susceptible_rate;
            highest_susceptible_rate_day <- cycle_number/144;
        }	
        
        int highest_exposure_rate_day <- 0;
        float highest_exposure_rate <- 0.0;
        reflex exposed_rate when: exposed_rate > highest_exposure_rate{
            highest_exposure_rate <- exposed_rate;
            highest_exposure_rate_day <- cycle_number/144;
        }	
        
        
        int highest_infection_rate_day <- 0;
        float highest_infection_rate <- 0.0;
        reflex infection_rate when: infection_rate > highest_infection_rate{
            highest_infection_rate <- infection_rate;
            highest_infection_rate_day <- cycle_number/144;
        }
        
        int highest_recovery_rate_day <- 0;
        float highest_recovery_rate <- 0.0;
        reflex recovery_rate when: recovery_rate > highest_recovery_rate{
            highest_recovery_rate <- recovery_rate;
            highest_recovery_rate_day <- cycle_number/144;
        }
        
        reflex end_simulation when: cycle_number = 31680 #cycles{
            do pause;
        }//end cycle number counter
        
        reflex print {
            if(current_date.hour = 0 and current_date.minute = 0){
                write "" + (current_day - 1) + " & " + init_human_pop + " & " + susceptible +
                 " & " + exposed + " & " + infected + " & " + recovered + " ";	
            }
            if(cycle_number = 31680){
                write "\\\\hline";
                write "\\\\textbf{ " + "S - " + (highest_susceptible_rate) + "} & " +
                 highest_susceptible_rate_day + " \\\\ ";
                write "\\\\textbf{ " + "E - " + (highest_exposure_rate) + "} & " +
                 highest_exposure_rate_day + " \\\\ ";
                write "\\\\textbf{ " + "I - " + (highest_infection_rate) + "} & " +
                 highest_infection_rate_day + " \\\\ ";
                write "\\\\textbf{ " + "R - " + (highest_recovery_rate) + "} & " +
                 highest_recovery_rate_day + " \\\\ ";
            }
        }
    
    }//end global
    
    experiment main_experiment type: gui {
        parameter "Initial Human Population" var: init_human_pop;
        parameter "Infection distance" var: infection_distance;
        parameter "Exposure distance" var: exposed_distance;
        parameter "Proba infection" var: proba_infection min: 0.0 max: 1.0;
        parameter "Proba exposure" var: proba_exposed min: 0.0 max: 1.0;
        parameter "Nb people infected at init" var: init_inf ;
        //parameter "Social Distancing" var: social_distancing;
        
        output {
            monitor "Initial Population" value: init_human_pop;
            //monitor "Current Population" value: total_pop;
            monitor "Number of susceptible agents" value: susceptible;
            monitor "Number of exposed agents" value: exposed;
            monitor "Number of infected agents" value: infected;
            monitor "Number of recovered agents" value: recovered;
            
            monitor "Susceptible rate:" value: susceptible_rate;
            monitor "Exposure rate:" value: exposed_rate;
            monitor "Infection rate:" value: infection_rate;
            monitor "Recovery rate:" value: recovery_rate;
            
            monitor "Current day" value: cycle_number/144;
            monitor "Current time:" value: current_date.hour;
            
            monitor "Highest susceptible rate" value: highest_susceptible_rate;
            monitor "Highest exposure rate" value: highest_exposure_rate;
            monitor "Highest infection rate" value: highest_infection_rate;
            monitor "Highest recovery rate" value: highest_recovery_rate;
            
            
            monitor "Highest susceptible rate day" value: highest_susceptible_rate_day;
            monitor "Highest exposure rate day" value: highest_exposure_rate_day;
            monitor "Highest infection rate day" value: highest_infection_rate_day;
            monitor "Highest recovery rate day" value: highest_recovery_rate_day;
            
            display view3D type: opengl antialias: false {
                light #ambient intensity:20;
                light #default intensity:(is_night ? 127:255);
                // 'default' aspect is used automatically  
                species space;
                //species road;
                species hallway ;
                
                 
                species road;
                species people aspect: sphere3D; 
                species room transparency: 0.3;
                species qroom transparency: 0.3;
                         
            }//end map
            
            display SI type: java2D refresh: every(20  #cycles) {
                chart "SEIR Rate" type: series 
                x_label: "TIME"
                y_label: "RATE"
                x_tick_line_visible: false
                y_range: [0,1.0] // can be modified depending
                                 // maximum rate
                {
                    // data "susceptible" value: susceptible_rate color: #blue ;
                    
                    data "infected" value: infection_rate color: #red;
                    data "exposed" value: exposed_rate color: #yellow;
                    data "recovered" value: recovery_rate color: #green ;
                    data "susceptible" value: susceptible_rate color: #blue ;
                    
                }
            }
        }
     

    
    

\end{verbatimtab}
\normalsize

\end{document}
\chapter{Introduction}
\label{chap:intro}


\section{Background of the Study}

Since the first COVID-19 case was recorded in the Philippines on January 30, 2020, discussions about implementing a lockdown ensued. By the second week of March 2020, localized lockdowns were initiated, leading to the temporary closure of academic institutions by the Philippine government\cite{worldhealthorganization_2023_who}. This closure affected nearly 24 million Filipino students, compelling them to adapt to alternative modes of instruction such as modular and distance learning. However, challenges arose as a significant portion of the Filipino population lacked access to reliable internet connections, disparities in educational access and quality.

Despite school closures being a preventive measure to curb the spread of the virus, it became increasingly evident that such actions had adverse effects on students' learning. Reports indicated difficulties in maintaining engagement and academic progress, particularly among students from marginalized communities. The closure also highlighted the importance of schools not only as centers of education but also as providers of essential services such as nutrition and psycho social support.

Over the course of two years, from the initial one-week suspension to an extended period of lockdown, it became increasingly apparent that schools played a significant role in the transmission of COVID-19, particularly in densely populated areas with limited infrastructure for infection control \cite{jones_2017_school}.

As the fear of COVID-19 gradually diminished, there was a phased reopening of schools. Safety measures including vaccinations and Non-Pharmaceutical Interventions (NPIs) such as social distancing and lockdowns were implemented to mitigate transmission. However, some regions experienced a rise in infections following the reopening of schools. For instance, in Spain, the second wave of infections began before the commencement of the school year in September 2020, underscoring the complexities of managing public health risks in educational settings \cite{soriano_2021_main}.

Given these circumstances, it raises critical questions about the readiness of Philippine public schools in managing another outbreak of a respiratory infectious disease, including the adequacy of infrastructure, resources, and policies to safeguard the health and well-being of students, teachers, and staff amidst evolving public health challenges.

\section{Statement of the Problem}

In the Philippine educational context, the prevalence of congestion and overpopulation within classrooms presents a significant challenge in preventing the transmission of respiratory infectious diseases. The close proximity of students and staff due to limited space and high student-to-classroom ratios increases the risk of disease spread within school environments \cite{g_2023_facility}. Despite efforts to implement preventive measures such as vaccinations, mask-wearing, and promoting physical distancing, the persistent issue of congestion and overpopulation exacerbates the vulnerability of the school community to respiratory infectious diseases.

Therefore, there is a critical need to address the issue of congestion and overpopulation within Philippine classrooms to mitigate the risk of respiratory infectious disease transmission. By identifying effective strategies to alleviate congestion and optimize classroom space utilization, schools can create safer environments that promote the health and well-being of students and staff \cite{g_2023_facility}. Additionally, exploring innovative approaches to complement existing preventive measures is essential in ensuring the overall safety and resilience of the Philippine school community against respiratory infectious diseases

\indent \indent The data for population were collected from Pines City National High School. In this study, it aims to answer the following objectives:

\section{Objective of the Study}
\vspace{8pt} %
\indent \indent In this study, the model will categorize the population into five compartments: susceptible (\textbf{S}), exposed (\textbf{E}), infected (\textbf{I}), and recovered (\textbf{R}) agents. For each experiment, the model will track the number of individuals in each compartment at every iteration of the simulation (t). Different scenarios will be implemented in each simulation to test their respective effects. Through this approach, the study aims to address the following general and specific objectives.

\subsection{General Objective of the Study}

\indent \indent 
 This study aims to comprehensively understand the dynamics of respiratory infectious disease transmission within the school environment and identify key factors influencing disease spread, including individual protection rates, exposure rates, and the implementation of lockdown and quarantine areas, in order to inform the development of effective prevention and control strategies.
 
\subsection{Specific Objective of the Study}

\indent \indent
Specifically, this study aims to :
\begin{itemize}
	
	\item[1.] Assess the impact of Varying Protection Rates (0\%, 25\%, 50\%, 75\%, 100\%) on Disease Transmission within the PCNHS environment; 
	\item [2.] Examine how having different initial percentages of exposed individuals (0\%, 25\%, 50\%, 75\%) within the PCNHS environment contribute to the overall respiratory infectious disease transmission rates; and,
	\item[3.] Evaluate the effect of lockdown and quarantine area implementation on the transmission of respiratory infectious disease within the PCNHS environment 
\end{itemize}


\section{Significance of the Study}

Through agent-based modeling and compartmental modeling, we can replicate the behavior of respiratory infectious disease such as the COVID-19 virus within a school environment. This study focuses on investigating the transmission dynamics within the campus of PCNHS.

The study holds significant importance in understanding and addressing respiratory infectious disease transmission within school environments. 
Firstly, by assessing the impact of varying protection rates ranging from 0\% to 100\%, the study provides valuable insights into the effectiveness of immunity measures in controlling disease spread. This information is crucial for policymakers and school administrators in implementing appropriate preventive measures to safeguard the health of students and staff. 
Additionally, the examination of different initial percentages of exposed individuals within the school environment contributes to understanding overall disease transmission rates, aiding in the assessment of outbreak risks and the determination of necessary interventions. Furthermore, the evaluation of lockdown and quarantine measures within the school setting sheds light on the effectiveness of non-pharmaceutical interventions in controlling disease transmission. This information is instrumental in developing strategies to minimize disease spread and maintain a safe learning environment. 
Overall, the study's findings have significant implications for public health policy and school management practices, providing evidence-based insights into effective strategies for preventing and managing respiratory infectious diseases within school settings, ultimately contributing to efforts to protect the health and well-being of students, staff, and the broader community.

\section{Scope and Limitation}

The study focuses on assessing respiratory infectious disease transmission specifically within public school settings, with a particular emphasis on PCNHS, which serves as the virtual environment for the simulation. Utilizing an agent-based and compartmental modeling approach, the study allows for the simulation of disease transmission dynamics within the school environment. It explores the impact of parameters such as protection rate, exposure rate, and the implementation of lockdown and quarantine measures on disease transmission. Through scenario analyses and simulations, the study aims to provide insights into the effectiveness of various intervention strategies in mitigating disease spread within the school setting.

However, it's important to note several limitations. Firstly, the data and findings of this study may only be directly applicable to the specific context of PCNHS. Differences in school layout, student demographics, and local environmental factors may lead to variations in disease transmission dynamics between schools. Additionally, the study does not consider outside factors that may contribute to transmissions, such as community transmission rates, prevalence of the disease in the surrounding area, and effectiveness of public health measures implemented outside the school setting. Understanding these limitations is crucial for interpreting the findings accurately and for making informed decisions regarding disease control and prevention strategies in similar settings.




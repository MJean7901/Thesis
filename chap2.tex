\chapter{Review of Related Literature}
\label{chap:Review of Related Literature}
 
 \section{Respiratory Infectious Diseases}

Respiratory infectious diseases (RIDs) are infections of the respiratory tract, which includes the lungs, airways, and other organs that are involved in breathing\cite{a2011_infectious}. RIDs are one of the most common human infections in all age groups and an important cause of mortality and morbidity worldwide. RIDs can be caused by a variety of pathogens, including viruses which are tiny infectious agents that live on organisms by replicating, bacteria are defined as single-celled microorganisms, that can decide grow and spread in the body, fungi are microorganisms that have cell walls made from a substance called chitin, and parasites that feeds on another organism to survive \cite{a2020_covid19}. Some common examples of RIDs include Pneumonia, an inflammation of the lung tissue, which can be caused by bacteria, viruses, or fungi. Tuberculosis, is a chronic infectious disease caused by the bacterium Mycobacterium tuberculosis\cite{a2011_infectious}. Influenza, is a respiratory illness caused by the influenza virus. Bronchitis: Inflammation of the bronchi, which are the tubes that carry air to and from the lungs. Sinusitis: Inflammation of the sinuses, which are air-filled cavities in the skull. RIDs can be spread through a variety of ways, including:

\begin{itemize}
	\item Airborne transmission: RIDs can be spread through the air when an infected person coughs or sneezes \cite{a2011_infectious}
	\item Droplet transmission: RIDs can be spread through contact with droplets from an infected person's cough or sneeze.
	\item Contact transmission: RIDs can be spread  through contact with an infected person's secretions, such as saliva or mucus.
	\item Fomite transmission: RIDs can be spread through contact with contaminated objects, such as doorknobs or towels. 
\end{itemize} 
The symptoms of RIDs can vary depending on the type of infection and the severity of the illness. However, some common symptoms of RIDs includes cough, fever, shortness of breath, chest pain, runny nose, sinus pain, sore throat, and, muscle aches\cite{a2011_infectious}. 

\subsection{COVID-19}
The latest of another respiratory infectious disease is COVID-19. Although coronaviruses have been causing human infections since the 1960s, it wasn't until the last two decades that their potential to trigger deadly epidemics became evident. COVID-19 marks the third major outbreak of a respiratory disease caused by a coronavirus within twenty years, and it has had a profound impact on the global socioeconomic equilibrium . It was recognized in December 2019 in Wuhan, Hubei Province, China\cite{a2020_what}. Through genome analysis, which studies an organism's genome, it was shown that it is a novel coronavirus related to SARS-CoV. With this discovery, it was named severe acute respiratory syndrome coronavirus 2 (SARS-CoV-2). On January 30, 2020, the World Health Organization declared the COVID-19 outbreak a Public Health Emergency of International Concern, and then, on March 11, 2020, it escalated the classification to a pandemic. \cite{worldhealthorganization_2020_covid19}\\

The source, transmission, and severity of 2019-nCoV are still mostly shrouded in uncertainty. Initially, a significant number of early infected patients were associated with the Huanan seafood wholesale market in Wuhan, China. Nevertheless, 13 out of the 41 early cases had no discernible connection to the market. Other sources and animals were also examined, like bats\cite{king_2011_virus}. SARS-CoV-2 or COVID-19 is a zoonotic virus with the ability to transmit from animals to humans and also among humans through airborne aerosols, a type of transmission that conveys infectious diseases through small particles suspended in the air. Specifically, the novel coronavirus exhibits a notably high rate of human-to-human \cite{agentbased} transmission, resulting in a broad range of clinical symptoms among infected patients.\\ \cite{explanation}

Having a high rate of human-to-human transmission has caused it to spread rapidly worldwide. As of this writing, there are 772,190,439 confirmed cases in the world, with a global death toll of 6,961,001 . The primary mode of human-to-human transmission of SARS-CoV-2 occurs when individuals are in close proximity to an infected person, typically through exposure to coughing, sneezing, respiratory droplets, or aerosols.


\section{Public Schools Health Protocol for COVID-19}   
After more than two years of school closures, a multitude of public school students are poised to return to classrooms this academic year following the Department of Education's (DepEd) directive to transition to five-day, in-person classes by November 2. However, school administrators may face considerable challenges in planning to avert overcrowding once students gradually resume on-campus attendance \cite{ditsworth_2019_infection}.

Nationwide, public schools could potentially face a shortage of approximately 400,000 classrooms if they adhere to a class size limit of 20 students to uphold physical distancing measures. This estimation is based on enrollment data from the 2020-2021 school year and information from the National School Building Inventory from the 2019-2020 school year \cite{chi_deped_nodate}.

In accordance with DepEd's guidelines for minimum health standards in schools during the initial year of the pandemic, as outlined in DepEd Order No. 14, s. 2020, students are expected to maintain a physical distance of one meter apart, with classroom occupancy ranging from 18 to 21 students, contingent upon the utilization of desks or armchairs. On the other hand, establishing a maximum class size of 40—similar to the typical pre-pandemic class size in public schools—would resolve classroom shortages for all regions except Calabarzon, NCR, and BARMM \cite{chi_2022_deped}.

Even if public schools opt to abandon the one-meter apart rule and accommodate 40 students per classroom, an estimated additional 10,000 classrooms would still need to be constructed. For example, NCR would still face a deficit of 5,793 classrooms, while Calabarzon would require an additional 3,624 classrooms. Additionally, BARMM would necessitate approximately 500 more classrooms.

In the Cordillera region, recent data from DepEd-CAR's Education Support Services Division indicates a requirement for 1,604 elementary classrooms, 1,381 secondary classrooms, and 121 integrated school classrooms. Mountain Province exhibits the highest deficit, with shortages of 355 elementary classrooms and 425 secondary classrooms, as per enrollment figures for 2022 \cite{g_2023_facility}.

According to Engineer Christopher Hadsan, the head of the Education Facilities Section, budget allocations for the region in past years and for the upcoming year 2023 have been inadequate to address the shortfall. Their most recent estimation suggests that constructing a classroom measuring 7×9 meters, capable of accommodating 40 to 45 students, incurs a cost of P3.5 million \cite{jones_2017_school}.

The simulation environment selected for this study is Pines City National High School, situated on Palma Street in Baguio City. As one of the most populous public high schools in the area, it accommodates a total of 2,136 students from grades 7 to 10, along with 89 teaching and non-teaching staff, totaling 2,225 individuals. Each classroom at the school accommodates between 40 and 55 students, which contravenes the DepEd protocol stipulating a maximum of 18 to 21 students per classroom to mitigate the transmission of COVID-19.

\section{Agent Based Modeling}

Agent-Based Modeling (ABM) belongs to a category of simulations where every member of a population is portrayed as a unique entity or "agent." These agents possess individual traits and behaviors and engage in interactions with one another, fostering opportunities for the transmission of contagions. Investigating epidemics is crucial, not just for comprehending outbreaks, but also for tackling associated social issues like governmental instability, crime, poverty, and inequality \cite{castiglione_2009_agent}.

\subsection{ABM for COVID-19 Transmission}

In a study developoed by Rojhun O. Macalinao, Jcob C. Malaguit, and Destiny S. Lutero developed an agent-based model to study COVID-19 transmission within Philippine classrooms. Utilizing NetLogo, their model examines the impact of student and teacher interactions on virus spread across four classroom layouts. The model features two types of agents, teachers, and students, each categorized as susceptible or infected. Agents exhibit mobility, and if a susceptible agent is within a two-seat node radius of an infected one, transmission can occur. Additionally, class rotations are incorporated, allowing students to move between classrooms. Macalinao et al.'s findings indicate that within-classroom mobility and class rotations contribute to increased COVID-19 transmission rates. Moreover, increasing the number of students per classroom layout correlates with higher infection rates within classrooms \cite{abm_covid19}.
\\
Another application of Agent Based Modeling is the study conducted by Christian Alvin H. Buhat, et. al. They developed an agent-based model and compartmental model (SEIR) to simulate the spread of respiratory infectious diseases between two neighboring cities. The study incorporates preventive measures such as social distancing and lockdowns within a city, along with the impact of protective measures. Factors including the likelihood of travel between cities and within them during lockdowns, as well as the initial percentage of exposed and infected individuals, influence the increase in newly-infected cases in both models. Results from simulations indicate several key findings: (i) An increase in exposed individuals correlates with a rise in new infections, underscoring the importance of enhanced testing and isolation efforts. (ii) Protective measures with effectiveness levels of 75-100\% significantly impede disease transmission. (iii) Travel within and between cities may be feasible under strict preventive measures, such as non-pharmaceutical interventions. (iv) Implementing lockdowns in neighboring cities during periods of high disease transmission risk, while emphasizing social distancing and protective measures, is optimal. Both the agent-based and compartmental models exhibit similar qualitative dynamics, with differences attributed to spatio-temporal heterogeneity and stochasticity. These models offer valuable insights for policymakers in formulating infectious disease-related policies aimed at safeguarding individuals while facilitating population movement. \cite{unknown}
\\
For other disease other than COVID-19, Agent Based Model was also used to asses the influenza interactions at the host level. They describe a novel agent-based model (ABM) of influenza transmission during interaction with another respiratory pathogen. The ABM produces authentic data for both pathogens, encompassing weekly incidences of PI cases, carriage rates, epidemic size, and epidemic timing. Notably, varied interaction hypotheses yielded diverse transmission patterns, leading to significant fluctuations in the associated PI burden. Additionally, the interaction strength played a pivotal role: instances where influenza heightened pneumococcus acquisition saw 4–27\% of the PI burden during the influenza season attributed to influenza, contingent upon the interaction strength. \cite{arduin-2017}

\section{Gama Platform}
For the purpose of building spatially explicit agent-based simulations, GAMA is an open-source modeling and simulation platform that is simple to use. It was designed to be used in any application domain. GAMA users have created models for usage in a variety of application domains, including urban mobility, epidemiology, disaster evacuation strategy design, and climate change adaptation.\\ \cite{unknown-author-no-date}

The high level of openness that accompanies the generality of the agent-based approach that GAMA promotes is demonstrated, for instance, by the creation of plugins tailored to particular requirements or by the ability to invoke GAMA from other programs or languages (like R or Python). The more than 2000 users of GAMA can use it for a wide range of applications due to its openness: communication tools, serious games, negotiation help, scientific simulation, scenario exploration and visualization.\\

Based on the RCP framework made available by Eclipse, GAMA is a single application. Using this one program, which is also called a platform, users can perform the majority of modeling and simulation tasks—such as editing models and simulating, visualizing, and exploring them with specialized tools—without the need for additional third-party software.